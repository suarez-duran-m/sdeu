\documentclass[twoside, final, 10pt]{articleMine}
\usepackage[english]{babel}
\usepackage[sc]{mathpazo}
\usepackage{a4wide}
\usepackage{subfigure}

\usepackage{hyperref}
\usepackage{amsmath,amssymb}
\usepackage[right]{lineno}
\usepackage{xspace}
\usepackage{accents}
\usepackage{graphicx}
\usepackage{booktabs}
\usepackage{color}
\usepackage{units}
\usepackage{enumitem}
\usepackage{todonotes}
\usepackage[capitalize]{cleveref}
\usepackage{nameref}
\usepackage{csquotes}

\usepackage{listings}
\usepackage{color}

\definecolor{dkgreen}{rgb}{0,0.6,0}
\definecolor{gray}{rgb}{0.5,0.5,0.5}
\definecolor{mauve}{rgb}{0.58,0,0.82}

\lstset{frame=tb,
  language=XML,
  aboveskip=3mm,
  belowskip=3mm,
  showstringspaces=false,
  columns=flexible,
  basicstyle={\small\ttfamily},
  numbers=none,
  numberstyle=\tiny\color{gray},
  keywordstyle=\color{blue},
  commentstyle=\color{dkgreen},
  stringstyle=\color{mauve},
  breaklines=true,
  breakatwhitespace=true,
  tabsize=3
}

\usepackage[thinc]{esdiff}


\usepackage{booktabs}
\usepackage{mcite}


\graphicspath{{plots/}}
\newcommand*\patchAmsMathEnvironmentForLineno[1]{%
  \expandafter\let\csname old#1\expandafter\endcsname\csname #1\endcsname
  \expandafter\let\csname oldend#1\expandafter\endcsname\csname end#1\endcsname
  \renewenvironment{#1}%
     {\linenomath\csname old#1\endcsname}%
     {\csname oldend#1\endcsname\endlinenomath}}%
\newcommand*\patchBothAmsMathEnvironmentsForLineno[1]{%
  \patchAmsMathEnvironmentForLineno{#1}%
  \patchAmsMathEnvironmentForLineno{#1*}}%
\AtBeginDocument{%
\patchBothAmsMathEnvironmentsForLineno{equation}%
\patchBothAmsMathEnvironmentsForLineno{align}%
\patchBothAmsMathEnvironmentsForLineno{flalign}%
\patchBothAmsMathEnvironmentsForLineno{alignat}%
\patchBothAmsMathEnvironmentsForLineno{gather}%
\patchBothAmsMathEnvironmentsForLineno{multline}%
}

\newcommand{\putat}[3]{\begin{picture}(0,0)(0,0)\put(#1,#2){#3}\end{picture}} % just a shorthand
\parskip 4.2pt           % sets spacing between paragraphs
%\parindend 0pt           % sets spacing between paragraphs
\def\Offline{\mbox{$\overline{\rm
Off}$\hspace{.05em}\raisebox{.4ex}{$\underline{\rm line}$}}\xspace}
\def\OfflineB{\mbox{$\bf\overline{\rm\bf
Off}$\hspace{.05em}\raisebox{.4ex}{$\bf\underline{\rm\bf line}$}}\xspace}

\newcommand{\qpkDist}{$Q^\mathrm{Pk}_\mathrm{Dist}$\,}
\newcommand{\qpkNormDist}{$Q^\mathrm{Pk}_\mathrm{NormDist}$\,}

% equations ...
\newcommand{\be}{\begin{equation}}
\newcommand{\ee}{\end{equation}}
\newcommand{\ben}{\begin{enumerate}}
\newcommand{\een}{\end{enumerate}}
\newcommand{\bi}{\begin{itemize}}
\newcommand{\ei}{\end{itemize}}
\newcommand{\bbe}{\begin{equation*}}
\newcommand{\eee}{\end{equation*}}
\newcommand{\bber}{\begin{equation*}\textcolor{red}}
\newcommand{\eeer}{\end{equation*}}
\newcommand{\bc}{\begin{center}}
\newcommand{\ec}{\end{center}}
\newcommand{\bea}{\begin{eqnarray}}
\newcommand{\eea}{\end{eqnarray}}
\newcommand{\bem}{\begin{pmatrix}}
\newcommand{\eem}{\end{pmatrix}}
\newcommand{\bbea}{\begin{eqnarray*}}
\newcommand{\eeea}{\end{eqnarray*}}
\newcommand{\bcols}{\begin{columns}[T]}
\newcommand{\ecols}{\end{columns}}
\newcommand{\bcol}[1]{\begin{column}{#1\textwidth}}
\newcommand{\ecol}{\end{column}}
%variance
\newcommand{\V}[1]{\mathrm{V}[{#1}]}
% efficiency:
\newcommand{\eps}{\ensuremath\varepsilon}

% escape math mode:
\newcommand{\mr}[1]{\mathrm{#1}}

% bold face:
\newcommand{\mbf}[1]{\mathbf{#1}}

% N_e:
\newcommand{\Ne}{\ensuremath N_\mathrm{e}}
% dEdX:
\newcommand{\dEdX}{\ensuremath\mathrm{d}E/\mathrm{d}X}
% Xmax:
\newcommand{\Xmax}{\ensuremath X_\mathrm{max}}
% meanXmax:
\newcommand{\meanXmax}{\ensuremath \langle X_\mathrm{max}\rangle}
% meanlnA:
\newcommand{\lnA}{\ensuremath \langle \ln \mathrm{A}\rangle}
% reference a float:
\newcommand{\rf}[2]{\mbox{#1 \ref{#2} }}

% microns
%%%%%%%%%%%%%%%%%
% Math Italic A %
%%%%%%%%%%%%%%%%%
\def\re@DeclareMathSymbol#1#2#3#4{%
    \let#1=\undefined
    \DeclareMathSymbol{#1}{#2}{#3}{#4}}

\DeclareSymbolFont{lettersA}{U}{pxmia}{m}{it}
\SetSymbolFont{lettersA}{bold}{U}{pxmia}{bx}{it}
\DeclareFontSubstitution{U}{pxmia}{m}{it}

\DeclareSymbolFontAlphabet{\mathfrak}{lettersA}
\re@DeclareMathSymbol{\muup}{\mathord}{lettersA}{"16}


% microns and microseconds
\newcommand{\mum}{\ensuremath\muup m}
\newcommand{\mus}{\ensuremath\muup s}
\newcommand{\gcm}{g/cm$^2$}

%programs
\newcommand{\GEANE}{\texttt{GEANE}}
\newcommand{\GEANT}{\texttt{GEANT}}
\newcommand{\lcgen}{\texttt{l3cgen}}
\newcommand{\CORSIKA}{\texttt{CORSIKA}}
\newcommand{\VENUS}{\texttt{VENUS}}
\newcommand{\GHEISHA}{\texttt{GEISHA}}

%Calibration
\newcommand{\qpkvem}{$Q^\mathrm{Peak}_\mathrm{VEM}$\,}

%class 1
\newcommand{\cone}{\mbox{class 1}}
\newcommand{\mc}[3]{\multicolumn{#1}{#2}{#3}}
\newcommand{\mcb}[2]{\multicolumn{#1}{c}{#2}}
\newcommand{\ttp}[2]{#1$\cdot$10$^#2$}
\newcommand{\bad}{\textcolor{red}{$\ominus$}}
\newcommand{\ok}{\textcolor{green}{$\oplus$}}
% misc abbreviations

\newcommand{\s}{$\:$}

\newcommand{\ra}{\ensuremath\rightarrow}
\newcommand{\vsp}[1]{\vspace*{#1cm}}
\newcommand{\hsp}[1]{\hspace*{#1cm}}
\def\s1000{S(\unit[1000]{m})}
%%%%%%%%%%%%%%%%%%%%%%%%%%%%%%%%%%%%%%%%%%%%%%%%%%
%
%  some abbreviations from physics.sty
%
%%%%%%%%%%%%%%%%%%%%%%%%%%%%%%%%%%%%%%%%%%%%%%%%%%


\def\EeV{\ifmmode {\mathrm{\ Ee\kern -0.1em V}}\else
                   \textrm{Ee\kern -0.1em V}\fi}%
\def\PeV{\ifmmode {\mathrm{\ Pe\kern -0.1em V}}\else
                   \textrm{Pe\kern -0.1em V}\fi}%
\def\TeV{\ifmmode {\mathrm{\ Te\kern -0.1em V}}\else
                   \textrm{Te\kern -0.1em V}\fi}%
\def\MeV{\ifmmode {\mathrm{\ Me\kern -0.1em V}}\else
                   \textrm{Me\kern -0.1em V}\fi}%
\def\GeV{\ifmmode {\mathrm{\ Ge\kern -0.1em V}}\else
                   \textrm{Ge\kern -0.1em V}\fi}%
\def\keV{\ifmmode {\mathrm{\ ke\kern -0.1em V}}\else
                   \textrm{ke\kern -0.1em V}\fi}%
\def\MeV{\ifmmode {\mathrm{\ Me\kern -0.1em V}}\else
                   \textrm{Me\kern -0.1em V}\fi}%
\def\eV{\ifmmode {\mathrm{\ e\kern -0.1em V}}\else
                   \textrm{e\kern -0.1em V}\fi}%
\def\Zo{\ensuremath{\mathrm {Z}}}
\def\Wp{\ensuremath{\mathrm {W^+}}}
\def\Wm{\ensuremath{\mathrm {W^-}}}
\def\epem{\ensuremath{\mathrm{e^+e^-}}}%
\def\mm{\ensuremath{\mathrm{\mu^+ \mu^-}}}%
\def\antibar#1{\ensuremath{#1\bar{#1}}}%
\def\nbar{\ensuremath{\bar{\nu}}}
\def\nnbar{\antibar{\nu}}%

% offline
\def\Offline{\mbox{$\overline{\rm
Off}$\hspace{.05em}\raisebox{.3ex}{$\underline{\rm line}$}}\xspace}
\def\OfflineB{\mbox{$\bf\overline{\rm\bf
Off}$\hspace{.05em}\raisebox{.2ex}{$\bf\underline{\rm\bf line}$}}\xspace}
\newcommand{\HRule}{\rule{\linewidth}{1mm}}

%%% Local Variables: 
%%% mode: latex
%%% TeX-master: t
%%% End: 



\begin{document}
\setpagewiselinenumbers
\modulolinenumbers[2]

\linenumbers

\renewcommand\linenumberfont{\small\rmfamily}
\begin{flushright}
%GAP-2021-xx
\end{flushright}

\begin{flushright}
  \rule{\linewidth}{0.5mm}
  \\[17mm]
  \vspace*{-3ex}{\Large Accuracy of $Q^\mathrm{Peak}_\mathrm{VEM}$ fit for UB and UUB}
  \large
  \parbox[b]{15cm}
  {
    \begin{flushright}
      Mauricio Su\'arez-Dur\'an and  Ioana~C.~Mari\c{s}
      \\[6mm]
      {\small Universit{\'e} Libre de Bruxelles, Belgium}
    \end{flushright}
  }
  \\[5mm]
  \rule{\linewidth}{0.5mm}
\end{flushright}
%
%\vspace*{-5ex}
%
\thispagestyle{empty}
\noindent

\begin{abstract}
  \noindent
  From the installation of the Upgraded Unified Board (UUB) is
  expected to have a better accuracy for the estimation of the
  \qpkvem values respect of the previous Unified Board (UB).
  Here, the process to calculate the accuracy of the \qpkvem
  values for the UUB and UB is presented, and the results of its
  application on $80$ UUB stations, and their respective UB
  version, are showed.
\end{abstract}

%
%\newpage
%
\thispagestyle{empty}
$\;$
%\listoftodos
%\newpage
\noindent
\clearpage

\section*{Accuracy calculation}

For the calculation of the \qpkvem accuracy, the charge
histograms measured from August 1st, to October 19th of 2021, by
80 UUB stations (and for each PMT) were fitting to obtain the
respective \qpkvem. On the other hand, with the aim of compare
the accuracy both version, UUB and UB, the same fitting was
performed to the same stations but before to become in UUB, i.e.
from August 1st, to October 19th of 2020, 2019, and 2018. Figure
\ref{fig1747QpkValues}, top row, shows an example for the
distribution of \qpkvem values (\qpkDist) for station 1747, UUB
and UB version.\\\\For each station, every \qpkvem was normalized
respect the average of the respective PMT-\qpkDist, and a single
distribution is built by summing the respective three
distributions. In this way, we obtain a preliminary distribution
that represents the \qpkvem normalized for each station. The last
one is preliminary because some times, a certain PMT (anomalous
PMT)could have a wide \qpkDist respect the other PMTs (see right
side in figure \ref{figMultipleQpksSt843}). So, to discard these
anomalous PMTs, the RMS of the preliminary distribution (Pre-RMS)
is used in the next way: if the RMS of some PMT-\qpkDist is
bigger than $1.3$ times the Pre-RMS, so it is discarded, from UB
and UUB. As an example, figure \ref{fig1747QpkValues}, lower row,
shows the this \qpkNormDist for the station 1747. A Gauss
function was fitted to this last distribution in order to the get
the first two moments of the distribution.
\vspace{.5cm}

\begin{figure}[!t]
  \centering
  \subfigure {
    \includegraphics[width=0.49\textwidth]{../../plots/filteredPMTsSt1747.pdf}
    \includegraphics[width=0.49\textwidth]{../../plots/filteredUbPMTsSt1747.pdf}
  }
  \subfigure {
    \includegraphics[width=0.49\textwidth]{../../plots/filteredSt1747.pdf}
    \includegraphics[width=0.49\textwidth]{../../plots/filteredUbSt1747.pdf}
  }
  \caption{Upper row, distribution of the fitted \qpkvem values
  (\qpkDist, in the plot) for station 1747, for each PMT (red
  PMT1, blue PMT2, and green PMT3), UUB (left) and UB (right)
  version. Lower row, For same station, results for \qpkvem
  normalized respect the average of the respective PMTs. The
  magenta line represents the \qpkNormDist, i.e. the sum of
  \qpkDist with a RMS lower or equal to $1.3$ times the $\sigma$
  of the preliminary distribution (see text for details). In red,
  a Gauss function fitted to \qpkNormDist.}
  \label{fig1747QpkValues}
\end{figure}
\clearpage

\noindent For some PMTs, the \qpkDist hasmore than one peak, as
the left side of figure \ref{figMultipleQpksSt843} shows. In
theses cases, each peak was fitted using a Gauss function, and
the \qpkNormDist was calculate respect each mean one; for this,
the \qpkvem values into $\pm\sigma$, of the respective Gauss
function, have been normalized. For instance, the right side of
figure \ref{figMultipleQpksSt843} presents the results of the
last procedure for station 843.\\\\The accuracy of \qpkvem for
one station is calculate then as the ratio $\sigma/\mu$, from a
Gauss function fitted to the respective \qpkNormDist. The upper
row of figure \ref{figAccuracyResults} shows the results for the
accuracy of the fitted \qpkvem for the $80$ stations, UUB and UB.
In the left side, the accuracy is presented as function of
station ID, and in right side the distribution for UUB and UB
accuracy are presented, with a mean for $\sigma/\mu$ of
$2.015\pm0.715$\,\% for UB and $1.705\pm0.636$\,\% for UUB. In
the bottom row of the same figure, the difference between these
two accuracy, i.e. $\left(\sigma/\mu\right)_{\mathrm{UB}}
-\left(\sigma/\mu\right)_{\mathrm{UUB}}$ is presented, as
function of the station ID (left side), and the respective
distribution of these differences (right side). As it can be
seen, mean of the latest distribution is $0.288\pm0.977$\,\%
which is according with the expected results, the accuracy for
UUB is better than the one for UB.

\begin{figure}[!t]
  \centering
  \subfigure {
    \includegraphics[width=0.49\textwidth]{../../plots/filteredPMTsSt843.pdf}
    \includegraphics[width=0.49\textwidth]{../../plots/filteredSt843.pdf}
  }
  \caption{On left side, for station 843, \qpkDist with more than
  one peak: PMTs 2 and 3, blue and green lines, respectively. The
  PMT1, red line, presents a wide distribution as can be seen on
  the right figure (black line). On the right, it is possible to
  see how the \qpkDist for PMT1 is discarded to build the
  \qpkNormDist (magenta line). The red line corresponds to
  the Gauss function fitted to the latest distribution.}
  \label{figMultipleQpksSt843}
\end{figure}
\clearpage


\begin{figure}[!t]
  \centering
  \subfigure {
    \includegraphics[width=0.5\textwidth]{../../plots/relSigmaQpkVsStationsId.pdf}
    \includegraphics[width=0.5\textwidth]{../../plots/accuracyQpksFitsUbUubAllStAllPmt_v2.pdf}
  }
  \subfigure {
    \includegraphics[width=0.5\textwidth]{../../plots/diffRelSigmaQpkVsStationsId.pdf}
    \includegraphics[width=0.5\textwidth]{../../plots/distDiffSigmaQpk.pdf}
  }
  \caption{Results for the accuracy of the \qpkvem fitting for
  $80$ stations, UUB and UB. Upper row: in left side, the
  accuracy ($\sigma/\mu$) is presented for each station. The
  horizontal lines represents the averages ($\left< \sigma/\mu
  \right>$) for UUB (red line) and UB (blue line). In the right
  side, the distribution of the $\sigma/\mu$ values is presented
  for both version, UUB (red) and UB (blue). Bottom row:
  difference between the accuracy of UB and UUB,
  $\left(\sigma/\mu\right)_{\mathrm{UB}})-\left(\sigma/\mu\right)_{\mathrm{UUB}}$,
  left side as function of the station ID, right side the
  distribution of these values. Here, it is possible to see that
  the accuracy for UUB is better than the one for UB.}
  \label{figAccuracyResults}
\end{figure}
\textcolor{white}{hi}
\clearpage

\end{document}
