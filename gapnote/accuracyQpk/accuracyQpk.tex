\documentclass[twoside, final, 10pt]{articleMine}
\usepackage[english]{babel}
\usepackage[sc]{mathpazo}
\usepackage{a4wide}
\usepackage{subfigure}

\usepackage{hyperref}
\usepackage{amsmath,amssymb}
\usepackage[right]{lineno}
\usepackage{xspace}
\usepackage{accents}
\usepackage{graphicx}
\usepackage{booktabs}
\usepackage{color}
\usepackage{units}
\usepackage{enumitem}
\usepackage{todonotes}
\usepackage[capitalize]{cleveref}
\usepackage{nameref}
\usepackage{csquotes}

\usepackage{listings}
\usepackage{color}

\definecolor{dkgreen}{rgb}{0,0.6,0}
\definecolor{gray}{rgb}{0.5,0.5,0.5}
\definecolor{mauve}{rgb}{0.58,0,0.82}

\lstset{frame=tb,
  language=XML,
  aboveskip=3mm,
  belowskip=3mm,
  showstringspaces=false,
  columns=flexible,
  basicstyle={\small\ttfamily},
  numbers=none,
  numberstyle=\tiny\color{gray},
  keywordstyle=\color{blue},
  commentstyle=\color{dkgreen},
  stringstyle=\color{mauve},
  breaklines=true,
  breakatwhitespace=true,
  tabsize=3
}

\usepackage[thinc]{esdiff}


\usepackage{booktabs}
\usepackage{mcite}


\graphicspath{{plots/}}
\newcommand*\patchAmsMathEnvironmentForLineno[1]{%
  \expandafter\let\csname old#1\expandafter\endcsname\csname #1\endcsname
  \expandafter\let\csname oldend#1\expandafter\endcsname\csname end#1\endcsname
  \renewenvironment{#1}%
     {\linenomath\csname old#1\endcsname}%
     {\csname oldend#1\endcsname\endlinenomath}}%
\newcommand*\patchBothAmsMathEnvironmentsForLineno[1]{%
  \patchAmsMathEnvironmentForLineno{#1}%
  \patchAmsMathEnvironmentForLineno{#1*}}%
\AtBeginDocument{%
\patchBothAmsMathEnvironmentsForLineno{equation}%
\patchBothAmsMathEnvironmentsForLineno{align}%
\patchBothAmsMathEnvironmentsForLineno{flalign}%
\patchBothAmsMathEnvironmentsForLineno{alignat}%
\patchBothAmsMathEnvironmentsForLineno{gather}%
\patchBothAmsMathEnvironmentsForLineno{multline}%
}

\newcommand{\putat}[3]{\begin{picture}(0,0)(0,0)\put(#1,#2){#3}\end{picture}} % just a shorthand
\parskip 4.2pt           % sets spacing between paragraphs
%\parindend 0pt           % sets spacing between paragraphs
\def\Offline{\mbox{$\overline{\rm
Off}$\hspace{.05em}\raisebox{.4ex}{$\underline{\rm line}$}}\xspace}
\def\OfflineB{\mbox{$\bf\overline{\rm\bf
Off}$\hspace{.05em}\raisebox{.4ex}{$\bf\underline{\rm\bf line}$}}\xspace}

\newcommand{\qpkDist}{$Q^\mathrm{Pk}_\mathrm{Dist}$\,}
\newcommand{\qpkNormDist}{$Q^\mathrm{Pk}_\mathrm{NormDist}$\,}

% equations ...
\newcommand{\be}{\begin{equation}}
\newcommand{\ee}{\end{equation}}
\newcommand{\ben}{\begin{enumerate}}
\newcommand{\een}{\end{enumerate}}
\newcommand{\bi}{\begin{itemize}}
\newcommand{\ei}{\end{itemize}}
\newcommand{\bbe}{\begin{equation*}}
\newcommand{\eee}{\end{equation*}}
\newcommand{\bber}{\begin{equation*}\textcolor{red}}
\newcommand{\eeer}{\end{equation*}}
\newcommand{\bc}{\begin{center}}
\newcommand{\ec}{\end{center}}
\newcommand{\bea}{\begin{eqnarray}}
\newcommand{\eea}{\end{eqnarray}}
\newcommand{\bem}{\begin{pmatrix}}
\newcommand{\eem}{\end{pmatrix}}
\newcommand{\bbea}{\begin{eqnarray*}}
\newcommand{\eeea}{\end{eqnarray*}}
\newcommand{\bcols}{\begin{columns}[T]}
\newcommand{\ecols}{\end{columns}}
\newcommand{\bcol}[1]{\begin{column}{#1\textwidth}}
\newcommand{\ecol}{\end{column}}
%variance
\newcommand{\V}[1]{\mathrm{V}[{#1}]}
% efficiency:
\newcommand{\eps}{\ensuremath\varepsilon}

% escape math mode:
\newcommand{\mr}[1]{\mathrm{#1}}

% bold face:
\newcommand{\mbf}[1]{\mathbf{#1}}

% N_e:
\newcommand{\Ne}{\ensuremath N_\mathrm{e}}
% dEdX:
\newcommand{\dEdX}{\ensuremath\mathrm{d}E/\mathrm{d}X}
% Xmax:
\newcommand{\Xmax}{\ensuremath X_\mathrm{max}}
% meanXmax:
\newcommand{\meanXmax}{\ensuremath \langle X_\mathrm{max}\rangle}
% meanlnA:
\newcommand{\lnA}{\ensuremath \langle \ln \mathrm{A}\rangle}
% reference a float:
\newcommand{\rf}[2]{\mbox{#1 \ref{#2} }}

% microns
%%%%%%%%%%%%%%%%%
% Math Italic A %
%%%%%%%%%%%%%%%%%
\def\re@DeclareMathSymbol#1#2#3#4{%
    \let#1=\undefined
    \DeclareMathSymbol{#1}{#2}{#3}{#4}}

\DeclareSymbolFont{lettersA}{U}{pxmia}{m}{it}
\SetSymbolFont{lettersA}{bold}{U}{pxmia}{bx}{it}
\DeclareFontSubstitution{U}{pxmia}{m}{it}

\DeclareSymbolFontAlphabet{\mathfrak}{lettersA}
\re@DeclareMathSymbol{\muup}{\mathord}{lettersA}{"16}


% microns and microseconds
\newcommand{\mum}{\ensuremath\muup m}
\newcommand{\mus}{\ensuremath\muup s}
\newcommand{\gcm}{g/cm$^2$}

%programs
\newcommand{\GEANE}{\texttt{GEANE}}
\newcommand{\GEANT}{\texttt{GEANT}}
\newcommand{\lcgen}{\texttt{l3cgen}}
\newcommand{\CORSIKA}{\texttt{CORSIKA}}
\newcommand{\VENUS}{\texttt{VENUS}}
\newcommand{\GHEISHA}{\texttt{GEISHA}}

%Calibration
\newcommand{\qpkvem}{$Q^\mathrm{Peak}_\mathrm{VEM}$\,}

%class 1
\newcommand{\cone}{\mbox{class 1}}
\newcommand{\mc}[3]{\multicolumn{#1}{#2}{#3}}
\newcommand{\mcb}[2]{\multicolumn{#1}{c}{#2}}
\newcommand{\ttp}[2]{#1$\cdot$10$^#2$}
\newcommand{\bad}{\textcolor{red}{$\ominus$}}
\newcommand{\ok}{\textcolor{green}{$\oplus$}}
% misc abbreviations

\newcommand{\s}{$\:$}

\newcommand{\ra}{\ensuremath\rightarrow}
\newcommand{\vsp}[1]{\vspace*{#1cm}}
\newcommand{\hsp}[1]{\hspace*{#1cm}}
\def\s1000{S(\unit[1000]{m})}
%%%%%%%%%%%%%%%%%%%%%%%%%%%%%%%%%%%%%%%%%%%%%%%%%%
%
%  some abbreviations from physics.sty
%
%%%%%%%%%%%%%%%%%%%%%%%%%%%%%%%%%%%%%%%%%%%%%%%%%%


\def\EeV{\ifmmode {\mathrm{\ Ee\kern -0.1em V}}\else
                   \textrm{Ee\kern -0.1em V}\fi}%
\def\PeV{\ifmmode {\mathrm{\ Pe\kern -0.1em V}}\else
                   \textrm{Pe\kern -0.1em V}\fi}%
\def\TeV{\ifmmode {\mathrm{\ Te\kern -0.1em V}}\else
                   \textrm{Te\kern -0.1em V}\fi}%
\def\MeV{\ifmmode {\mathrm{\ Me\kern -0.1em V}}\else
                   \textrm{Me\kern -0.1em V}\fi}%
\def\GeV{\ifmmode {\mathrm{\ Ge\kern -0.1em V}}\else
                   \textrm{Ge\kern -0.1em V}\fi}%
\def\keV{\ifmmode {\mathrm{\ ke\kern -0.1em V}}\else
                   \textrm{ke\kern -0.1em V}\fi}%
\def\MeV{\ifmmode {\mathrm{\ Me\kern -0.1em V}}\else
                   \textrm{Me\kern -0.1em V}\fi}%
\def\eV{\ifmmode {\mathrm{\ e\kern -0.1em V}}\else
                   \textrm{e\kern -0.1em V}\fi}%
\def\Zo{\ensuremath{\mathrm {Z}}}
\def\Wp{\ensuremath{\mathrm {W^+}}}
\def\Wm{\ensuremath{\mathrm {W^-}}}
\def\epem{\ensuremath{\mathrm{e^+e^-}}}%
\def\mm{\ensuremath{\mathrm{\mu^+ \mu^-}}}%
\def\antibar#1{\ensuremath{#1\bar{#1}}}%
\def\nbar{\ensuremath{\bar{\nu}}}
\def\nnbar{\antibar{\nu}}%

% offline
\def\Offline{\mbox{$\overline{\rm
Off}$\hspace{.05em}\raisebox{.3ex}{$\underline{\rm line}$}}\xspace}
\def\OfflineB{\mbox{$\bf\overline{\rm\bf
Off}$\hspace{.05em}\raisebox{.2ex}{$\bf\underline{\rm\bf line}$}}\xspace}
\newcommand{\HRule}{\rule{\linewidth}{1mm}}

%%% Local Variables: 
%%% mode: latex
%%% TeX-master: t
%%% End: 



\begin{document}
\setpagewiselinenumbers
\modulolinenumbers[2]

\linenumbers

\renewcommand\linenumberfont{\small\rmfamily}
\begin{flushright}
%GAP-2021-xx
\end{flushright}

\begin{flushright}
  \rule{\linewidth}{0.5mm}
  \\[17mm]
  \vspace*{-3ex}{\Large Accuracy of $Q^\mathrm{Peak}_\mathrm{VEM}$ fit for UB and UUB}
  \large
  \parbox[b]{15cm}
  {
    \begin{flushright}
      Mauricio Su\'arez-Dur\'an and  Ioana~C.~Mari\c{s}
      \\[6mm]
      {\small Universit{\'e} Libre de Bruxelles, Belgium}
    \end{flushright}
  }
  \\[5mm]
  \rule{\linewidth}{0.5mm}
\end{flushright}
%
%\vspace*{-5ex}
%
\thispagestyle{empty}
\noindent

\begin{abstract}
  \noindent
  From the installation of the Upgraded Unified Board (UUB) is
  expected to have a better accuracy for the estimation of the
  \qpkvem values respect of the previous Unified Board (UB).
  Here, the process to calculate the accuracy of the \qpkvem
  values for the UUB and UB is presented, and the results of its
  application on $80$ UUB stations, and their respective UB
  version, are showed.
\end{abstract}

%
%\newpage
%
\thispagestyle{empty}
$\;$
%\listoftodos
%\newpage
\noindent
\clearpage

\section{Fitting histograms}
\label{secFitting}

\begin{figure}[!t]
  \centering
  \subfigure {
    \includegraphics[width=.5\textwidth]{../../plots/chargeHisto863.pdf}
    \includegraphics[width=.5\textwidth]{../../plots/chargeDerHisto863.pdf}
  }
  \caption{Algorithm's steps applied to get the \qpkvem from the
  calibration histograms. Left: typical charge calibration
  histogram, where the black line represents the smoothing
  histogram ($H_S$) after applying 15-bin sliding window. Right:
  first derivative histogram ($H_{DS}$) in black, and smoothing
  of this last one in red ($H_{SDS}$); here, the green vertical
  line shows the maximum or the first approach of \qpkvem.}
  \label{figChargeDerivative}
\end{figure}

The \qpkvem value is obtained from the charge calibration
histogram, an algorithm based on the first derivative has been
implemented and applied. The goal of this procedure is to
estimate the \qpkvem as the maximum of a second order polynomial,
fitting the respective hump muon. In this sense, the first
derivative of the histogram works as first approach to locate
this hump, and from there to estimate the fit range. This
algorithm is described below, and showed in figure
\ref{figChargeDerivative}:

\begin{enumerate}
  \item Smoothing the histogram using a 15-bin sliding window,
    $H_S$.
  \item Obtaining the first derivative of the $H_S$, by
    \begin{equation}
      \frac{f(x+1)-f(x-1)}{2h} \, ,
    \end{equation}
    which is called $H_{DS}$.
  \item Smoothing $H_{DS}$, by 15-bin sliding window, and
    obtaining $H_{SDS}$.
  \item Searching for the estimated \qpkvem, i.e. first bin for
    $H_{SDS}$ equal to zero, from right to left.
  \item Fixing the fitting range using n-bin leftward and n-bin
    rightward from the estimated \qpkvem.
\end{enumerate}
The last algorithm' step requires an extra procedure in order to
fixed the number of n-bin. Using a set of histograms, the
stability of the fit (in terms of RMS/$\left<
Q^\mathrm{Peak}_\mathrm{VEM} \right>$) was plotted as a function
of the n-bin. For this, the estimated \qpkvem from the derivative
(step 4.) was used as initial parameter, and from this point, a
number of n-bin was fixed with an extra condition of not reaching
the valley of the histogram. Finally, the
RMS/$\left<Q^\mathrm{Peak}_\mathrm{VEM}\right>$ vs n-bin was
plotted, and it is presented in the first row of figure
\ref{figApplyingAlgorithm}. There, the stability of the fit
is seen at and after 30-bin, and this last one is chosen as the
number of n-bin for the algorithm' step 5. In the bottom row of
figure \ref{figApplyingAlgorithm}, an example of applying this
procedure is presented. Section \ref{secQpkVsTime} contains the
\qpkvem values obtained by this method from 75 UUB stations
and for the same stations but in UB version (using the same
algorithm).

\clearpage

\begin{figure}[!t]
  \centering
  \subfigure {
    \includegraphics[width=.5\textwidth]{../../plots/uubChRmsFitBinsLrSt863.pdf}
    \includegraphics[width=.5\textwidth]{../../plots/uubChRmsFitBinsLrSt863zoom.pdf}
  }
  \subfigure{
    \includegraphics[width=.5\textwidth]{../../plots/chargeFitPoly2863.pdf}
    \includegraphics[width=.5\textwidth]{../../plots/chargeFitResidualsPoly2863.pdf}
  }
  \caption{Top row, results for the stability of fitting the hump
  muon (RMS/$\left<Q^\mathrm{Peak}_\mathrm{VEM}\right>$) as
  function of the n-bin. After 30 bins the stability is reached.
  Bottom row, an example of applying the algorithm; here the
  vertical green line shows the hump VEM obtained from the first
  derivative (step 4.), and the vertical blue line shows the
  \qpkvem obtained as the maximum of the fitted second order
  polynomial.}
  \label{figApplyingAlgorithm}
\end{figure}

\section{Accuracy calculation}

For the calculation of the \qpkvem accuracy, each of one of these
values was normalized respect the average of the respective
PMT-\qpkDist, and a single distribution is built by summing the
respective three distributions. In this way, we obtain a
distribution that represents the \qpkvem normalized for each
station. A Gauss function is fitted to this last distribution in
order to the get the first two moments of the distribution, which
are used to determine the accuracy, $\mu/\sigma$.\\\\With the aim
of compare the accuracy both version, UUB and UB, the former
procedure was applied to same stations before to become in UUB,
i.e. from August 1st, to October 19th of 2018, UB, and from
August 1st, to October 19th of 2021, UUB. Figure
\ref{fig1747QpkValues}, top row, shows an example for the
distribution of \qpkvem values (\qpkDist) for station 1208, UUB
and UB version, meanwhile bottom row shows the distribution for
the \qpkvem normalized.\\\\For some PMTs, the \qpkDist has more
than one peak, this imply a wide distribution for the normalized
values, as it can be seen in figure \ref{figMultipleQpksSt843}.
To avoid this kind of distribution, a selection over the
stations, and PMTs, was applied. This selection consisted of
choosing a time window in which the \qpkvem was constant, i.e.
fitting a constant function and checking for a fit quality of
$\chi^2/\mathrm{ndf}<1.0$. Those PMT-\qpkDist, for the respective
station, that not agree with this criteria, were not taken in to
the account, for instance, if certain UB-PMT disagree, so the
respective UUB-PMT was ignore too, and vice versa. In section
\ref{secQpkVsTime} the \qpkvem as function of time, for all UUB
stations, is presented. There, the stations with PMT-\qpkDist
selected are indicated. In total, 23 stations were chosen with 52
PMT-\qpkDist, respectively.
%\vspace{.5cm}
\clearpage

\begin{figure}[!t]
  \centering
  \subfigure {
    \includegraphics[width=0.49\textwidth]{../../plots/filteredPMTsSt1208.pdf}
    \includegraphics[width=0.49\textwidth]{../../plots/filteredUbPMTsSt1208.pdf}
  }
  \subfigure {
    \includegraphics[width=0.49\textwidth]{../../plots/filteredSt1208.pdf}
    \includegraphics[width=0.49\textwidth]{../../plots/filteredUbSt1208.pdf}
  }
  \caption{Upper row, distribution of the fitted \qpkvem values
  (\qpkDist, in the plot) for station 1747, for each PMT (red
  PMT1, blue PMT2, and green PMT3), UUB (left) and UB (right)
  version. Lower row, For same station, results for \qpkvem
  normalized respect the average of the respective PMTs. The
  magenta line represents the \qpkNormDist, i.e. the sum of
  \qpkDist with a RMS lower or equal to $1.3$ times the $\sigma$
  of the preliminary distribution (see text for details). In red,
  a Gauss function fitted to \qpkNormDist.}
  \label{fig1747QpkValues}
\end{figure}

\noindent The upper row of figure \ref{figAccuracyResults} shows
the results for the accuracy of the fitted \qpkvem for the 23
chosen stations. In the left side, the accuracy is presented as
function of station ID, and in right side the distribution for
UUB and UB accuracy are presented. There, it can been seen that
the mean for the accuracy ($\left<\sigma/\mu \right>$) for UB is
$1.08\pm0.06$\,\%, meanwhile for UUB this value is
$1.20\pm0.06$\,\%. In the bottom row of the same figure, the
difference between two accuracy, i.e.
$\left(\sigma/\mu\right)_{\mathrm{UB}}
-\left(\sigma/\mu\right)_{\mathrm{UUB}}$ is presented, as
function of the station ID (left side), and the respective
distribution of these differences (right side). As it can be
seen, the mean of the this distribution is $-0.115\pm0.401$\,\%.
\clearpage

\begin{figure}[!t]
  \centering
  \subfigure {
    \includegraphics[width=0.49\textwidth]{../../plots/filteredPMTsSt843.pdf}
    \includegraphics[width=0.49\textwidth]{../../plots/filteredSt843.pdf}
  }
  \caption{On left side, for station 843, \qpkDist with more than
  one peak: PMTs 2 and 3, blue and green lines, respectively. The
  PMT1, red line, presents a wide distribution as can be seen on
  the right figure (black line). On the right, it is possible to
  see how the \qpkDist for PMT1 is discarded to build the
  \qpkNormDist (magenta line). The red line corresponds to
  the Gauss function fitted to the latest distribution.}
  \label{figMultipleQpksSt843}
\end{figure}

Hi.
\clearpage


\begin{figure}[!t]
  \centering
  \subfigure {
    \includegraphics[width=0.5\textwidth]{../../plots2/relSigmaQpkVsStationsId.pdf}
    \includegraphics[width=0.5\textwidth]{../../plots2/accuracyQpksFitsUbUubAllStAllPmt_v2.pdf}
  }
  \subfigure {
    \includegraphics[width=0.5\textwidth]{../../plots2/diffRelSigmaQpkVsStationsId.pdf}
    \includegraphics[width=0.5\textwidth]{../../plots2/distDiffSigmaQpk.pdf}
  }
  \caption{Results for the accuracy of the \qpkvem fitting for
  $75$ stations, UUB and UB. Upper row: in left side, the
  accuracy ($\sigma/\mu$) is presented for each station. The
  horizontal lines represents the averages ($\left< \sigma/\mu
  \right>$) for UUB (red line) and UB (blue line). In the right
  side, the distribution of the $\sigma/\mu$ values is presented
  for both version, UUB (red) and UB (blue). Bottom row:
  difference between the accuracy of UB and UUB,
  $\left(\sigma/\mu\right)_{\mathrm{UB}})-\left(\sigma/\mu\right)_{\mathrm{UUB}}$,
  left side as function of the station ID, right side the
  distribution of these values. Here, it is possible to see that
  the accuracy for UUB is better than the one for UB.}
  \label{figAccuracyResults}
\end{figure}
\textcolor{white}{hi}
\clearpage

\section{\qpkvem as function of time}
\label{secQpkVsTime}

\begin{figure}[!b]
  \centering
  \subfigure {
    \includegraphics[width=0.49\textwidth]{../../plots/qpksVsTimeSt545UB.pdf}
    \includegraphics[width=0.49\textwidth]{../../plots/qpksVsTimeSt545UUB.pdf}
  }
  \subfigure {
    \includegraphics[width=0.49\textwidth]{../../plots/chi2NdfQpksVsTimeSt545UB.pdf}
    \includegraphics[width=0.49\textwidth]{../../plots/chi2NdfQpksVsTimeSt545UUB.pdf}
  }   
  \subfigure {
    \includegraphics[width=0.49\textwidth]{../../plots/qpksVsTimeSt804UB.pdf}
    \includegraphics[width=0.49\textwidth]{../../plots/qpksVsTimeSt804UUB.pdf}
  }
  \subfigure {
    \includegraphics[width=0.49\textwidth]{../../plots/qpksVsTimeSt806UB.pdf}
    \includegraphics[width=0.49\textwidth]{../../plots/qpksVsTimeSt806UUB.pdf}
  }
  \caption{\qpkvem values obtained applying the algorithm
  presented in section \ref{secFitting}. The first row shows the
  average of the \qpkvem per day plotting as function of time,
  for the station  545, left UB, and right UUB version. In the
  second raw, the quality of the respective fit (the average of
  $\chi^2/$ndf per day), same station, is plotting as function of
  time. The third and fourth row present the same results as row
  1, for stations 804 and 806. if the label ``Selected'' is seen,
  it means that some, or all, PMTs were chosen to calculate the
  accuracy of \qpkvem.}
  %\label{figQpksTime}
\end{figure}
\clearpage

hi
\begin{figure}[!b]
  \centering
  \subfigure {
    \includegraphics[width=0.49\textwidth]{../../plots/qpksVsTimeSt827UB.pdf}
    \includegraphics[width=0.49\textwidth]{../../plots/qpksVsTimeSt827UUB.pdf}
  }
  \subfigure {
    \includegraphics[width=0.49\textwidth]{../../plots/qpksVsTimeSt830UB.pdf}
    \includegraphics[width=0.49\textwidth]{../../plots/qpksVsTimeSt830UUB.pdf}
  }
  \subfigure {
    \includegraphics[width=0.49\textwidth]{../../plots/qpksVsTimeSt832UB.pdf}
    \includegraphics[width=0.49\textwidth]{../../plots/qpksVsTimeSt832UUB.pdf}
  }
  \subfigure {
    \includegraphics[width=0.49\textwidth]{../../plots/qpksVsTimeSt833UB.pdf}
    \includegraphics[width=0.49\textwidth]{../../plots/qpksVsTimeSt833UUB.pdf}
  }
  \caption{\qpkvem values obtained applying the algorithm
  presented in section \ref{secFitting}. Each row shows the
  average of the \qpkvem per day plotting as function of time,
  left UB, and right UUB version. If the label ``Selected'' is
  seen, it means that some, or all, PMTs were chosen to calculate
  the accuracy of \qpkvem.}
\end{figure}
\clearpage

hi
\begin{figure}[!b]
  \centering
  \subfigure {
    \includegraphics[width=0.49\textwidth]{../../plots/qpksVsTimeSt836UB.pdf}
    \includegraphics[width=0.49\textwidth]{../../plots/qpksVsTimeSt836UUB.pdf}
  }
  \subfigure {
    \includegraphics[width=0.49\textwidth]{../../plots/qpksVsTimeSt840UB.pdf}
    \includegraphics[width=0.49\textwidth]{../../plots/qpksVsTimeSt840UUB.pdf}
  }
  \subfigure {
    \includegraphics[width=0.49\textwidth]{../../plots/qpksVsTimeSt843UB.pdf}
    \includegraphics[width=0.49\textwidth]{../../plots/qpksVsTimeSt843UUB.pdf}
  }
  \subfigure {
    \includegraphics[width=0.49\textwidth]{../../plots/qpksVsTimeSt846UB.pdf}
    \includegraphics[width=0.49\textwidth]{../../plots/qpksVsTimeSt846UUB.pdf}
  }
  \caption{\qpkvem values obtained applying the algorithm
  presented in section \ref{secFitting}. Each row shows the
  average of the \qpkvem per day plotting as function of time,
  left UB, and right UUB version. If the label ``Selected'' is
  seen, it means that some, or all, PMTs were chosen to calculate
  the accuracy of \qpkvem.}
\end{figure}
\clearpage

hi
\begin{figure}[!b]
  \centering
  \subfigure {
    \includegraphics[width=0.49\textwidth]{../../plots/qpksVsTimeSt849UB.pdf}
    \includegraphics[width=0.49\textwidth]{../../plots/qpksVsTimeSt849UUB.pdf}
  }
  \subfigure {
    \includegraphics[width=0.49\textwidth]{../../plots/qpksVsTimeSt850UB.pdf}
    \includegraphics[width=0.49\textwidth]{../../plots/qpksVsTimeSt850UUB.pdf}
  }
  \subfigure {
    \includegraphics[width=0.49\textwidth]{../../plots/qpksVsTimeSt851UB.pdf}
    \includegraphics[width=0.49\textwidth]{../../plots/qpksVsTimeSt851UUB.pdf}
  }
  \subfigure {
    \includegraphics[width=0.49\textwidth]{../../plots/qpksVsTimeSt853UB.pdf}
    \includegraphics[width=0.49\textwidth]{../../plots/qpksVsTimeSt853UUB.pdf}
  }
  \caption{\qpkvem values obtained applying the algorithm
  presented in section \ref{secFitting}. Each row shows the 
  average of the \qpkvem per day plotting as function of time,
  left UB, and right UUB version. If the label ``Selected'' is
  seen, it means that some, or all, PMTs were chosen to calculate
  the accuracy of \qpkvem.}
\end{figure}
\clearpage

hi
\begin{figure}[!b]
  \centering
  \subfigure {
    \includegraphics[width=0.49\textwidth]{../../plots/qpksVsTimeSt849UB.pdf}
    \includegraphics[width=0.49\textwidth]{../../plots/qpksVsTimeSt849UUB.pdf}
  }
  \subfigure {
    \includegraphics[width=0.49\textwidth]{../../plots/qpksVsTimeSt850UB.pdf}
    \includegraphics[width=0.49\textwidth]{../../plots/qpksVsTimeSt850UUB.pdf}
  }
  \subfigure {
    \includegraphics[width=0.49\textwidth]{../../plots/qpksVsTimeSt851UB.pdf}
    \includegraphics[width=0.49\textwidth]{../../plots/qpksVsTimeSt851UUB.pdf}
  }
  \subfigure {
    \includegraphics[width=0.49\textwidth]{../../plots/qpksVsTimeSt853UB.pdf}
    \includegraphics[width=0.49\textwidth]{../../plots/qpksVsTimeSt853UUB.pdf}
  }
  \caption{\qpkvem values obtained applying the algorithm
  presented in section \ref{secFitting}. Each row shows the 
  average of the \qpkvem per day plotting as function of time,
  left UB, and right UUB version. If the label ``Selected'' is
  seen, it means that some, or all, PMTs were chosen to calculate
  the accuracy of \qpkvem.}
\end{figure}
\clearpage

hi
\begin{figure}[!b]
  \centering
  \subfigure {
    \includegraphics[width=0.49\textwidth]{../../plots/qpksVsTimeSt856UB.pdf}
    \includegraphics[width=0.49\textwidth]{../../plots/qpksVsTimeSt856UUB.pdf}
  }
  \subfigure {
    \includegraphics[width=0.49\textwidth]{../../plots/qpksVsTimeSt859UB.pdf}
    \includegraphics[width=0.49\textwidth]{../../plots/qpksVsTimeSt859UUB.pdf}
  }
  \subfigure {
    \includegraphics[width=0.49\textwidth]{../../plots/qpksVsTimeSt860UB.pdf}
    \includegraphics[width=0.49\textwidth]{../../plots/qpksVsTimeSt860UUB.pdf}
  }
  \subfigure {
    \includegraphics[width=0.49\textwidth]{../../plots/qpksVsTimeSt861UB.pdf}
    \includegraphics[width=0.49\textwidth]{../../plots/qpksVsTimeSt861UUB.pdf}
  }
  \caption{\qpkvem values obtained applying the algorithm
  presented in section \ref{secFitting}. Each row shows the 
  average of the \qpkvem per day plotting as function of time,
  left UB, and right UUB version. If the label ``Selected'' is
  seen, it means that some, or all, PMTs were chosen to calculate
  the accuracy of \qpkvem.}
\end{figure}
\clearpage

hi
\begin{figure}[!b]
  \centering
  \subfigure {
    \includegraphics[width=0.49\textwidth]{../../plots/qpksVsTimeSt862UB.pdf}
    \includegraphics[width=0.49\textwidth]{../../plots/qpksVsTimeSt862UUB.pdf}
  }
  \subfigure {
    \includegraphics[width=0.49\textwidth]{../../plots/qpksVsTimeSt863UB.pdf}
    \includegraphics[width=0.49\textwidth]{../../plots/qpksVsTimeSt863UUB.pdf}
  }
  \subfigure {
    \includegraphics[width=0.49\textwidth]{../../plots/qpksVsTimeSt864UB.pdf}
    \includegraphics[width=0.49\textwidth]{../../plots/qpksVsTimeSt864UUB.pdf}
  }
  \subfigure {
    \includegraphics[width=0.49\textwidth]{../../plots/qpksVsTimeSt866UB.pdf}
    \includegraphics[width=0.49\textwidth]{../../plots/qpksVsTimeSt866UUB.pdf}
  }
  \caption{\qpkvem values obtained applying the algorithm
  presented in section \ref{secFitting}. Each row shows the 
  average of the \qpkvem per day plotting as function of time,
  left UB, and right UUB version. If the label ``Selected'' is
  seen, it means that some, or all, PMTs were chosen to calculate
  the accuracy of \qpkvem.}
\end{figure}
\clearpage

hi
\begin{figure}[!b]
  \centering
  \subfigure {
    \includegraphics[width=0.49\textwidth]{../../plots/qpksVsTimeSt868UB.pdf}
    \includegraphics[width=0.49\textwidth]{../../plots/qpksVsTimeSt868UUB.pdf}
  }
  \subfigure {
    \includegraphics[width=0.49\textwidth]{../../plots/qpksVsTimeSt871UB.pdf}
    \includegraphics[width=0.49\textwidth]{../../plots/qpksVsTimeSt871UUB.pdf}
  }
  \subfigure {
    \includegraphics[width=0.49\textwidth]{../../plots/qpksVsTimeSt907UB.pdf}
    \includegraphics[width=0.49\textwidth]{../../plots/qpksVsTimeSt907UUB.pdf}
  }
  \subfigure {
    \includegraphics[width=0.49\textwidth]{../../plots/qpksVsTimeSt1185UB.pdf}
    \includegraphics[width=0.49\textwidth]{../../plots/qpksVsTimeSt1185UUB.pdf}
  }
  \caption{\qpkvem values obtained applying the algorithm
  presented in section \ref{secFitting}. Each row shows the 
  average of the \qpkvem per day plotting as function of time,
  left UB, and right UUB version. If the label ``Selected'' is
  seen, it means that some, or all, PMTs were chosen to calculate
  the accuracy of \qpkvem.}
\end{figure}
\clearpage

hi
\begin{figure}[!b]
  \centering
  \subfigure {
    \includegraphics[width=0.49\textwidth]{../../plots/qpksVsTimeSt1190UB.pdf}
    \includegraphics[width=0.49\textwidth]{../../plots/qpksVsTimeSt1190UUB.pdf}
  }
  \subfigure {
    \includegraphics[width=0.49\textwidth]{../../plots/qpksVsTimeSt1191UB.pdf}
    \includegraphics[width=0.49\textwidth]{../../plots/qpksVsTimeSt1191UUB.pdf}
  }
  \subfigure {
    \includegraphics[width=0.49\textwidth]{../../plots/qpksVsTimeSt1198UB.pdf}
    \includegraphics[width=0.49\textwidth]{../../plots/qpksVsTimeSt1198UUB.pdf}
  }
  \subfigure {
    \includegraphics[width=0.49\textwidth]{../../plots/qpksVsTimeSt1205UB.pdf}
    \includegraphics[width=0.49\textwidth]{../../plots/qpksVsTimeSt1205UUB.pdf}
  }
  \caption{\qpkvem values obtained applying the algorithm
  presented in section \ref{secFitting}. Each row shows the 
  average of the \qpkvem per day plotting as function of time,
  left UB, and right UUB version. If the label ``Selected'' is
  seen, it means that some, or all, PMTs were chosen to calculate
  the accuracy of \qpkvem.}
\end{figure}
\clearpage

hi
\begin{figure}[!b]
  \centering
  \subfigure {
    \includegraphics[width=0.49\textwidth]{../../plots/qpksVsTimeSt1207UB.pdf}
    \includegraphics[width=0.49\textwidth]{../../plots/qpksVsTimeSt1207UUB.pdf}
  }
  \subfigure {
    \includegraphics[width=0.49\textwidth]{../../plots/qpksVsTimeSt1208UB.pdf}
    \includegraphics[width=0.49\textwidth]{../../plots/qpksVsTimeSt1208UUB.pdf}
  }
  \subfigure {
    \includegraphics[width=0.49\textwidth]{../../plots/qpksVsTimeSt1209UB.pdf}
    \includegraphics[width=0.49\textwidth]{../../plots/qpksVsTimeSt1209UUB.pdf}
  }
  \subfigure {
    \includegraphics[width=0.49\textwidth]{../../plots/qpksVsTimeSt1210UB.pdf}
    \includegraphics[width=0.49\textwidth]{../../plots/qpksVsTimeSt1210UUB.pdf}
  }
  \caption{\qpkvem values obtained applying the algorithm
  presented in section \ref{secFitting}. Each row shows the 
  average of the \qpkvem per day plotting as function of time,
  left UB, and right UUB version. If the label ``Selected'' is
  seen, it means that some, or all, PMTs were chosen to calculate
  the accuracy of \qpkvem.}
\end{figure}
\clearpage

hi
\begin{figure}[!b]
  \centering
  \subfigure {
    \includegraphics[width=0.49\textwidth]{../../plots/qpksVsTimeSt1211UB.pdf}
    \includegraphics[width=0.49\textwidth]{../../plots/qpksVsTimeSt1211UUB.pdf}
  }
  \subfigure {
    \includegraphics[width=0.49\textwidth]{../../plots/qpksVsTimeSt1213UB.pdf}
    \includegraphics[width=0.49\textwidth]{../../plots/qpksVsTimeSt1213UUB.pdf}
  }
  \subfigure {
    \includegraphics[width=0.49\textwidth]{../../plots/qpksVsTimeSt1214UB.pdf}
    \includegraphics[width=0.49\textwidth]{../../plots/qpksVsTimeSt1214UUB.pdf}
  }
  \subfigure {
    \includegraphics[width=0.49\textwidth]{../../plots/qpksVsTimeSt1216UB.pdf}
    \includegraphics[width=0.49\textwidth]{../../plots/qpksVsTimeSt1216UUB.pdf}
  }
  \caption{\qpkvem values obtained applying the algorithm
  presented in section \ref{secFitting}. Each row shows the 
  average of the \qpkvem per day plotting as function of time,
  left UB, and right UUB version. If the label ``Selected'' is
  seen, it means that some, or all, PMTs were chosen to calculate
  the accuracy of \qpkvem.}
\end{figure}
\clearpage

hi
\begin{figure}[!b]
  \centering
  \subfigure {
    \includegraphics[width=0.49\textwidth]{../../plots/qpksVsTimeSt1217UB.pdf}
    \includegraphics[width=0.49\textwidth]{../../plots/qpksVsTimeSt1217UUB.pdf}
  }
  \subfigure {
    \includegraphics[width=0.49\textwidth]{../../plots/qpksVsTimeSt1218UB.pdf}
    \includegraphics[width=0.49\textwidth]{../../plots/qpksVsTimeSt1218UUB.pdf}
  }
  \subfigure {
    \includegraphics[width=0.49\textwidth]{../../plots/qpksVsTimeSt1219UB.pdf}
    \includegraphics[width=0.49\textwidth]{../../plots/qpksVsTimeSt1219UUB.pdf}
  }
  \subfigure {
    \includegraphics[width=0.49\textwidth]{../../plots/qpksVsTimeSt1220UB.pdf}
    \includegraphics[width=0.49\textwidth]{../../plots/qpksVsTimeSt1220UUB.pdf}
  }
  \caption{\qpkvem values obtained applying the algorithm
  presented in section \ref{secFitting}. Each row shows the 
  average of the \qpkvem per day plotting as function of time,
  left UB, and right UUB version. If the label ``Selected'' is
  seen, it means that some, or all, PMTs were chosen to calculate
  the accuracy of \qpkvem.}
\end{figure}
\clearpage1

hi
\begin{figure}[!b]
  \centering
  \subfigure {
    \includegraphics[width=0.49\textwidth]{../../plots/qpksVsTimeSt1221UB.pdf}
    \includegraphics[width=0.49\textwidth]{../../plots/qpksVsTimeSt1221UUB.pdf}
  }
  \subfigure {
    \includegraphics[width=0.49\textwidth]{../../plots/qpksVsTimeSt1222UB.pdf}
    \includegraphics[width=0.49\textwidth]{../../plots/qpksVsTimeSt1222UUB.pdf}
  }
  \subfigure {
    \includegraphics[width=0.49\textwidth]{../../plots/qpksVsTimeSt1223UB.pdf}
    \includegraphics[width=0.49\textwidth]{../../plots/qpksVsTimeSt1223UUB.pdf}
  }
  \subfigure {
    \includegraphics[width=0.49\textwidth]{../../plots/qpksVsTimeSt1224UB.pdf}
    \includegraphics[width=0.49\textwidth]{../../plots/qpksVsTimeSt1224UUB.pdf}
  }
  \caption{\qpkvem values obtained applying the algorithm
  presented in section \ref{secFitting}. Each row shows the 
  average of the \qpkvem per day plotting as function of time,
  left UB, and right UUB version. If the label ``Selected'' is
  seen, it means that some, or all, PMTs were chosen to calculate
  the accuracy of \qpkvem.}
\end{figure}
\clearpage

hi
\begin{figure}[!b]
  \centering
  \subfigure {
    \includegraphics[width=0.49\textwidth]{../../plots/qpksVsTimeSt1225UB.pdf}
    \includegraphics[width=0.49\textwidth]{../../plots/qpksVsTimeSt1225UUB.pdf}
  }
  \subfigure {
    \includegraphics[width=0.49\textwidth]{../../plots/qpksVsTimeSt1227UB.pdf}
    \includegraphics[width=0.49\textwidth]{../../plots/qpksVsTimeSt1227UUB.pdf}
  }
  \subfigure {
    \includegraphics[width=0.49\textwidth]{../../plots/qpksVsTimeSt1729UB.pdf}
    \includegraphics[width=0.49\textwidth]{../../plots/qpksVsTimeSt1729UUB.pdf}
  }
  \subfigure {
    \includegraphics[width=0.49\textwidth]{../../plots/qpksVsTimeSt1733UB.pdf}
    \includegraphics[width=0.49\textwidth]{../../plots/qpksVsTimeSt1733UUB.pdf}
  }
  \caption{\qpkvem values obtained applying the algorithm
  presented in section \ref{secFitting}. Each row shows the 
  average of the \qpkvem per day plotting as function of time,
  left UB, and right UUB version. If the label ``Selected'' is
  seen, it means that some, or all, PMTs were chosen to calculate
  the accuracy of \qpkvem.}
\end{figure}
\clearpage

hi
\begin{figure}[!b]
  \centering
  \subfigure {
    \includegraphics[width=0.49\textwidth]{../../plots/qpksVsTimeSt1735UB.pdf}
    \includegraphics[width=0.49\textwidth]{../../plots/qpksVsTimeSt1735UUB.pdf}
  }
  \subfigure {
    \includegraphics[width=0.49\textwidth]{../../plots/qpksVsTimeSt1736UB.pdf}
    \includegraphics[width=0.49\textwidth]{../../plots/qpksVsTimeSt1736UUB.pdf}
  }
  \subfigure {
    \includegraphics[width=0.49\textwidth]{../../plots/qpksVsTimeSt1737UB.pdf}
    \includegraphics[width=0.49\textwidth]{../../plots/qpksVsTimeSt1737UUB.pdf}
  }
  \subfigure {
    \includegraphics[width=0.49\textwidth]{../../plots/qpksVsTimeSt1738UB.pdf}
    \includegraphics[width=0.49\textwidth]{../../plots/qpksVsTimeSt1738UUB.pdf}
  }
  \caption{\qpkvem values obtained applying the algorithm
  presented in section \ref{secFitting}. Each row shows the 
  average of the \qpkvem per day plotting as function of time,
  left UB, and right UUB version. If the label ``Selected'' is
  seen, it means that some, or all, PMTs were chosen to calculate
  the accuracy of \qpkvem.}
\end{figure}
\clearpage

hi
\begin{figure}[!b]
  \centering
  \subfigure {
    \includegraphics[width=0.49\textwidth]{../../plots/qpksVsTimeSt1739UB.pdf}
    \includegraphics[width=0.49\textwidth]{../../plots/qpksVsTimeSt1739UUB.pdf}
  }
  \subfigure {
    \includegraphics[width=0.49\textwidth]{../../plots/qpksVsTimeSt1740UB.pdf}
    \includegraphics[width=0.49\textwidth]{../../plots/qpksVsTimeSt1740UUB.pdf}
  }
  \subfigure {
    \includegraphics[width=0.49\textwidth]{../../plots/qpksVsTimeSt1741UB.pdf}
    \includegraphics[width=0.49\textwidth]{../../plots/qpksVsTimeSt1741UUB.pdf}
  }
  \subfigure {
    \includegraphics[width=0.49\textwidth]{../../plots/qpksVsTimeSt1742UB.pdf}
    \includegraphics[width=0.49\textwidth]{../../plots/qpksVsTimeSt1742UUB.pdf}
  }
  \caption{\qpkvem values obtained applying the algorithm
  presented in section \ref{secFitting}. Each row shows the 
  average of the \qpkvem per day plotting as function of time,
  left UB, and right UUB version. If the label ``Selected'' is
  seen, it means that some, or all, PMTs were chosen to calculate
  the accuracy of \qpkvem.}
\end{figure}
\clearpage

hi
\begin{figure}[!b]
  \centering
  \subfigure {
    \includegraphics[width=0.49\textwidth]{../../plots/qpksVsTimeSt1743UB.pdf}
    \includegraphics[width=0.49\textwidth]{../../plots/qpksVsTimeSt1743UUB.pdf}
  }
  \subfigure {
    \includegraphics[width=0.49\textwidth]{../../plots/qpksVsTimeSt1744UB.pdf}
    \includegraphics[width=0.49\textwidth]{../../plots/qpksVsTimeSt1744UUB.pdf}
  }
  \subfigure {
    \includegraphics[width=0.49\textwidth]{../../plots/qpksVsTimeSt1745UB.pdf}
    \includegraphics[width=0.49\textwidth]{../../plots/qpksVsTimeSt1745UUB.pdf}
  }
  \subfigure {
    \includegraphics[width=0.49\textwidth]{../../plots/qpksVsTimeSt1746UB.pdf}
    \includegraphics[width=0.49\textwidth]{../../plots/qpksVsTimeSt1746UUB.pdf}
  }
  \caption{\qpkvem values obtained applying the algorithm
  presented in section \ref{secFitting}. Each row shows the 
  average of the \qpkvem per day plotting as function of time,
  left UB, and right UUB version. If the label ``Selected'' is
  seen, it means that some, or all, PMTs were chosen to calculate
  the accuracy of \qpkvem.}
\end{figure}
\clearpage

hi
\begin{figure}[!b]
  \centering
  \subfigure {
    \includegraphics[width=0.49\textwidth]{../../plots/qpksVsTimeSt1747UB.pdf}
    \includegraphics[width=0.49\textwidth]{../../plots/qpksVsTimeSt1747UUB.pdf}
  }
  \subfigure {
    \includegraphics[width=0.49\textwidth]{../../plots/qpksVsTimeSt1779UB.pdf}
    \includegraphics[width=0.49\textwidth]{../../plots/qpksVsTimeSt1779UUB.pdf}
  }
  \subfigure {
    \includegraphics[width=0.49\textwidth]{../../plots/qpksVsTimeSt1791UB.pdf}
    \includegraphics[width=0.49\textwidth]{../../plots/qpksVsTimeSt1791UUB.pdf}
  }
  \subfigure {
    \includegraphics[width=0.49\textwidth]{../../plots/qpksVsTimeSt1798UB.pdf}
    \includegraphics[width=0.49\textwidth]{../../plots/qpksVsTimeSt1798UUB.pdf}
  }
  \caption{\qpkvem values obtained applying the algorithm
  presented in section \ref{secFitting}. Each row shows the 
  average of the \qpkvem per day plotting as function of time,
  left UB, and right UUB version. If the label ``Selected'' is
  seen, it means that some, or all, PMTs were chosen to calculate
  the accuracy of \qpkvem.}
\end{figure}
\clearpage

hi
\begin{figure}[!b]
  \centering
  \subfigure {
    \includegraphics[width=0.49\textwidth]{../../plots/qpksVsTimeSt1817UB.pdf}
    \includegraphics[width=0.49\textwidth]{../../plots/qpksVsTimeSt1817UUB.pdf}
  }
  \subfigure {
    \includegraphics[width=0.49\textwidth]{../../plots/qpksVsTimeSt1818UB.pdf}
    \includegraphics[width=0.49\textwidth]{../../plots/qpksVsTimeSt1818UUB.pdf}
  }
  \subfigure {
    \includegraphics[width=0.49\textwidth]{../../plots/qpksVsTimeSt1819UB.pdf}
    \includegraphics[width=0.49\textwidth]{../../plots/qpksVsTimeSt1819UUB.pdf}
  }
  \subfigure {
    \includegraphics[width=0.49\textwidth]{../../plots/qpksVsTimeSt1851UB.pdf}
    \includegraphics[width=0.49\textwidth]{../../plots/qpksVsTimeSt1851UUB.pdf}
  }
  \caption{\qpkvem values obtained applying the algorithm
  presented in section \ref{secFitting}. Each row shows the 
  average of the \qpkvem per day plotting as function of time,
  left UB, and right UUB version. If the label ``Selected'' is
  seen, it means that some, or all, PMTs were chosen to calculate
  the accuracy of \qpkvem.}
\end{figure}
\clearpage

hi
\begin{figure}[!b]
  \centering
  \subfigure {
    \includegraphics[width=0.49\textwidth]{../../plots/qpksVsTimeSt1854UB.pdf}
    \includegraphics[width=0.49\textwidth]{../../plots/qpksVsTimeSt1854UUB.pdf}
  }
  \subfigure {
    \includegraphics[width=0.49\textwidth]{../../plots/qpksVsTimeSt1878UB.pdf}
    \includegraphics[width=0.49\textwidth]{../../plots/qpksVsTimeSt1878UUB.pdf}
  }
  \subfigure {
    \includegraphics[width=0.49\textwidth]{../../plots/qpksVsTimeSt1880UB.pdf}
    \includegraphics[width=0.49\textwidth]{../../plots/qpksVsTimeSt1880UUB.pdf}
  }
  \subfigure {
    \includegraphics[width=0.49\textwidth]{../../plots/qpksVsTimeSt1881UB.pdf}
    \includegraphics[width=0.49\textwidth]{../../plots/qpksVsTimeSt1881UUB.pdf}
  }
  \caption{\qpkvem values obtained applying the algorithm
  presented in section \ref{secFitting}. Each row shows the 
  average of the \qpkvem per day plotting as function of time,
  left UB, and right UUB version. If the label ``Selected'' is
  seen, it means that some, or all, PMTs were chosen to calculate
  the accuracy of \qpkvem.}
\end{figure}
\clearpage


\end{document}
