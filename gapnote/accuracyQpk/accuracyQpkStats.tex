\documentclass[twoside, final, 10pt]{articleMine}
\usepackage[english]{babel}
\usepackage[sc]{mathpazo}
\usepackage{a4wide}
\usepackage{subfigure}

\usepackage{hyperref}
\usepackage{amsmath,amssymb}
\usepackage[right]{lineno}
\usepackage{xspace}
\usepackage{accents}
\usepackage{graphicx}
\usepackage{booktabs}
\usepackage{color}
\usepackage{units}
\usepackage{enumitem}
\usepackage{todonotes}
\usepackage[capitalize]{cleveref}
\usepackage{nameref}
\usepackage{csquotes}

\usepackage{listings}
\usepackage{color}

\definecolor{dkgreen}{rgb}{0,0.6,0}
\definecolor{gray}{rgb}{0.5,0.5,0.5}
\definecolor{mauve}{rgb}{0.58,0,0.82}

\lstset{frame=tb,
  language=XML,
  aboveskip=3mm,
  belowskip=3mm,
  showstringspaces=false,
  columns=flexible,
  basicstyle={\small\ttfamily},
  numbers=none,
  numberstyle=\tiny\color{gray},
  keywordstyle=\color{blue},
  commentstyle=\color{dkgreen},
  stringstyle=\color{mauve},
  breaklines=true,
  breakatwhitespace=true,
  tabsize=3
}

\usepackage[thinc]{esdiff}


\usepackage{booktabs}
\usepackage{mcite}


\graphicspath{{plots/}}
\newcommand*\patchAmsMathEnvironmentForLineno[1]{%
  \expandafter\let\csname old#1\expandafter\endcsname\csname #1\endcsname
  \expandafter\let\csname oldend#1\expandafter\endcsname\csname end#1\endcsname
  \renewenvironment{#1}%
     {\linenomath\csname old#1\endcsname}%
     {\csname oldend#1\endcsname\endlinenomath}}%
\newcommand*\patchBothAmsMathEnvironmentsForLineno[1]{%
  \patchAmsMathEnvironmentForLineno{#1}%
  \patchAmsMathEnvironmentForLineno{#1*}}%
\AtBeginDocument{%
\patchBothAmsMathEnvironmentsForLineno{equation}%
\patchBothAmsMathEnvironmentsForLineno{align}%
\patchBothAmsMathEnvironmentsForLineno{flalign}%
\patchBothAmsMathEnvironmentsForLineno{alignat}%
\patchBothAmsMathEnvironmentsForLineno{gather}%
\patchBothAmsMathEnvironmentsForLineno{multline}%
}

\newcommand{\putat}[3]{\begin{picture}(0,0)(0,0)\put(#1,#2){#3}\end{picture}} % just a shorthand
\parskip 4.2pt           % sets spacing between paragraphs
%\parindend 0pt           % sets spacing between paragraphs
\def\Offline{\mbox{$\overline{\rm
Off}$\hspace{.05em}\raisebox{.4ex}{$\underline{\rm line}$}}\xspace}
\def\OfflineB{\mbox{$\bf\overline{\rm\bf
Off}$\hspace{.05em}\raisebox{.4ex}{$\bf\underline{\rm\bf line}$}}\xspace}

\newcommand{\qpkDist}{$Q^\mathrm{Pk}_\mathrm{Dist}$\,}
\newcommand{\qpkNormDist}{$Q^\mathrm{Pk}_\mathrm{NormDist}$\,}

% equations ...
\newcommand{\be}{\begin{equation}}
\newcommand{\ee}{\end{equation}}
\newcommand{\ben}{\begin{enumerate}}
\newcommand{\een}{\end{enumerate}}
\newcommand{\bi}{\begin{itemize}}
\newcommand{\ei}{\end{itemize}}
\newcommand{\bbe}{\begin{equation*}}
\newcommand{\eee}{\end{equation*}}
\newcommand{\bber}{\begin{equation*}\textcolor{red}}
\newcommand{\eeer}{\end{equation*}}
\newcommand{\bc}{\begin{center}}
\newcommand{\ec}{\end{center}}
\newcommand{\bea}{\begin{eqnarray}}
\newcommand{\eea}{\end{eqnarray}}
\newcommand{\bem}{\begin{pmatrix}}
\newcommand{\eem}{\end{pmatrix}}
\newcommand{\bbea}{\begin{eqnarray*}}
\newcommand{\eeea}{\end{eqnarray*}}
\newcommand{\bcols}{\begin{columns}[T]}
\newcommand{\ecols}{\end{columns}}
\newcommand{\bcol}[1]{\begin{column}{#1\textwidth}}
\newcommand{\ecol}{\end{column}}
%variance
\newcommand{\V}[1]{\mathrm{V}[{#1}]}
% efficiency:
\newcommand{\eps}{\ensuremath\varepsilon}

% escape math mode:
\newcommand{\mr}[1]{\mathrm{#1}}

% bold face:
\newcommand{\mbf}[1]{\mathbf{#1}}

% N_e:
\newcommand{\Ne}{\ensuremath N_\mathrm{e}}
% dEdX:
\newcommand{\dEdX}{\ensuremath\mathrm{d}E/\mathrm{d}X}
% Xmax:
\newcommand{\Xmax}{\ensuremath X_\mathrm{max}}
% meanXmax:
\newcommand{\meanXmax}{\ensuremath \langle X_\mathrm{max}\rangle}
% meanlnA:
\newcommand{\lnA}{\ensuremath \langle \ln \mathrm{A}\rangle}
% reference a float:
\newcommand{\rf}[2]{\mbox{#1 \ref{#2} }}

% microns
%%%%%%%%%%%%%%%%%
% Math Italic A %
%%%%%%%%%%%%%%%%%
\def\re@DeclareMathSymbol#1#2#3#4{%
    \let#1=\undefined
    \DeclareMathSymbol{#1}{#2}{#3}{#4}}

\DeclareSymbolFont{lettersA}{U}{pxmia}{m}{it}
\SetSymbolFont{lettersA}{bold}{U}{pxmia}{bx}{it}
\DeclareFontSubstitution{U}{pxmia}{m}{it}

\DeclareSymbolFontAlphabet{\mathfrak}{lettersA}
\re@DeclareMathSymbol{\muup}{\mathord}{lettersA}{"16}


% microns and microseconds
\newcommand{\mum}{\ensuremath\muup m}
\newcommand{\mus}{\ensuremath\muup s}
\newcommand{\gcm}{g/cm$^2$}

%programs
\newcommand{\GEANE}{\texttt{GEANE}}
\newcommand{\GEANT}{\texttt{GEANT}}
\newcommand{\lcgen}{\texttt{l3cgen}}
\newcommand{\CORSIKA}{\texttt{CORSIKA}}
\newcommand{\VENUS}{\texttt{VENUS}}
\newcommand{\GHEISHA}{\texttt{GEISHA}}

%Calibration
\newcommand{\qpkvem}{$Q^\mathrm{Peak}_\mathrm{VEM}$\,}

%class 1
\newcommand{\cone}{\mbox{class 1}}
\newcommand{\mc}[3]{\multicolumn{#1}{#2}{#3}}
\newcommand{\mcb}[2]{\multicolumn{#1}{c}{#2}}
\newcommand{\ttp}[2]{#1$\cdot$10$^#2$}
\newcommand{\bad}{\textcolor{red}{$\ominus$}}
\newcommand{\ok}{\textcolor{green}{$\oplus$}}
% misc abbreviations

\newcommand{\s}{$\:$}

\newcommand{\ra}{\ensuremath\rightarrow}
\newcommand{\vsp}[1]{\vspace*{#1cm}}
\newcommand{\hsp}[1]{\hspace*{#1cm}}
\def\s1000{S(\unit[1000]{m})}
%%%%%%%%%%%%%%%%%%%%%%%%%%%%%%%%%%%%%%%%%%%%%%%%%%
%
%  some abbreviations from physics.sty
%
%%%%%%%%%%%%%%%%%%%%%%%%%%%%%%%%%%%%%%%%%%%%%%%%%%


\def\EeV{\ifmmode {\mathrm{\ Ee\kern -0.1em V}}\else
                   \textrm{Ee\kern -0.1em V}\fi}%
\def\PeV{\ifmmode {\mathrm{\ Pe\kern -0.1em V}}\else
                   \textrm{Pe\kern -0.1em V}\fi}%
\def\TeV{\ifmmode {\mathrm{\ Te\kern -0.1em V}}\else
                   \textrm{Te\kern -0.1em V}\fi}%
\def\MeV{\ifmmode {\mathrm{\ Me\kern -0.1em V}}\else
                   \textrm{Me\kern -0.1em V}\fi}%
\def\GeV{\ifmmode {\mathrm{\ Ge\kern -0.1em V}}\else
                   \textrm{Ge\kern -0.1em V}\fi}%
\def\keV{\ifmmode {\mathrm{\ ke\kern -0.1em V}}\else
                   \textrm{ke\kern -0.1em V}\fi}%
\def\MeV{\ifmmode {\mathrm{\ Me\kern -0.1em V}}\else
                   \textrm{Me\kern -0.1em V}\fi}%
\def\eV{\ifmmode {\mathrm{\ e\kern -0.1em V}}\else
                   \textrm{e\kern -0.1em V}\fi}%
\def\Zo{\ensuremath{\mathrm {Z}}}
\def\Wp{\ensuremath{\mathrm {W^+}}}
\def\Wm{\ensuremath{\mathrm {W^-}}}
\def\epem{\ensuremath{\mathrm{e^+e^-}}}%
\def\mm{\ensuremath{\mathrm{\mu^+ \mu^-}}}%
\def\antibar#1{\ensuremath{#1\bar{#1}}}%
\def\nbar{\ensuremath{\bar{\nu}}}
\def\nnbar{\antibar{\nu}}%

% offline
\def\Offline{\mbox{$\overline{\rm
Off}$\hspace{.05em}\raisebox{.3ex}{$\underline{\rm line}$}}\xspace}
\def\OfflineB{\mbox{$\bf\overline{\rm\bf
Off}$\hspace{.05em}\raisebox{.2ex}{$\bf\underline{\rm\bf line}$}}\xspace}
\newcommand{\HRule}{\rule{\linewidth}{1mm}}

%%% Local Variables: 
%%% mode: latex
%%% TeX-master: t
%%% End: 



\begin{document}
\setpagewiselinenumbers
\modulolinenumbers[2]

\linenumbers

\renewcommand\linenumberfont{\small\rmfamily}
\begin{flushright}
%GAP-2021-xx
\end{flushright}

\begin{flushright}
  \rule{\linewidth}{0.5mm}
  \\[17mm]
  \vspace*{-3ex}{\Large Accuracy of $Q^\mathrm{Peak}_\mathrm{VEM}$ fit for UB and UUB}
  \large
  \parbox[b]{15cm}
  {
    \begin{flushright}
      Mauricio Su\'arez-Dur\'an and  Ioana~C.~Mari\c{s}
      \\[6mm]
      {\small Universit{\'e} Libre de Bruxelles, Belgium}
    \end{flushright}
  }
  \\[5mm]
  \rule{\linewidth}{0.5mm}
\end{flushright}
%
%\vspace*{-5ex}
%
\thispagestyle{empty}
\noindent

\begin{abstract}
  \noindent
  From the deployment of the Upgraded Unified Board (UUB) it is 
  expected to have a better accuracy for the calculation of the
  \qpkvem values, respect of the previous Unified Board (UB).
  Here, a method to calculate the accuracy of the \qpkvem values
  for the UUB and UB is presented, including a new method
  to fit the muon hump in charge calibration histogram. In this
  way, the accuracy was calculate for $75$ UUB stations, and
  their respective UB version.
\end{abstract}


\thispagestyle{empty}
$\;$
%\listoftodos
%\newpage
\noindent
\clearpage

\section{Introduction}
The deployment of the Upgraded Unified Board (UUB) for the
surface array\,\cite{augerPrimeDesign} requires a systematic
monitoring and calibration, using as reference the previous
Unified Board (UB). The \qpkvem is one of the more important
quantities to calculate and monitoring because it is used to
normalize the SD signals, making possible the comparison among
different WCDs, and also to understand the signals
fluctuations\,\cite{gap2003-030}.\\\\In this work, the accuracy
of the \qpkvem is calculated as a checker of its stability,
extracting its value through fitting the muon hump in the charge
calibration histograms\,\cite{BERTOU2006839} and comparing
between UB and UUB. In order to guarantee the precision in this
calculation, two methods have been implemented, first one
regarding the fitting of the muon hump, and second one to chose a
time window of seven days in a row which the fitted \qpkvem
values were stable, in both version UB and UUB.

\section{Fitting the muon hump}
\label{secFitting}

\begin{figure}[!t]
  \label{figChargeDerivative}
  \centering
  \subfigure {
    \includegraphics[width=.5\textwidth]{../../plots/chargeHisto863.pdf}
    \includegraphics[width=.5\textwidth]{../../plots/chargeDerHisto863.pdf}
  }
  \caption{Algorithm to fit the muon hump in calibration charge
  histograms (see details in the text). Left: a sample of a
  typical charge calibration histogram, the black line represents
  the smoothing histogram after applying a 15-bin sliding window
  ($H_S$). Right: in black the first derivative histogram
  ($H_{DS}$) and its respective smoothing in red ($H_{SDS}$).
  Here, the green vertical line shows the maximum or the first
  approach for \qpkvem.}
  \label{figChargeDerivative}
\end{figure}

The \qpkvem value is obtained from charge calibration histograms
by fitting the muon hump. We have implemented an algorithm which
uses the first derivative to perform this fit, obtaining in this
way a first approach of the hump position, and from there the
fitting range. The figure \ref{figChargeDerivative} illustrated
this algorithm, which detailed next:
\begin{enumerate}
  \item Smoothing the histogram using a 15-bin sliding window,
    $H_S$.
  \item Obtaining the first derivative from $H_S$, applying
    \begin{equation}
      \frac{f(x+1)-f(x-1)}{2h} \, ,
    \end{equation}
    and named as $H_{DS}$.
  \item Smoothing $H_{DS}$, by 15-bin sliding window, and
    getting $H_{SDS}$.
  \item Searching for the estimated \qpkvem, i.e. first bin for
    $H_{SDS}$ equal to zero, from right to left.
  \item Fixing the fitting range using n-bin leftward and n-bin
    rightward from the estimated \qpkvem.
\end{enumerate}
\clearpage 

\begin{figure}[!t]
  \label{figFitStabNbins}
  \centering
  \subfigure {
    \includegraphics[width=.5\textwidth]{../../plots/uubChRmsFitBinsLrSt863.pdf}
    \includegraphics[width=.5\textwidth]{../../plots/uubChRmsFitBinsLrSt863zoom.pdf}
  }
  \subfigure{
    \includegraphics[width=.5\textwidth]{../../plots/chargeFitPoly2863.pdf}
    \includegraphics[width=.5\textwidth]{../../plots/chargeFitResidualsPoly2863.pdf}
  }
  \caption{Top row, results for fitting stability of the hump
  muon, RMS/$\left<Q^\mathrm{Peak}_\mathrm{VEM}\right>$, as
  function of the n-bin. After 30 bins the stability is reached.
  Bottom row, an example of application of the algorithm; here
  the vertical green line shows the hump VEM obtained from the
  first derivative (step 4.), and the vertical blue line shows
  the \qpkvem obtained as the maximum of the fitted second order
  polynomial.}
  \label{figFitStabNbins}
\end{figure}

The last step in the algorithm requires an extra procedure in
order to fixed the number of n-bin. So, a set of histograms have
been fitted using different values for n, and then the stability
of the fit,
RMS$_\mathrm{n}$/$\left<Q^\mathrm{Peak}_\mathrm{VEM,\,n}
\right>$, have been checked; avoiding to fit the valley of the
histogram. The results of this procedure are presented in
figure \ref{figFitStabNbins}. There, the stability of the fit is
reached after 30-bin, which chosen as the number of n-bin for the
step 5 of the algorithm. In the bottom row of the same figure, an
example of applying this procedure is
presented.
%Section \ref{secQpkVsTime} contains the \qpkvem values
%obtained by this method for 75 UUB stations and for the same
%stations but in UB version, using the same algorithm.

\section{Sliding window algorithm for \qpkvem accuracy
calculation}
\label{secSlidWindowAlgo}

The calculation of the accuracy is based on the selection of a
period of time in which the \qpkvem as a function of time was
stable, in this case we have chosen a full week (7-day-row). To
determine this time window, we have used the charge calibration
histograms measured from August to November, 2021 for UUB and
2018 for UB. From these set of data, the average of \qpkvem per
day was plotted as function of time for each one of the 75, UUB
and UB.\\\\In order to chose the respective 7-day-row with a good
statistic, a first condition was set in terms of the minimum of
\qpkvem values per day needed in each day. To apply this
condition, the distribution of \qpkvem per day was calculate, and
it is presented in figure \ref{figDistQpkPerDay}, from which we
have chosen as $10$ the minimum of \qpkvem values per day.
\clearpage

\begin{figure}[!t]
  \label{figDistQpkPerDay}
  \centering
  \subfigure {
    \includegraphics[width=.8\textwidth]{../../plots2/qpkDistPerDayUbUub.pdf}
  }
  \caption{Distribution of \qpkvem per day for 75 stations, from
  August to November, with UUB (red, year 2021) and UB (blue,
  year 2018) electronics version. The black line represents the
  respective Gaussian fit.}
  \label{figDistQpkPerDay}
\end{figure}

\noindent The sliding window algorithm was applied to each
station, and each PMT, following next steps:
\begin{enumerate}
  \item The $\left< Q^{\mathrm{Peak}}_{\mathrm{VEM}}\right>$ per
    day as function of time is calculate from August to November.
  \item Starting from the first day of August, a
    [7-day-series]$_0$ is build.
  \item A check for continuity is applied, i.e. if each day has
    not more then 10 \qpkvem values, or the 7 days are not
    consecutive (some day has not data), a new series is built,
    e.g. if series $i$ has a discontinuity in day 3 jumping to
    day 5, a new 7-day-series is calculated from day 5.
  \item If [7-day-series]$_i$ is consecutive, a linear fit is
    applied, and the respective slope and p-value are stored.
\end{enumerate}

\noindent The figure \ref{figLogPvalVsSlop} shows the
distribution for the p-values as function of the normalized slope
(respect the average of the \qpkvem during the 7 days), obtained
by applying the former algorithm to the 75 stations, all PMTs.
There, it is seen how the normalized slope is closer to zero for
UB than UUB, whereas for both version the $\log_{10}($p-value$)$
distribution has a hot spot for values bigger than $-4$.\\\\With
these results, and taking into the account that the number of
degree of freedom is 5 for all the fits\footnote{Here, only fits
with 5 DOF were used; different ones were rejected.}, we chose
all fits matching with: $\chi^2<10$, i.e.
$\log_{10}($p-value$)<-1.12$, and a normalized slope between
$\mu\pm\sigma$ (x axis, figure \ref{figLogPvalVsSlop}). The
results after these cuts are presented in figure
\ref{figProjSlop}, where the distribution of the normalized slope
is presented for both UB and UUB. With these results, we can
chose the best 7-day-row to perform the accuracy calculation.
\clearpage

\begin{figure}[!t]
  \label{figLogPvalVsSlop}
  \centering
  \subfigure {
    \includegraphics[width=0.55\textwidth]{../../plots2/chi2VsSlopUb2.pdf}
    \includegraphics[width=0.55\textwidth]{../../plots2/chi2VsSlopUub2.pdf}
  }
  \caption{Distribution of p-value as function of the normalized
  slope (i.e. divided by the average of the \qpkvem during the 7
  days). Values obtained after applied the sliding window method
  to 75 stations, all PMT (see text for details). On the left the
  distribution for UB, and on the right the one for UUB. A hot
  spot for p-values bigger than $-4$ is observed for UB and UUB,
  whereas the normalized slope closer to zero for UB than UUB.}
  \label{figLogPvalVsSlop}
\end{figure}
\section{Accuracy calculation}

The calculation for the \qpkvem accuracy to each station was
perform following the next steps for each station, for both UB
and UUB version:
\begin{enumerate}
  \item The best 7-day-row is chosen per PMT as the one with the
    normalized slope closest to zero and
    $\log_{10}($p-value$)<-1.12$.
  \item If for a certain station, either UB or UUB, some PMT has
    not a 7-day-row in agreement with the previous requirements,
    this PMT is remove for the calculation from both UB and UUB.
  \item To each PMT, a singular normalized distribution of the
    respective \qpkvem values is built; into the respective.
  \item A Gaussian function is fitted to each PMT normalized
    distribution and then the accuracy is calculated as:
    $\sigma/\mu$, respectively.
\end{enumerate}
An example of the steps 3 and 4 of this method are illustrated
in figure \ref{figQpkNormalizedSt1224} for UB and UUB. There, it
is possible to see that the distribution of the normalized
\qpkvem values fits to a Gaussian function, for the both
versions.\\\\
\clearpage


\begin{figure}[!t]
  \label{figProjSlop}
  \centering
  \subfigure {
    \includegraphics[width=0.55\textwidth]{../../plots2/chi2VsSlopUbProjSlop2.pdf}
    \includegraphics[width=0.55\textwidth]{../../plots2/chi2VsSlopUubProjSlop2.pdf}
  }
  \caption{Distribution of normalized slopes fitted for
  [7-days-row]$_i$ matching: $\chi^2<10$
  ($\log_{10}($p-value$)>-1.12$), and normalized slope
  $\mu\pm\sigma$ (x axis, figure  \ref{figLogPvalVsSlop}). For UB
  case, the mean of the distribution is closer to zero than UUB
  case.}
  \label{figProjSlop}
\end{figure}

\noindent The results for the \qpkvem accuracy are presented in
figure \ref{figAccuracyResults}, where the accuracy per station
is presented on the left, whereas the its distribution is showed
on the right panel. There, it is seen that the accuracy for UUB
takes a value of $1.61\,\%\pm0.06\,\%$, and $1.36\,\%\pm0.06\,\%$
for UB. The latest value differ from the $2\,\%$ calculated in
\cite{gap2002-046} for the engineering array.\\\\On the other
hand, in figure \ref{figDiffAcc} the difference between the
accuracy for UB minus the one for UUB is presented per station,
left plot, together its distribution, right plot. There, it is
possible to confirm that the \qpkvem accuracy for UUB stations
are in general bigger than the ones for UBs, with a mean value
for this difference of $-0.252\,\%\pm0.052\,\%$.\\\\Examples of
ourliers in figure \ref{figAccuracyResults}, and in figure
\ref{figDiffAcc} are presented in figures \ref{figOutlier1223}
and \ref{figOutlier1224}, respectively.
\clearpage

\begin{figure}[!t]
  \label{figQpkNormalizedSt1224}
  \centering
  \subfigure {
    \includegraphics[width=0.55\textwidth]{../../plots2/qpkdDistUBSt1224.pdf}
    \includegraphics[width=0.55\textwidth]{../../plots2/qpkdDistUUBSt1224.pdf}
  }
  \subfigure {
    \includegraphics[width=0.55\textwidth]{../../plots2/qpkdNormDistUBSt1224.pdf}
    \includegraphics[width=0.55\textwidth]{../../plots2/qpkdNormDistUUBSt1224.pdf}
  }  
  \caption{Example of the method applied to calculate the
  accuracy of the \qpkvem, station 1224. Upper row, distribution
  of the \qpkvem values from the best 7-day-raw selected; left
  UB, right UUB. Lower row, former distribution normalized
  respect the average of the respective PMT. The red line
  represents the Gauss function fitted.}
  \label{figQpkNormalizedSt1224}
\end{figure}

\begin{figure}[!t]
  \label{figAccuracyResults}
  \centering
  \subfigure {
    \includegraphics[width=.55\textwidth]{../../plots2/accQpkFitUbUubPerSt_Stats2.pdf}
    \includegraphics[width=.55\textwidth]{../../plots2/accQpkFitUbUubAllStAllPmt_Stats2.pdf}
  }
  \caption{Results for the accuracy calculation of the \qpkvem
  fitting, using the algorithm describe in section 
  \ref{secSlidWindowAlgo}; in blue color for UB, and in red color
  for UUB. In the right side, the distribution of the
  $\sigma/\mu$ values is presented for both version, UUB (red)
  and UB (blue). Here, it is possible to see that the accuracy
  for UUB is better than the one for UB.}
  \label{figAccuracyResults}
\end{figure}
\textcolor{white}{hi}
\clearpage

\begin{figure}
  \label{figDiffAcc}
  \centering
  \subfigure {
    \includegraphics[width=0.55\textwidth]{../../plots2/accQpkFitUbUubDiffPerSt_Stats2.pdf}
    \includegraphics[width=0.55\textwidth]{../../plots2/accQpkFitUbUubDistDiff_Stats2.pdf}
  }
  \caption{Difference between the accuracy for UB minus the one
  for UUB. Left, the plot of this difference per station. Right,
  its distribution. Here, it is possible to see how per UB
  stations the accuracy is lower than the one for the respective
  UUB version.}
  \label{figDiffAcc}
\end{figure}

\begin{figure}[!t]
  \label{figOutlier1223}
  \centering
  \subfigure {
    \includegraphics[width=0.55\textwidth]{../../plots2/qpkdDistUBSt1223.pdf}
    \includegraphics[width=0.55\textwidth]{../../plots2/qpkdDistUUBSt1223.pdf}
  }
  \subfigure {
    \includegraphics[width=0.55\textwidth]{../../plots2/qpkdNormDistUBSt1223.pdf}
    \includegraphics[width=0.55\textwidth]{../../plots2/qpkdNormDistUUBSt1223.pdf}
  }  
  \caption{Station 1223 with an outlier accuracy ($>10\,\%$).
  Here, it is possible to see how the PMT3 produce a wide
  distribution for the normalized \qpkvem.}
  \label{figOutlier1223}
\end{figure}

\textcolor{white}{hi}
\clearpage

\begin{figure}[!t]
  \label{figOutlier1224}
  \centering
  \subfigure {
    \includegraphics[width=0.55\textwidth]{../../plots2/qpkdDistUBSt1224.pdf}
    \includegraphics[width=0.55\textwidth]{../../plots2/qpkdDistUUBSt1224.pdf}
  }
  \subfigure {
    \includegraphics[width=0.55\textwidth]{../../plots2/qpkdNormDistUBSt1224.pdf}
    \includegraphics[width=0.55\textwidth]{../../plots2/qpkdNormDistUUBSt1224.pdf}
  }  
  \caption{Station 1224 with an outlier value for the difference
  between accuracy UB minus accuracy UUB ($<-1\,\%$). Here, it is
  seen that the normalized distribution of the \qpkvem values is
  wider in the UUB version than in UB one.}
  \label{figOutlier1224}
\end{figure}

\textcolor{white}{hi}
\clearpage

\section{Summary and conclusions}
We have developed and presented a different method to fit the
muon hump in charge calibration histograms, and from there to
estimate the respective \qpkvem value, for UUB and UB
version; as it was presented in figures \ref{figChargeDerivative}
and \ref{figFitStabNbins}.\\\\A sliding window method was
implemented to find the most stable period for the \qpkvem
values, i.e. during 7 consecutive days in a row; for both
version, UB and UUB.\\\\We have calculate the accuracy of the
\qpkvem using the normalized distribution of the \qpkvem values
for the most stable period. From there, we have calculated the
accuracy for UUB version as $1.61\,\%\pm0.06\,\%$, and
$1.36\,\%\pm0.06\,\%$ for UB, which differ from the $2\,\%$
calculated in \cite{gap2002-046} for the engineering array.



\begin{thebibliography}{}
  \setlength{\itemsep}{0.0pt}
    \bibitem{augerPrimeDesign}
      A.~Aab \textit{et al.}, ``The Pierre Auger Observatory
      Upgrade - Preliminary Design Report,'' arXiv:1604.03637
      [astro-ph.IM].
    \bibitem{gap2003-030} M. Ave, P. Bauleo, T. Yamamoto,
      {\em Signal Fluctuation in the Auger Surface Detector
      Array}, GAP 2003-030
    \bibitem{BERTOU2006839} X. Bertou and P.S. Allison 
      and C. Bonifazi and \textit{et al.}, {\em Calibration of
      the surface array of the Pierre Auger Observatory}, Nuclear
      Instruments and Methods in Physics Research Section A:
      Accelerators, Spectrometers, Detectors and Associated
      Equipment, vol. 568, number 2, 2006 
    \bibitem{gap2002-046} A. Tripathi, K. Arisaka, M. Healy, D.
      Barnhill, W. Slater, {\em A Systematic Calibration of
      Surface Detectors using Muon Data from the Engineering
      Array}, GAP 2002-046
\end{thebibliography}

\end{document}
