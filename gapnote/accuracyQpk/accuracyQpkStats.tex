\documentclass[twoside, final, 10pt]{articleMine}
\usepackage[english]{babel}
\usepackage[sc]{mathpazo}
\usepackage{a4wide}
\usepackage{subfigure}

\usepackage{hyperref}
\usepackage{amsmath,amssymb}
\usepackage[right]{lineno}
\usepackage{xspace}
\usepackage{accents}
\usepackage{graphicx}
\usepackage{booktabs}
\usepackage{color}
\usepackage{units}
\usepackage{enumitem}
\usepackage{todonotes}
\usepackage[capitalize]{cleveref}
\usepackage{nameref}
\usepackage{csquotes}

\usepackage{listings}
\usepackage{color}

\definecolor{dkgreen}{rgb}{0,0.6,0}
\definecolor{gray}{rgb}{0.5,0.5,0.5}
\definecolor{mauve}{rgb}{0.58,0,0.82}

\lstset{frame=tb,
  language=XML,
  aboveskip=3mm,
  belowskip=3mm,
  showstringspaces=false,
  columns=flexible,
  basicstyle={\small\ttfamily},
  numbers=none,
  numberstyle=\tiny\color{gray},
  keywordstyle=\color{blue},
  commentstyle=\color{dkgreen},
  stringstyle=\color{mauve},
  breaklines=true,
  breakatwhitespace=true,
  tabsize=3
}

\usepackage[thinc]{esdiff}


\usepackage{booktabs}
\usepackage{mcite}


\graphicspath{{plots/}}
\newcommand*\patchAmsMathEnvironmentForLineno[1]{%
  \expandafter\let\csname old#1\expandafter\endcsname\csname #1\endcsname
  \expandafter\let\csname oldend#1\expandafter\endcsname\csname end#1\endcsname
  \renewenvironment{#1}%
     {\linenomath\csname old#1\endcsname}%
     {\csname oldend#1\endcsname\endlinenomath}}%
\newcommand*\patchBothAmsMathEnvironmentsForLineno[1]{%
  \patchAmsMathEnvironmentForLineno{#1}%
  \patchAmsMathEnvironmentForLineno{#1*}}%
\AtBeginDocument{%
\patchBothAmsMathEnvironmentsForLineno{equation}%
\patchBothAmsMathEnvironmentsForLineno{align}%
\patchBothAmsMathEnvironmentsForLineno{flalign}%
\patchBothAmsMathEnvironmentsForLineno{alignat}%
\patchBothAmsMathEnvironmentsForLineno{gather}%
\patchBothAmsMathEnvironmentsForLineno{multline}%
}

\newcommand{\putat}[3]{\begin{picture}(0,0)(0,0)\put(#1,#2){#3}\end{picture}} % just a shorthand
\parskip 4.2pt           % sets spacing between paragraphs
%\parindend 0pt           % sets spacing between paragraphs
\def\Offline{\mbox{$\overline{\rm
Off}$\hspace{.05em}\raisebox{.4ex}{$\underline{\rm line}$}}\xspace}
\def\OfflineB{\mbox{$\bf\overline{\rm\bf
Off}$\hspace{.05em}\raisebox{.4ex}{$\bf\underline{\rm\bf line}$}}\xspace}

\newcommand{\qpkDist}{$Q^\mathrm{Pk}_\mathrm{Dist}$\,}
\newcommand{\qpkNormDist}{$Q^\mathrm{Pk}_\mathrm{NormDist}$\,}

% equations ...
\newcommand{\be}{\begin{equation}}
\newcommand{\ee}{\end{equation}}
\newcommand{\ben}{\begin{enumerate}}
\newcommand{\een}{\end{enumerate}}
\newcommand{\bi}{\begin{itemize}}
\newcommand{\ei}{\end{itemize}}
\newcommand{\bbe}{\begin{equation*}}
\newcommand{\eee}{\end{equation*}}
\newcommand{\bber}{\begin{equation*}\textcolor{red}}
\newcommand{\eeer}{\end{equation*}}
\newcommand{\bc}{\begin{center}}
\newcommand{\ec}{\end{center}}
\newcommand{\bea}{\begin{eqnarray}}
\newcommand{\eea}{\end{eqnarray}}
\newcommand{\bem}{\begin{pmatrix}}
\newcommand{\eem}{\end{pmatrix}}
\newcommand{\bbea}{\begin{eqnarray*}}
\newcommand{\eeea}{\end{eqnarray*}}
\newcommand{\bcols}{\begin{columns}[T]}
\newcommand{\ecols}{\end{columns}}
\newcommand{\bcol}[1]{\begin{column}{#1\textwidth}}
\newcommand{\ecol}{\end{column}}
%variance
\newcommand{\V}[1]{\mathrm{V}[{#1}]}
% efficiency:
\newcommand{\eps}{\ensuremath\varepsilon}

% escape math mode:
\newcommand{\mr}[1]{\mathrm{#1}}

% bold face:
\newcommand{\mbf}[1]{\mathbf{#1}}

% N_e:
\newcommand{\Ne}{\ensuremath N_\mathrm{e}}
% dEdX:
\newcommand{\dEdX}{\ensuremath\mathrm{d}E/\mathrm{d}X}
% Xmax:
\newcommand{\Xmax}{\ensuremath X_\mathrm{max}}
% meanXmax:
\newcommand{\meanXmax}{\ensuremath \langle X_\mathrm{max}\rangle}
% meanlnA:
\newcommand{\lnA}{\ensuremath \langle \ln \mathrm{A}\rangle}
% reference a float:
\newcommand{\rf}[2]{\mbox{#1 \ref{#2} }}

% microns
%%%%%%%%%%%%%%%%%
% Math Italic A %
%%%%%%%%%%%%%%%%%
\def\re@DeclareMathSymbol#1#2#3#4{%
    \let#1=\undefined
    \DeclareMathSymbol{#1}{#2}{#3}{#4}}

\DeclareSymbolFont{lettersA}{U}{pxmia}{m}{it}
\SetSymbolFont{lettersA}{bold}{U}{pxmia}{bx}{it}
\DeclareFontSubstitution{U}{pxmia}{m}{it}

\DeclareSymbolFontAlphabet{\mathfrak}{lettersA}
\re@DeclareMathSymbol{\muup}{\mathord}{lettersA}{"16}


% microns and microseconds
\newcommand{\mum}{\ensuremath\muup m}
\newcommand{\mus}{\ensuremath\muup s}
\newcommand{\gcm}{g/cm$^2$}

%programs
\newcommand{\GEANE}{\texttt{GEANE}}
\newcommand{\GEANT}{\texttt{GEANT}}
\newcommand{\lcgen}{\texttt{l3cgen}}
\newcommand{\CORSIKA}{\texttt{CORSIKA}}
\newcommand{\VENUS}{\texttt{VENUS}}
\newcommand{\GHEISHA}{\texttt{GEISHA}}

%Calibration
\newcommand{\qpkvem}{$Q^\mathrm{Peak}_\mathrm{VEM}$\,}

%class 1
\newcommand{\cone}{\mbox{class 1}}
\newcommand{\mc}[3]{\multicolumn{#1}{#2}{#3}}
\newcommand{\mcb}[2]{\multicolumn{#1}{c}{#2}}
\newcommand{\ttp}[2]{#1$\cdot$10$^#2$}
\newcommand{\bad}{\textcolor{red}{$\ominus$}}
\newcommand{\ok}{\textcolor{green}{$\oplus$}}
% misc abbreviations

\newcommand{\s}{$\:$}

\newcommand{\ra}{\ensuremath\rightarrow}
\newcommand{\vsp}[1]{\vspace*{#1cm}}
\newcommand{\hsp}[1]{\hspace*{#1cm}}
\def\s1000{S(\unit[1000]{m})}
%%%%%%%%%%%%%%%%%%%%%%%%%%%%%%%%%%%%%%%%%%%%%%%%%%
%
%  some abbreviations from physics.sty
%
%%%%%%%%%%%%%%%%%%%%%%%%%%%%%%%%%%%%%%%%%%%%%%%%%%


\def\EeV{\ifmmode {\mathrm{\ Ee\kern -0.1em V}}\else
                   \textrm{Ee\kern -0.1em V}\fi}%
\def\PeV{\ifmmode {\mathrm{\ Pe\kern -0.1em V}}\else
                   \textrm{Pe\kern -0.1em V}\fi}%
\def\TeV{\ifmmode {\mathrm{\ Te\kern -0.1em V}}\else
                   \textrm{Te\kern -0.1em V}\fi}%
\def\MeV{\ifmmode {\mathrm{\ Me\kern -0.1em V}}\else
                   \textrm{Me\kern -0.1em V}\fi}%
\def\GeV{\ifmmode {\mathrm{\ Ge\kern -0.1em V}}\else
                   \textrm{Ge\kern -0.1em V}\fi}%
\def\keV{\ifmmode {\mathrm{\ ke\kern -0.1em V}}\else
                   \textrm{ke\kern -0.1em V}\fi}%
\def\MeV{\ifmmode {\mathrm{\ Me\kern -0.1em V}}\else
                   \textrm{Me\kern -0.1em V}\fi}%
\def\eV{\ifmmode {\mathrm{\ e\kern -0.1em V}}\else
                   \textrm{e\kern -0.1em V}\fi}%
\def\Zo{\ensuremath{\mathrm {Z}}}
\def\Wp{\ensuremath{\mathrm {W^+}}}
\def\Wm{\ensuremath{\mathrm {W^-}}}
\def\epem{\ensuremath{\mathrm{e^+e^-}}}%
\def\mm{\ensuremath{\mathrm{\mu^+ \mu^-}}}%
\def\antibar#1{\ensuremath{#1\bar{#1}}}%
\def\nbar{\ensuremath{\bar{\nu}}}
\def\nnbar{\antibar{\nu}}%

% offline
\def\Offline{\mbox{$\overline{\rm
Off}$\hspace{.05em}\raisebox{.3ex}{$\underline{\rm line}$}}\xspace}
\def\OfflineB{\mbox{$\bf\overline{\rm\bf
Off}$\hspace{.05em}\raisebox{.2ex}{$\bf\underline{\rm\bf line}$}}\xspace}
\newcommand{\HRule}{\rule{\linewidth}{1mm}}

%%% Local Variables: 
%%% mode: latex
%%% TeX-master: t
%%% End: 



\begin{document}
\setpagewiselinenumbers
\modulolinenumbers[2]

\linenumbers

\renewcommand\linenumberfont{\small\rmfamily}
\begin{flushright}
%GAP-2021-xx
\end{flushright}

\begin{flushright}
  \rule{\linewidth}{0.5mm}
  \\[17mm]
  \vspace*{-3ex}{\Large Accuracy of $Q^\mathrm{Peak}_\mathrm{VEM}$ fit for UB and UUB}
  \large
  \parbox[b]{15cm}
  {
    \begin{flushright}
      Mauricio Su\'arez-Dur\'an and  Ioana~C.~Mari\c{s}
      \\[6mm]
      {\small Universit{\'e} Libre de Bruxelles, Belgium}
    \end{flushright}
  }
  \\[5mm]
  \rule{\linewidth}{0.5mm}
\end{flushright}
%
%\vspace*{-5ex}
%
\thispagestyle{empty}
\noindent

\begin{abstract}
  \noindent
  From the installation of the Upgraded Unified Board (UUB) is
  expected to have a better accuracy for the estimation of the
  \qpkvem values respect of the previous Unified Board (UB).
  Here, the process to calculate the accuracy of the \qpkvem
  values for the UUB and UB is presented, and the results of its
  application on $80$ UUB stations, and their respective UB
  version, are showed.
\end{abstract}

%
%\newpage
%
\thispagestyle{empty}
$\;$
%\listoftodos
%\newpage
\noindent
\clearpage

\section{Fitting histograms}
\label{secFitting}

\begin{figure}[!t]
  \centering
  \subfigure {
    \includegraphics[width=.5\textwidth]{../../plots/chargeHisto863.pdf}
    \includegraphics[width=.5\textwidth]{../../plots/chargeDerHisto863.pdf}
  }
  \caption{Algorithm's steps applied to get the \qpkvem from the
  calibration histograms. Left: typical charge calibration
  histogram, where the black line represents the smoothing
  histogram ($H_S$) after applying 15-bin sliding window. Right:
  first derivative histogram ($H_{DS}$) in black, and smoothing
  of this last one in red ($H_{SDS}$); here, the green vertical
  line shows the maximum or the first approach of \qpkvem.}
  \label{figChargeDerivative}
\end{figure}

The \qpkvem value is obtained from the charge calibration
histogram, an algorithm based on the first derivative has been
implemented and applied. The goal of this procedure is to
estimate the \qpkvem as the maximum of a second order polynomial,
fitting the respective hump muon. In this sense, the first
derivative of the histogram works as first approach to locate
this hump, and from there to estimate the fit range. This
algorithm is described below, and showed in figure
\ref{figChargeDerivative}:

\begin{enumerate}
  \item Smoothing the histogram using a 15-bin sliding window,
    $H_S$.
  \item Obtaining the first derivative of the $H_S$, by
    \begin{equation}
      \frac{f(x+1)-f(x-1)}{2h} \, ,
    \end{equation}
    which is called $H_{DS}$.
  \item Smoothing $H_{DS}$, by 15-bin sliding window, and
    obtaining $H_{SDS}$.
  \item Searching for the estimated \qpkvem, i.e. first bin for
    $H_{SDS}$ equal to zero, from right to left.
  \item Fixing the fitting range using n-bin leftward and n-bin
    rightward from the estimated \qpkvem.
\end{enumerate}
The last algorithm' step requires an extra procedure in order to
fixed the number of n-bin. Using a set of histograms, the
stability of the fit (in terms of RMS/$\left<
Q^\mathrm{Peak}_\mathrm{VEM} \right>$) was plotted as a function
of the n-bin. For this, the estimated \qpkvem from the derivative
(step 4.) was used as initial parameter, and from this point, a
number of n-bin was fixed with an extra condition of not reaching
the valley of the histogram. Finally, the
RMS/$\left<Q^\mathrm{Peak}_\mathrm{VEM}\right>$ vs n-bin was
plotted, and it is presented in the first row of figure
\ref{figApplyingAlgorithm}. There, the stability of the fit
is seen at and after 30-bin, and this last one is chosen as the
number of n-bin for the algorithm' step 5. In the bottom row of
figure \ref{figApplyingAlgorithm}, an example of applying this
procedure is presented. Section \ref{secQpkVsTime} contains the
\qpkvem values obtained by this method from 75 UUB stations
and for the same stations but in UB version (using the same
algorithm).
\clearpage

\textcolor{white}{hi}
\begin{figure}[!t]
  \centering
  \subfigure {
    \includegraphics[width=.5\textwidth]{../../plots/uubChRmsFitBinsLrSt863.pdf}
    \includegraphics[width=.5\textwidth]{../../plots/uubChRmsFitBinsLrSt863zoom.pdf}
  }
  \subfigure{
    \includegraphics[width=.5\textwidth]{../../plots/chargeFitPoly2863.pdf}
    \includegraphics[width=.5\textwidth]{../../plots/chargeFitResidualsPoly2863.pdf}
  }
  \caption{Top row, results for the stability of fitting the hump
  muon (RMS/$\left<Q^\mathrm{Peak}_\mathrm{VEM}\right>$) as
  function of the n-bin. After 30 bins the stability is reached.
  Bottom row, an example of applying the algorithm; here the
  vertical green line shows the hump VEM obtained from the first
  derivative (step 4.), and the vertical blue line shows the
  \qpkvem obtained as the maximum of the fitted second order
  polynomial.}
  \label{figApplyingAlgorithm}
\end{figure}
\newpage

\section{Moving window algorithm}
\label{secMovingWindowAlgo}
\begin{figure}[!t]
  \centering
  \subfigure {
    \includegraphics[width=1.\textwidth]{../../plots2/distDiffAveQpkDayPmts.pdf}
    %\includegraphics[width=0.49\textwidth]{../../plots2/chi2VsSlopPmt1Uub.pdf}
  }
  \caption{Average of number of histograms fitted per day,
  $< N_{Q^{pk}_{VEM}}>$, for UB and UUB, from August
  to November, 2018 and 2021 respectively. In global, the UUB
  version produces $\sim19.5$ T3 triggers per day, around 4 less
  than the same ones for UB.}
  \label{figDistDiffAveQpkDayPmts}
\end{figure}

The strategy followed to calculate the \qpkvem accuracy is based
on the selection of a time window in which the \qpkvem as a
function of time was stable, i.e. \qpkvem$(t)\sim$ constant. In
this sense, as first step the average of number of histograms
fitted per day, $<N_{Q^{pk}_{VEM}}>$, was calculated by PMT for
UB and UUB, and the distribution for the difference of this
averages (UB minus UUB) is presented in figure
\ref{figDistDiffAveQpkDayPmts}. There is possible to see that the
mean of each distribution is positive and around 4 histograms per 
day, which imply that, in average, the UUB version is producing
four T3 triggers less per day than UB; in global, the UUB version
produces $\sim19.5$ charge histograms per day while UB produces
$23.5$ of these same.\\\\The time window has been chosen as six
days, i.e. $5\%$ of total days in four months (August to
November). To identify the stable window into the four months, a
moving-window algorithm was applied for each PMT as follows
\begin{enumerate}
  \item Starting with the first day of August, an average of
    \qpkvem is calculate per day for six days, getting a first
    six-day-series.
  \item A linear fit is applied to previous six-day-series, and
    the respective slope and $\chi^2$ are stored.
  \item After the first [six-day-series]$_0$, a new
    [six-day-series]$_1$ is building replacing the sixth day in
    [six-day-series]$_0$ by the next day in the respective month.
  \item A check in the [six-day-series]$_i$ continuity is
    applied, i.e. checking the six days are consecutive, if not a
    new series is build starting for the first day after the
    discontinuity; for instance, if series $i$ has a
    discontinuity in day 3 jumping to day 5, so the new
    six-day-series is calculated starting from the day 5.
\end{enumerate}
\newpage

\begin{figure}[!t]
  \centering
  \subfigure {
    \includegraphics[width=0.49\textwidth]{../../plots2/chi2VsSlopPmt1Ub.pdf}
    \includegraphics[width=0.49\textwidth]{../../plots2/chi2VsSlopPmt1Uub.pdf}
  }
  \caption{$\chi^2$ Vs Slope. PMT1}
  \label{figChi2VsSlopPmt1}
\end{figure}

\begin{figure}[!t]
  \centering
  \subfigure {
    \includegraphics[width=0.49\textwidth]{../../plots2/chi2VsSlopPmt2Ub.pdf}
    \includegraphics[width=0.49\textwidth]{../../plots2/chi2VsSlopPmt2Uub.pdf}
  }
  \caption{$\chi^2$ Vs Slope. PMT2}
  \label{figChi2VsSlopPmt2}
\end{figure}


\begin{figure}[!t]
  \centering
  \subfigure {
    \includegraphics[width=0.49\textwidth]{../../plots2/chi2VsSlopPmt3Ub.pdf}
    \includegraphics[width=0.49\textwidth]{../../plots2/chi2VsSlopPmt3Uub.pdf}
  }
  \caption{$\chi^2$ Vs Slope. PMT3}
  \label{figChi2VsSlopPmt3}
\end{figure}

The results of this algorithm are shown in figures
\ref{figChi2VsSlopPmt1} to \ref{figChi2VsSlopPmt3}, where the
distribution of the $\chi^2$ versus the slope obtained from each
[six-day-series] is presented per PMT, for both UB and UUB.
There, it is seen that for the three PMTs, the slope distribution
is narrower for UB than UUB, with a clear hot-region seen for UB
PMTs but which it is not observable in UUB. On the other hand,
the mean of the slopes is close to zero for UB, but close to $-1$
for UUB. This means that, in average, the \qpkvem values for UUB
are decreasing at a rate of one FADC per day. Regarding the
quality of the fits, in average the $\chi^2$ is better for UUB
stations, $\left<\chi^2\right>\sim 15$, than UB,
$\left<\chi^2\right>\sim 22$.\\\\Figures
\ref{figChi2VsSlopProjSlopPmt1} to
\ref{figChi2VsSlopProjSlopPmt3} show the projection for the slope
distribution, where is clear that, for the three PMTs, this
distribution is narrow ($\sigma\sim1$), with $\mu\sim0$\,FADC/day
for UB, while for UUB it is wide ($\sigma\sim5$) and
$\mu\sim-1$\,FADC/day. The projections for $\chi^2$ distributions
are presented in figures from \ref{figChi2VsSlopProjChi2Pmt1} to
\ref{figChi2VsSlopProjChi2Pmt3} together with the respective
cumulative distribution, which one has been include with the aim
to set a criteria to reject \qpkvem according with its fit
quality, this due to the extended tail of the distribution and
that its maximum cover less than $30\,\%$ of the counts.
\clearpage

\begin{figure}[!t]
  \centering
  \subfigure {
    \includegraphics[width=0.49\textwidth]{../../plots2/chi2VsSlopPmt1UbProjSlop.pdf}
    \includegraphics[width=0.49\textwidth]{../../plots2/chi2VsSlopPmt1UubProjSlop.pdf}
  }
  \caption{Slope distribution for PMT1, left UB and right UUB.}
  \label{figChi2VsSlopProjSlopPmt1}
\end{figure}

\begin{figure}[!t]
  \centering
  \subfigure {
    \includegraphics[width=0.49\textwidth]{../../plots2/chi2VsSlopPmt2UbProjSlop.pdf}
    \includegraphics[width=0.49\textwidth]{../../plots2/chi2VsSlopPmt2UubProjSlop.pdf}
  }
  \caption{Slope distribution for PMT2, left UB and right UUB.}
  \label{figChi2VsSlopProjSlopPmt2}
\end{figure}

\begin{figure}[!t]
  \centering
  \subfigure {
    \includegraphics[width=0.49\textwidth]{../../plots2/chi2VsSlopPmt3UbProjSlop.pdf}
    \includegraphics[width=0.49\textwidth]{../../plots2/chi2VsSlopPmt3UubProjSlop.pdf}
  }
  \caption{Slope distribution for PMT3, left UB and right UUB.}
  \label{figChi2VsSlopProjSlopPmt3} 
\end{figure}

In this sense, a $\chi^2$ of $18$ for UB, and $12$ for UUB, has
been chosen as cut, values including the $60\,\%$ of the counts,
as it can be seen in the respective plots.

\section{Accuracy calculation}

For the calculation of the accuracy, a set of \qpkvem values, per
PMT, were used. This set corresponds to the [six-day-series] with
the minimum $\chi^2$ among the ones with slope between $-0.5$ and
$0.5$, for both UB and UUB. This range is inside one sigma of the
slope distributions, as figures from
\ref{figChi2VsSlopProjSlopPmt1} to
\ref{figChi2VsSlopProjSlopPmt3}. This imply that some PMTs, from
some stations, were discarded because there was not a
[six-day-series] with a slope into the former range. For
instance, if the PMT$_i$ from the UB station A was discarded, so
the same PMT$_i$ for UUB version was discarded too.
\clearpage

\begin{figure}[!t]
  \centering
  \subfigure {
    \includegraphics[width=0.49\textwidth]{../../plots2/chi2VsSlopPmt1UbProjChi2.pdf}
    \includegraphics[width=0.49\textwidth]{../../plots2/chi2VsSlopPmt1UubProjChi2.pdf}
  }
  \caption{$\chi^2$ distribution PMT1, left UB, right UUB. In
  red, the cumulative of the distribution is presented.}
  \label{figChi2VsSlopProjChi2Pmt1}
\end{figure}

\begin{figure}[!t]
  \centering
  \subfigure {
    \includegraphics[width=0.49\textwidth]{../../plots2/chi2VsSlopPmt2UbProjChi2.pdf}
    \includegraphics[width=0.49\textwidth]{../../plots2/chi2VsSlopPmt2UubProjChi2.pdf}
  }
  \caption{$\chi^2$ distribution for PMT2, left UB, right UUB. In
  red, the cumulative of the distribution is presented.}
  \label{figChi2VsSlopProjChi2Pmt2}
\end{figure}

\begin{figure}[!t]
  \centering
  \subfigure {
    \includegraphics[width=0.49\textwidth]{../../plots2/chi2VsSlopPmt3UbProjChi2.pdf}
    \includegraphics[width=0.49\textwidth]{../../plots2/chi2VsSlopPmt3UubProjChi2.pdf}
  }
  \caption{$\chi^2$ distribution for PMT3, left UB, right UUB. In
  red, the cumulative of the distribution is presented}
  \label{figChi2VsSlopProjChi2Pmt3}
\end{figure}

\noindent Using the [six-day-series] selected, and for the
respective PMT, the \qpkvem were normalized, respect the average
of the set, and a single distribution is built by summing the
distributions of the respective PMTs.\\\\In this way, we obtain a
distribution that represents the \qpkvem normalized for each
station. A Gauss function is fitted to this last distribution in
order to the get the first two moments of the distribution, which
are used to determine the accuracy, $\mu/\sigma$. The figure
\ref{fig1747QpkValues}, top row, shows an example for the
distribution of \qpkvem values (\qpkDist) for station $1208$, UUB
and UB version, meanwhile bottom row shows the distribution for
the \qpkvem normalized.
\clearpage

\begin{figure}[!t]
  \centering
  \subfigure {
    \includegraphics[width=0.49\textwidth]{../../plots/filteredPMTsSt1208.pdf}
    \includegraphics[width=0.49\textwidth]{../../plots/filteredUbPMTsSt1208.pdf}
  }
  \subfigure {
    \includegraphics[width=0.49\textwidth]{../../plots/filteredSt1208.pdf}
    \includegraphics[width=0.49\textwidth]{../../plots/filteredUbSt1208.pdf}
  }
  \caption{Upper row, distribution of the fitted \qpkvem values
  (\qpkDist, in the plot) for station 1747, for each PMT (red
  PMT1, blue PMT2, and green PMT3), UUB (left) and UB (right)
  version. Lower row, For same station, results for \qpkvem
  normalized respect the average of the respective PMTs. The
  magenta line represents the \qpkNormDist, i.e. the sum of
  \qpkDist with a RMS lower or equal to $1.3$ times the $\sigma$
  of the preliminary distribution (see text for details). In red,
  a Gauss function fitted to \qpkNormDist.}
  \label{fig1747QpkValues}
\end{figure}

The figure \ref{figAccuracyResults} shows the results for the
accuracy of the fitted \qpkvem, applying the algorithm presented
in section \ref{secMovingWindowAlgo}. There, the distribution of the
accuracy for UB, and UUB are presented, with a means of
$1.40\,\%\pm0.06\,\%$ and $1.97\,\%\pm0.17\,\%$, respectively.
\clearpage


\begin{figure}[!t]
  \centering
  \subfigure {
    \includegraphics[width=1.\textwidth]{../../plots2/accQpkFitUbUubAllStAllPmt_Stats.pdf}
  }
  \caption{Results for the accuracy calculation of the \qpkvem
  fitting, using the algorithm describe in section 
  \ref{secMovingWindowAlgo}. In the right side, the distribution
  of the $\sigma/\mu$ values is presented for both version, UUB
  (red) and UB (blue). Here, it is possible to see that the
  accuracy for UUB is better than the one for UB.}
  \label{figAccuracyResults}
\end{figure}
\textcolor{white}{hi}
\clearpage


\section{Results hand-selection}
\begin{figure}[!t]
  \centering
  \subfigure {
    \includegraphics[width=1.\textwidth]{../../plots2/accQpkFitUbUubAllStAllPmt_hand.pdf}
  }
  \caption{Results for the accuracy calculation of the \qpkvem
  fitting, selecting the [n-day-series] by hand. In the right
  side, the distribution of the $\sigma/\mu$ values is presented
  for both version, UUB (red) and UB (blue). Here, it is possible
  to see that the accuracy for UUB is better than the one for
  UB.}
  \label{figAccuracyResultsHand}
\end{figure}

The figure \ref{figAccuracyResultsHand} shows the results for the
accuracy of the fitted \qpkvem, selecting the [n-day-series] by
hand. The respective intervals are shown on figures in section
\ref{secQpkVsTime}, horizontal black line. There, the
distribution of the accuracy for UB, and UUB are presented, with
a means of $1.40\,\%\pm0.06\,\%$ and $1.97\,\%\pm0.17\,\%$,
respectively.
\clearpage




\section{\qpkvem as function of time}
\label{secQpkVsTime}

\begin{figure}[!b]
  \centering
  \subfigure {
    \includegraphics[width=0.49\textwidth]{../../plots/qpksVsTimeHandSt545UB.pdf}
    \includegraphics[width=0.49\textwidth]{../../plots/qpksVsTimeHandSt545UUB.pdf}
  }
  \subfigure {
    \includegraphics[width=0.49\textwidth]{../../plots/chi2NdfQpksVsTimeSt545UB.pdf}
    \includegraphics[width=0.49\textwidth]{../../plots/chi2NdfQpksVsTimeSt545UUB.pdf}
  }   
  \subfigure {
    \includegraphics[width=0.49\textwidth]{../../plots/qpksVsTimeSt804UB.pdf}
    \includegraphics[width=0.49\textwidth]{../../plots/qpksVsTimeSt804UUB.pdf}
  }
  \subfigure {
    \includegraphics[width=0.49\textwidth]{../../plots/qpksVsTimeSt806UB.pdf}
    \includegraphics[width=0.49\textwidth]{../../plots/qpksVsTimeSt806UUB.pdf}
  }
  \caption{\qpkvem values obtained applying the algorithm
  presented in section \ref{secFitting}. The first row shows the
  average of the \qpkvem per day plotting as function of time,
  for the station  545, left UB, and right UUB version. In the
  second raw, the quality of the respective fit (the average of
  $\chi^2/$ndf per day), same station, is plotting as function of
  time. The third and fourth row present the same results as row
  1, for stations 804 and 806. if the label ``Selected'' is seen,
  it means that some, or all, PMTs were chosen to calculate the
  accuracy of \qpkvem.}
  %\label{figQpksTime}
\end{figure}
\clearpage

hi
\begin{figure}[!b]
  \centering
  \subfigure {
    \includegraphics[width=0.49\textwidth]{../../plots/qpksVsTimeSt827UB.pdf}
    \includegraphics[width=0.49\textwidth]{../../plots/qpksVsTimeSt827UUB.pdf}
  }
  \subfigure {
    \includegraphics[width=0.49\textwidth]{../../plots/qpksVsTimeHandSt830UB.pdf}
    \includegraphics[width=0.49\textwidth]{../../plots/qpksVsTimeHandSt830UUB.pdf}
  }
  \subfigure {
    \includegraphics[width=0.49\textwidth]{../../plots/qpksVsTimeSt832UB.pdf}
    \includegraphics[width=0.49\textwidth]{../../plots/qpksVsTimeSt832UUB.pdf}
  }
  \subfigure {
    \includegraphics[width=0.49\textwidth]{../../plots/qpksVsTimeSt833UB.pdf}
    \includegraphics[width=0.49\textwidth]{../../plots/qpksVsTimeSt833UUB.pdf}
  }
  \caption{\qpkvem values obtained applying the algorithm
  presented in section \ref{secFitting}. Each row shows the
  average of the \qpkvem per day plotting as function of time,
  left UB, and right UUB version. If the label ``Selected'' is
  seen, it means that some, or all, PMTs were chosen to calculate
  the accuracy of \qpkvem.}
\end{figure}
\clearpage

hi
\begin{figure}[!b]
  \centering
  \subfigure {
    \includegraphics[width=0.49\textwidth]{../../plots/qpksVsTimeHandSt836UB.pdf}
    \includegraphics[width=0.49\textwidth]{../../plots/qpksVsTimeHandSt836UUB.pdf}
  }
  \subfigure {
    \includegraphics[width=0.49\textwidth]{../../plots/qpksVsTimeSt840UB.pdf}
    \includegraphics[width=0.49\textwidth]{../../plots/qpksVsTimeSt840UUB.pdf}
  }
  \subfigure {
    \includegraphics[width=0.49\textwidth]{../../plots/qpksVsTimeSt843UB.pdf}
    \includegraphics[width=0.49\textwidth]{../../plots/qpksVsTimeSt843UUB.pdf}
  }
  \subfigure {
    \includegraphics[width=0.49\textwidth]{../../plots/qpksVsTimeSt846UB.pdf}
    \includegraphics[width=0.49\textwidth]{../../plots/qpksVsTimeSt846UUB.pdf}
  }
  \caption{\qpkvem values obtained applying the algorithm
  presented in section \ref{secFitting}. Each row shows the
  average of the \qpkvem per day plotting as function of time,
  left UB, and right UUB version. If the label ``Selected'' is
  seen, it means that some, or all, PMTs were chosen to calculate
  the accuracy of \qpkvem.}
\end{figure}
\clearpage

hi
\begin{figure}[!b]
  \centering
  \subfigure {
    \includegraphics[width=0.49\textwidth]{../../plots/qpksVsTimeSt849UB.pdf}
    \includegraphics[width=0.49\textwidth]{../../plots/qpksVsTimeSt849UUB.pdf}
  }
  \subfigure {
    \includegraphics[width=0.49\textwidth]{../../plots/qpksVsTimeSt850UB.pdf}
    \includegraphics[width=0.49\textwidth]{../../plots/qpksVsTimeSt850UUB.pdf}
  }
  \subfigure {
    \includegraphics[width=0.49\textwidth]{../../plots/qpksVsTimeSt851UB.pdf}
    \includegraphics[width=0.49\textwidth]{../../plots/qpksVsTimeSt851UUB.pdf}
  }
  \subfigure {
    \includegraphics[width=0.49\textwidth]{../../plots/qpksVsTimeSt853UB.pdf}
    \includegraphics[width=0.49\textwidth]{../../plots/qpksVsTimeSt853UUB.pdf}
  }
  \caption{\qpkvem values obtained applying the algorithm
  presented in section \ref{secFitting}. Each row shows the 
  average of the \qpkvem per day plotting as function of time,
  left UB, and right UUB version. If the label ``Selected'' is
  seen, it means that some, or all, PMTs were chosen to calculate
  the accuracy of \qpkvem.}
\end{figure}
\clearpage

hi
\begin{figure}[!b]
  \centering
  \subfigure {
    \includegraphics[width=0.49\textwidth]{../../plots/qpksVsTimeSt849UB.pdf}
    \includegraphics[width=0.49\textwidth]{../../plots/qpksVsTimeSt849UUB.pdf}
  }
  \subfigure {
    \includegraphics[width=0.49\textwidth]{../../plots/qpksVsTimeSt850UB.pdf}
    \includegraphics[width=0.49\textwidth]{../../plots/qpksVsTimeSt850UUB.pdf}
  }
  \subfigure {
    \includegraphics[width=0.49\textwidth]{../../plots/qpksVsTimeSt851UB.pdf}
    \includegraphics[width=0.49\textwidth]{../../plots/qpksVsTimeSt851UUB.pdf}
  }
  \subfigure {
    \includegraphics[width=0.49\textwidth]{../../plots/qpksVsTimeSt853UB.pdf}
    \includegraphics[width=0.49\textwidth]{../../plots/qpksVsTimeSt853UUB.pdf}
  }
  \caption{\qpkvem values obtained applying the algorithm
  presented in section \ref{secFitting}. Each row shows the 
  average of the \qpkvem per day plotting as function of time,
  left UB, and right UUB version. If the label ``Selected'' is
  seen, it means that some, or all, PMTs were chosen to calculate
  the accuracy of \qpkvem.}
\end{figure}
\clearpage

hi
\begin{figure}[!b]
  \centering
  \subfigure {
    \includegraphics[width=0.49\textwidth]{../../plots/qpksVsTimeSt856UB.pdf}
    \includegraphics[width=0.49\textwidth]{../../plots/qpksVsTimeSt856UUB.pdf}
  }
  \subfigure {
    \includegraphics[width=0.49\textwidth]{../../plots/qpksVsTimeSt859UB.pdf}
    \includegraphics[width=0.49\textwidth]{../../plots/qpksVsTimeSt859UUB.pdf}
  }
  \subfigure {
    \includegraphics[width=0.49\textwidth]{../../plots/qpksVsTimeSt860UB.pdf}
    \includegraphics[width=0.49\textwidth]{../../plots/qpksVsTimeSt860UUB.pdf}
  }
  \subfigure {
    \includegraphics[width=0.49\textwidth]{../../plots/qpksVsTimeSt861UB.pdf}
    \includegraphics[width=0.49\textwidth]{../../plots/qpksVsTimeSt861UUB.pdf}
  }
  \caption{\qpkvem values obtained applying the algorithm
  presented in section \ref{secFitting}. Each row shows the 
  average of the \qpkvem per day plotting as function of time,
  left UB, and right UUB version. If the label ``Selected'' is
  seen, it means that some, or all, PMTs were chosen to calculate
  the accuracy of \qpkvem.}
\end{figure}
\clearpage

hi
\begin{figure}[!b]
  \centering
  \subfigure {
    \includegraphics[width=0.49\textwidth]{../../plots/qpksVsTimeSt862UB.pdf}
    \includegraphics[width=0.49\textwidth]{../../plots/qpksVsTimeSt862UUB.pdf}
  }
  \subfigure {
    \includegraphics[width=0.49\textwidth]{../../plots/qpksVsTimeSt863UB.pdf}
    \includegraphics[width=0.49\textwidth]{../../plots/qpksVsTimeSt863UUB.pdf}
  }
  \subfigure {
    \includegraphics[width=0.49\textwidth]{../../plots/qpksVsTimeSt864UB.pdf}
    \includegraphics[width=0.49\textwidth]{../../plots/qpksVsTimeSt864UUB.pdf}
  }
  \subfigure {
    \includegraphics[width=0.49\textwidth]{../../plots/qpksVsTimeSt866UB.pdf}
    \includegraphics[width=0.49\textwidth]{../../plots/qpksVsTimeSt866UUB.pdf}
  }
  \caption{\qpkvem values obtained applying the algorithm
  presented in section \ref{secFitting}. Each row shows the 
  average of the \qpkvem per day plotting as function of time,
  left UB, and right UUB version. If the label ``Selected'' is
  seen, it means that some, or all, PMTs were chosen to calculate
  the accuracy of \qpkvem.}
\end{figure}
\clearpage

hi
\begin{figure}[!b]
  \centering
  \subfigure {
    \includegraphics[width=0.49\textwidth]{../../plots/qpksVsTimeSt868UB.pdf}
    \includegraphics[width=0.49\textwidth]{../../plots/qpksVsTimeSt868UUB.pdf}
  }
  \subfigure {
    \includegraphics[width=0.49\textwidth]{../../plots/qpksVsTimeHandSt871UB.pdf}
    \includegraphics[width=0.49\textwidth]{../../plots/qpksVsTimeHandSt871UUB.pdf}
  }
  \subfigure {
    \includegraphics[width=0.49\textwidth]{../../plots/qpksVsTimeSt907UB.pdf}
    \includegraphics[width=0.49\textwidth]{../../plots/qpksVsTimeSt907UUB.pdf}
  }
  \subfigure {
    \includegraphics[width=0.49\textwidth]{../../plots/qpksVsTimeSt1185UB.pdf}
    \includegraphics[width=0.49\textwidth]{../../plots/qpksVsTimeSt1185UUB.pdf}
  }
  \caption{\qpkvem values obtained applying the algorithm
  presented in section \ref{secFitting}. Each row shows the 
  average of the \qpkvem per day plotting as function of time,
  left UB, and right UUB version. If the label ``Selected'' is
  seen, it means that some, or all, PMTs were chosen to calculate
  the accuracy of \qpkvem.}
\end{figure}
\clearpage

hi
\begin{figure}[!b]
  \centering
  \subfigure {
    \includegraphics[width=0.49\textwidth]{../../plots/qpksVsTimeSt1190UB.pdf}
    \includegraphics[width=0.49\textwidth]{../../plots/qpksVsTimeSt1190UUB.pdf}
  }
  \subfigure {
    \includegraphics[width=0.49\textwidth]{../../plots/qpksVsTimeSt1191UB.pdf}
    \includegraphics[width=0.49\textwidth]{../../plots/qpksVsTimeSt1191UUB.pdf}
  }
  \subfigure {
    \includegraphics[width=0.49\textwidth]{../../plots/qpksVsTimeSt1198UB.pdf}
    \includegraphics[width=0.49\textwidth]{../../plots/qpksVsTimeSt1198UUB.pdf}
  }
  \subfigure {
    \includegraphics[width=0.49\textwidth]{../../plots/qpksVsTimeSt1205UB.pdf}
    \includegraphics[width=0.49\textwidth]{../../plots/qpksVsTimeSt1205UUB.pdf}
  }
  \caption{\qpkvem values obtained applying the algorithm
  presented in section \ref{secFitting}. Each row shows the 
  average of the \qpkvem per day plotting as function of time,
  left UB, and right UUB version. If the label ``Selected'' is
  seen, it means that some, or all, PMTs were chosen to calculate
  the accuracy of \qpkvem.}
\end{figure}
\clearpage

hi
\begin{figure}[!b]
  \centering
  \subfigure {
    \includegraphics[width=0.49\textwidth]{../../plots/qpksVsTimeSt1207UB.pdf}
    \includegraphics[width=0.49\textwidth]{../../plots/qpksVsTimeSt1207UUB.pdf}
  }
  \subfigure {
    \includegraphics[width=0.49\textwidth]{../../plots/qpksVsTimeHandSt1208UB.pdf}
    \includegraphics[width=0.49\textwidth]{../../plots/qpksVsTimeHandSt1208UUB.pdf}
  }
  \subfigure {
    \includegraphics[width=0.49\textwidth]{../../plots/qpksVsTimeHandSt1209UB.pdf}
    \includegraphics[width=0.49\textwidth]{../../plots/qpksVsTimeHandSt1209UUB.pdf}
  }
  \subfigure {
    \includegraphics[width=0.49\textwidth]{../../plots/qpksVsTimeHandSt1210UB.pdf}
    \includegraphics[width=0.49\textwidth]{../../plots/qpksVsTimeHandSt1210UUB.pdf}
  }
  \caption{\qpkvem values obtained applying the algorithm
  presented in section \ref{secFitting}. Each row shows the 
  average of the \qpkvem per day plotting as function of time,
  left UB, and right UUB version. If the label ``Selected'' is
  seen, it means that some, or all, PMTs were chosen to calculate
  the accuracy of \qpkvem.}
\end{figure}
\clearpage

hi
\begin{figure}[!b]
  \centering
  \subfigure {
    \includegraphics[width=0.49\textwidth]{../../plots/qpksVsTimeSt1211UB.pdf}
    \includegraphics[width=0.49\textwidth]{../../plots/qpksVsTimeSt1211UUB.pdf}
  }
  \subfigure {
    \includegraphics[width=0.49\textwidth]{../../plots/qpksVsTimeSt1213UB.pdf}
    \includegraphics[width=0.49\textwidth]{../../plots/qpksVsTimeSt1213UUB.pdf}
  }
  \subfigure {
    \includegraphics[width=0.49\textwidth]{../../plots/qpksVsTimeSt1214UB.pdf}
    \includegraphics[width=0.49\textwidth]{../../plots/qpksVsTimeSt1214UUB.pdf}
  }
  \subfigure {
    \includegraphics[width=0.49\textwidth]{../../plots/qpksVsTimeHandSt1216UB.pdf}
    \includegraphics[width=0.49\textwidth]{../../plots/qpksVsTimeHandSt1216UUB.pdf}
  }
  \caption{\qpkvem values obtained applying the algorithm
  presented in section \ref{secFitting}. Each row shows the 
  average of the \qpkvem per day plotting as function of time,
  left UB, and right UUB version. If the label ``Selected'' is
  seen, it means that some, or all, PMTs were chosen to calculate
  the accuracy of \qpkvem.}
\end{figure}
\clearpage

hi
\begin{figure}[!b]
  \centering
  \subfigure {
    \includegraphics[width=0.49\textwidth]{../../plots/qpksVsTimeSt1217UB.pdf}
    \includegraphics[width=0.49\textwidth]{../../plots/qpksVsTimeSt1217UUB.pdf}
  }
  \subfigure {
    \includegraphics[width=0.49\textwidth]{../../plots/qpksVsTimeHandSt1218UB.pdf}
    \includegraphics[width=0.49\textwidth]{../../plots/qpksVsTimeHandSt1218UUB.pdf}
  }
  \subfigure {
    \includegraphics[width=0.49\textwidth]{../../plots/qpksVsTimeHandSt1219UB.pdf}
    \includegraphics[width=0.49\textwidth]{../../plots/qpksVsTimeHandSt1219UUB.pdf}
  }
  \subfigure {
    \includegraphics[width=0.49\textwidth]{../../plots/qpksVsTimeHandSt1220UB.pdf}
    \includegraphics[width=0.49\textwidth]{../../plots/qpksVsTimeHandSt1220UUB.pdf}
  }
  \caption{\qpkvem values obtained applying the algorithm
  presented in section \ref{secFitting}. Each row shows the 
  average of the \qpkvem per day plotting as function of time,
  left UB, and right UUB version. If the label ``Selected'' is
  seen, it means that some, or all, PMTs were chosen to calculate
  the accuracy of \qpkvem.}
\end{figure}
\clearpage1

hi
\begin{figure}[!b]
  \centering
  \subfigure {
    \includegraphics[width=0.49\textwidth]{../../plots/qpksVsTimeSt1221UB.pdf}
    \includegraphics[width=0.49\textwidth]{../../plots/qpksVsTimeSt1221UUB.pdf}
  }
  \subfigure {
    \includegraphics[width=0.49\textwidth]{../../plots/qpksVsTimeSt1222UB.pdf}
    \includegraphics[width=0.49\textwidth]{../../plots/qpksVsTimeSt1222UUB.pdf}
  }
  \subfigure {
    \includegraphics[width=0.49\textwidth]{../../plots/qpksVsTimeHandSt1223UB.pdf}
    \includegraphics[width=0.49\textwidth]{../../plots/qpksVsTimeHandSt1223UUB.pdf}
  }
  \subfigure {
    \includegraphics[width=0.49\textwidth]{../../plots/qpksVsTimeSt1224UB.pdf}
    \includegraphics[width=0.49\textwidth]{../../plots/qpksVsTimeSt1224UUB.pdf}
  }
  \caption{\qpkvem values obtained applying the algorithm
  presented in section \ref{secFitting}. Each row shows the 
  average of the \qpkvem per day plotting as function of time,
  left UB, and right UUB version. If the label ``Selected'' is
  seen, it means that some, or all, PMTs were chosen to calculate
  the accuracy of \qpkvem.}
\end{figure}
\clearpage

hi
\begin{figure}[!b]
  \centering
  \subfigure {
    \includegraphics[width=0.49\textwidth]{../../plots/qpksVsTimeSt1225UB.pdf}
    \includegraphics[width=0.49\textwidth]{../../plots/qpksVsTimeSt1225UUB.pdf}
  }
  \subfigure {
    \includegraphics[width=0.49\textwidth]{../../plots/qpksVsTimeSt1227UB.pdf}
    \includegraphics[width=0.49\textwidth]{../../plots/qpksVsTimeSt1227UUB.pdf}
  }
  \subfigure {
    \includegraphics[width=0.49\textwidth]{../../plots/qpksVsTimeSt1729UB.pdf}
    \includegraphics[width=0.49\textwidth]{../../plots/qpksVsTimeSt1729UUB.pdf}
  }
  \subfigure {
    \includegraphics[width=0.49\textwidth]{../../plots/qpksVsTimeSt1733UB.pdf}
    \includegraphics[width=0.49\textwidth]{../../plots/qpksVsTimeSt1733UUB.pdf}
  }
  \caption{\qpkvem values obtained applying the algorithm
  presented in section \ref{secFitting}. Each row shows the 
  average of the \qpkvem per day plotting as function of time,
  left UB, and right UUB version. If the label ``Selected'' is
  seen, it means that some, or all, PMTs were chosen to calculate
  the accuracy of \qpkvem.}
\end{figure}
\clearpage

hi
\begin{figure}[!b]
  \centering
  \subfigure {
    \includegraphics[width=0.49\textwidth]{../../plots/qpksVsTimeSt1735UB.pdf}
    \includegraphics[width=0.49\textwidth]{../../plots/qpksVsTimeSt1735UUB.pdf}
  }
  \subfigure {
    \includegraphics[width=0.49\textwidth]{../../plots/qpksVsTimeSt1736UB.pdf}
    \includegraphics[width=0.49\textwidth]{../../plots/qpksVsTimeSt1736UUB.pdf}
  }
  \subfigure {
    \includegraphics[width=0.49\textwidth]{../../plots/qpksVsTimeSt1737UB.pdf}
    \includegraphics[width=0.49\textwidth]{../../plots/qpksVsTimeSt1737UUB.pdf}
  }
  \subfigure {
    \includegraphics[width=0.49\textwidth]{../../plots/qpksVsTimeSt1738UB.pdf}
    \includegraphics[width=0.49\textwidth]{../../plots/qpksVsTimeSt1738UUB.pdf}
  }
  \caption{\qpkvem values obtained applying the algorithm
  presented in section \ref{secFitting}. Each row shows the 
  average of the \qpkvem per day plotting as function of time,
  left UB, and right UUB version. If the label ``Selected'' is
  seen, it means that some, or all, PMTs were chosen to calculate
  the accuracy of \qpkvem.}
\end{figure}
\clearpage

hi
\begin{figure}[!b]
  \centering
  \subfigure {
    \includegraphics[width=0.49\textwidth]{../../plots/qpksVsTimeHandSt1739UB.pdf}
    \includegraphics[width=0.49\textwidth]{../../plots/qpksVsTimeHandSt1739UUB.pdf}
  }
  \subfigure {
    \includegraphics[width=0.49\textwidth]{../../plots/qpksVsTimeHandSt1740UB.pdf}
    \includegraphics[width=0.49\textwidth]{../../plots/qpksVsTimeHandSt1740UUB.pdf}
  }
  \subfigure {
    \includegraphics[width=0.49\textwidth]{../../plots/qpksVsTimeHandSt1741UB.pdf}
    \includegraphics[width=0.49\textwidth]{../../plots/qpksVsTimeHandSt1741UUB.pdf}
  }
  \subfigure {
    \includegraphics[width=0.49\textwidth]{../../plots/qpksVsTimeSt1742UB.pdf}
    \includegraphics[width=0.49\textwidth]{../../plots/qpksVsTimeSt1742UUB.pdf}
  }
  \caption{\qpkvem values obtained applying the algorithm
  presented in section \ref{secFitting}. Each row shows the 
  average of the \qpkvem per day plotting as function of time,
  left UB, and right UUB version. If the label ``Selected'' is
  seen, it means that some, or all, PMTs were chosen to calculate
  the accuracy of \qpkvem.}
\end{figure}
\clearpage

hi
\begin{figure}[!b]
  \centering
  \subfigure {
    \includegraphics[width=0.49\textwidth]{../../plots/qpksVsTimeSt1743UB.pdf}
    \includegraphics[width=0.49\textwidth]{../../plots/qpksVsTimeSt1743UUB.pdf}
  }
  \subfigure {
    \includegraphics[width=0.49\textwidth]{../../plots/qpksVsTimeSt1744UB.pdf}
    \includegraphics[width=0.49\textwidth]{../../plots/qpksVsTimeSt1744UUB.pdf}
  }
  \subfigure {
    \includegraphics[width=0.49\textwidth]{../../plots/qpksVsTimeHandSt1745UB.pdf}
    \includegraphics[width=0.49\textwidth]{../../plots/qpksVsTimeHandSt1745UUB.pdf}
  }
  \subfigure {
    \includegraphics[width=0.49\textwidth]{../../plots/qpksVsTimeSt1746UB.pdf}
    \includegraphics[width=0.49\textwidth]{../../plots/qpksVsTimeSt1746UUB.pdf}
  }
  \caption{\qpkvem values obtained applying the algorithm
  presented in section \ref{secFitting}. Each row shows the 
  average of the \qpkvem per day plotting as function of time,
  left UB, and right UUB version. If the label ``Selected'' is
  seen, it means that some, or all, PMTs were chosen to calculate
  the accuracy of \qpkvem.}
\end{figure}
\clearpage

hi
\begin{figure}[!b]
  \centering
  \subfigure {
    \includegraphics[width=0.49\textwidth]{../../plots/qpksVsTimeSt1747UB.pdf}
    \includegraphics[width=0.49\textwidth]{../../plots/qpksVsTimeSt1747UUB.pdf}
  }
  \subfigure {
    \includegraphics[width=0.49\textwidth]{../../plots/qpksVsTimeHandSt1779UB.pdf}
    \includegraphics[width=0.49\textwidth]{../../plots/qpksVsTimeHandSt1779UUB.pdf}
  }
  \subfigure {
    \includegraphics[width=0.49\textwidth]{../../plots/qpksVsTimeHandSt1791UB.pdf}
    \includegraphics[width=0.49\textwidth]{../../plots/qpksVsTimeHandSt1791UUB.pdf}
  }
  \subfigure {
    \includegraphics[width=0.49\textwidth]{../../plots/qpksVsTimeSt1798UB.pdf}
    \includegraphics[width=0.49\textwidth]{../../plots/qpksVsTimeSt1798UUB.pdf}
  }
  \caption{\qpkvem values obtained applying the algorithm
  presented in section \ref{secFitting}. Each row shows the 
  average of the \qpkvem per day plotting as function of time,
  left UB, and right UUB version. If the label ``Selected'' is
  seen, it means that some, or all, PMTs were chosen to calculate
  the accuracy of \qpkvem.}
\end{figure}
\clearpage

hi
\begin{figure}[!b]
  \centering
  \subfigure {
    \includegraphics[width=0.49\textwidth]{../../plots/qpksVsTimeSt1817UB.pdf}
    \includegraphics[width=0.49\textwidth]{../../plots/qpksVsTimeSt1817UUB.pdf}
  }
  \subfigure {
    \includegraphics[width=0.49\textwidth]{../../plots/qpksVsTimeSt1818UB.pdf}
    \includegraphics[width=0.49\textwidth]{../../plots/qpksVsTimeSt1818UUB.pdf}
  }
  \subfigure {
    \includegraphics[width=0.49\textwidth]{../../plots/qpksVsTimeHandSt1819UB.pdf}
    \includegraphics[width=0.49\textwidth]{../../plots/qpksVsTimeHandSt1819UUB.pdf}
  }
  \subfigure {
    \includegraphics[width=0.49\textwidth]{../../plots/qpksVsTimeHandSt1851UB.pdf}
    \includegraphics[width=0.49\textwidth]{../../plots/qpksVsTimeHandSt1851UUB.pdf}
  }
  \caption{\qpkvem values obtained applying the algorithm
  presented in section \ref{secFitting}. Each row shows the 
  average of the \qpkvem per day plotting as function of time,
  left UB, and right UUB version. If the label ``Selected'' is
  seen, it means that some, or all, PMTs were chosen to calculate
  the accuracy of \qpkvem.}
\end{figure}
\clearpage

hi
\begin{figure}[!b]
  \centering
  \subfigure {
    \includegraphics[width=0.49\textwidth]{../../plots/qpksVsTimeSt1854UB.pdf}
    \includegraphics[width=0.49\textwidth]{../../plots/qpksVsTimeSt1854UUB.pdf}
  }
  \subfigure {
    \includegraphics[width=0.49\textwidth]{../../plots/qpksVsTimeHandSt1878UB.pdf}
    \includegraphics[width=0.49\textwidth]{../../plots/qpksVsTimeHandSt1878UUB.pdf}
  }
  \subfigure {
    \includegraphics[width=0.49\textwidth]{../../plots/qpksVsTimeHandSt1880UB.pdf}
    \includegraphics[width=0.49\textwidth]{../../plots/qpksVsTimeHandSt1880UUB.pdf}
  }
  \subfigure {
    \includegraphics[width=0.49\textwidth]{../../plots/qpksVsTimeSt1881UB.pdf}
    \includegraphics[width=0.49\textwidth]{../../plots/qpksVsTimeSt1881UUB.pdf}
  }
  \caption{\qpkvem values obtained applying the algorithm
  presented in section \ref{secFitting}. Each row shows the 
  average of the \qpkvem per day plotting as function of time,
  left UB, and right UUB version. If the label ``Selected'' is
  seen, it means that some, or all, PMTs were chosen to calculate
  the accuracy of \qpkvem.}
\end{figure}
\clearpage


\end{document}
