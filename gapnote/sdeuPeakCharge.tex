\documentclass[twoside, final, 10pt]{articleMine}
\usepackage[english]{babel}
\usepackage[sc]{mathpazo}
\usepackage{a4wide}
\usepackage{subfigure}

\usepackage{hyperref}
\usepackage{amsmath,amssymb}
\usepackage[right]{lineno}
\usepackage{xspace}
\usepackage{accents}
\usepackage{graphicx}
\usepackage{booktabs}
\usepackage{color}
\usepackage{units}
\usepackage{enumitem}
\usepackage{todonotes}
\usepackage[capitalize]{cleveref}
\usepackage{nameref}
\usepackage{csquotes}

\usepackage{listings}
\usepackage{color}

\definecolor{dkgreen}{rgb}{0,0.6,0}
\definecolor{gray}{rgb}{0.5,0.5,0.5}
\definecolor{mauve}{rgb}{0.58,0,0.82}

\lstset{frame=tb,
  language=XML,
  aboveskip=3mm,
  belowskip=3mm,
  showstringspaces=false,
  columns=flexible,
  basicstyle={\small\ttfamily},
  numbers=none,
  numberstyle=\tiny\color{gray},
  keywordstyle=\color{blue},
  commentstyle=\color{dkgreen},
  stringstyle=\color{mauve},
  breaklines=true,
  breakatwhitespace=true,
  tabsize=3
}

\usepackage[thinc]{esdiff}


\usepackage{booktabs}
\usepackage{mcite}


\graphicspath{{plots/}}
\newcommand*\patchAmsMathEnvironmentForLineno[1]{%
  \expandafter\let\csname old#1\expandafter\endcsname\csname #1\endcsname
  \expandafter\let\csname oldend#1\expandafter\endcsname\csname end#1\endcsname
  \renewenvironment{#1}%
     {\linenomath\csname old#1\endcsname}%
     {\csname oldend#1\endcsname\endlinenomath}}%
\newcommand*\patchBothAmsMathEnvironmentsForLineno[1]{%
  \patchAmsMathEnvironmentForLineno{#1}%
  \patchAmsMathEnvironmentForLineno{#1*}}%
\AtBeginDocument{%
\patchBothAmsMathEnvironmentsForLineno{equation}%
\patchBothAmsMathEnvironmentsForLineno{align}%
\patchBothAmsMathEnvironmentsForLineno{flalign}%
\patchBothAmsMathEnvironmentsForLineno{alignat}%
\patchBothAmsMathEnvironmentsForLineno{gather}%
\patchBothAmsMathEnvironmentsForLineno{multline}%
}

\newcommand{\putat}[3]{\begin{picture}(0,0)(0,0)\put(#1,#2){#3}\end{picture}} % just a shorthand


\parskip 4.2pt           % sets spacing between paragraphs
%\parindend 0pt           % sets spacing between paragraphs
\def\Offline{\mbox{$\overline{\rm
Off}$\hspace{.05em}\raisebox{.4ex}{$\underline{\rm line}$}}\xspace}
\def\OfflineB{\mbox{$\bf\overline{\rm\bf
Off}$\hspace{.05em}\raisebox{.4ex}{$\bf\underline{\rm\bf line}$}}\xspace}

% equations ...
\newcommand{\be}{\begin{equation}}
\newcommand{\ee}{\end{equation}}
\newcommand{\ben}{\begin{enumerate}}
\newcommand{\een}{\end{enumerate}}
\newcommand{\bi}{\begin{itemize}}
\newcommand{\ei}{\end{itemize}}
\newcommand{\bbe}{\begin{equation*}}
\newcommand{\eee}{\end{equation*}}
\newcommand{\bber}{\begin{equation*}\textcolor{red}}
\newcommand{\eeer}{\end{equation*}}
\newcommand{\bc}{\begin{center}}
\newcommand{\ec}{\end{center}}
\newcommand{\bea}{\begin{eqnarray}}
\newcommand{\eea}{\end{eqnarray}}
\newcommand{\bem}{\begin{pmatrix}}
\newcommand{\eem}{\end{pmatrix}}
\newcommand{\bbea}{\begin{eqnarray*}}
\newcommand{\eeea}{\end{eqnarray*}}
\newcommand{\bcols}{\begin{columns}[T]}
\newcommand{\ecols}{\end{columns}}
\newcommand{\bcol}[1]{\begin{column}{#1\textwidth}}
\newcommand{\ecol}{\end{column}}
%variance
\newcommand{\V}[1]{\mathrm{V}[{#1}]}
% efficiency:
\newcommand{\eps}{\ensuremath\varepsilon}

% escape math mode:
\newcommand{\mr}[1]{\mathrm{#1}}

% bold face:
\newcommand{\mbf}[1]{\mathbf{#1}}

% N_e:
\newcommand{\Ne}{\ensuremath N_\mathrm{e}}
% dEdX:
\newcommand{\dEdX}{\ensuremath\mathrm{d}E/\mathrm{d}X}
% Xmax:
\newcommand{\Xmax}{\ensuremath X_\mathrm{max}}
% meanXmax:
\newcommand{\meanXmax}{\ensuremath \langle X_\mathrm{max}\rangle}
% meanlnA:
\newcommand{\lnA}{\ensuremath \langle \ln \mathrm{A}\rangle}
% reference a float:
\newcommand{\rf}[2]{\mbox{#1 \ref{#2} }}

% microns
%%%%%%%%%%%%%%%%%
% Math Italic A %
%%%%%%%%%%%%%%%%%
\def\re@DeclareMathSymbol#1#2#3#4{%
    \let#1=\undefined
    \DeclareMathSymbol{#1}{#2}{#3}{#4}}

\DeclareSymbolFont{lettersA}{U}{pxmia}{m}{it}
\SetSymbolFont{lettersA}{bold}{U}{pxmia}{bx}{it}
\DeclareFontSubstitution{U}{pxmia}{m}{it}

\DeclareSymbolFontAlphabet{\mathfrak}{lettersA}
\re@DeclareMathSymbol{\muup}{\mathord}{lettersA}{"16}


% microns and microseconds
\newcommand{\mum}{\ensuremath\muup m}
\newcommand{\mus}{\ensuremath\muup s}
\newcommand{\gcm}{g/cm$^2$}

%programs
\newcommand{\GEANE}{\texttt{GEANE}}
\newcommand{\GEANT}{\texttt{GEANT}}
\newcommand{\lcgen}{\texttt{l3cgen}}
\newcommand{\CORSIKA}{\texttt{CORSIKA}}
\newcommand{\VENUS}{\texttt{VENUS}}
\newcommand{\GHEISHA}{\texttt{GEISHA}}

%Calibration
\newcommand{\qpkvem}{$Q^\mathrm{Peak}_\mathrm{VEM}$\,}

%class 1
\newcommand{\cone}{\mbox{class 1}}
\newcommand{\mc}[3]{\multicolumn{#1}{#2}{#3}}
\newcommand{\mcb}[2]{\multicolumn{#1}{c}{#2}}
\newcommand{\ttp}[2]{#1$\cdot$10$^#2$}
\newcommand{\bad}{\textcolor{red}{$\ominus$}}
\newcommand{\ok}{\textcolor{green}{$\oplus$}}
% misc abbreviations

\newcommand{\s}{$\:$}

\newcommand{\ra}{\ensuremath\rightarrow}
\newcommand{\vsp}[1]{\vspace*{#1cm}}
\newcommand{\hsp}[1]{\hspace*{#1cm}}
\def\s1000{S(\unit[1000]{m})}
%%%%%%%%%%%%%%%%%%%%%%%%%%%%%%%%%%%%%%%%%%%%%%%%%%
%
%  some abbreviations from physics.sty
%
%%%%%%%%%%%%%%%%%%%%%%%%%%%%%%%%%%%%%%%%%%%%%%%%%%


\def\EeV{\ifmmode {\mathrm{\ Ee\kern -0.1em V}}\else
                   \textrm{Ee\kern -0.1em V}\fi}%
\def\PeV{\ifmmode {\mathrm{\ Pe\kern -0.1em V}}\else
                   \textrm{Pe\kern -0.1em V}\fi}%
\def\TeV{\ifmmode {\mathrm{\ Te\kern -0.1em V}}\else
                   \textrm{Te\kern -0.1em V}\fi}%
\def\MeV{\ifmmode {\mathrm{\ Me\kern -0.1em V}}\else
                   \textrm{Me\kern -0.1em V}\fi}%
\def\GeV{\ifmmode {\mathrm{\ Ge\kern -0.1em V}}\else
                   \textrm{Ge\kern -0.1em V}\fi}%
\def\keV{\ifmmode {\mathrm{\ ke\kern -0.1em V}}\else
                   \textrm{ke\kern -0.1em V}\fi}%
\def\MeV{\ifmmode {\mathrm{\ Me\kern -0.1em V}}\else
                   \textrm{Me\kern -0.1em V}\fi}%
\def\eV{\ifmmode {\mathrm{\ e\kern -0.1em V}}\else
                   \textrm{e\kern -0.1em V}\fi}%
\def\Zo{\ensuremath{\mathrm {Z}}}
\def\Wp{\ensuremath{\mathrm {W^+}}}
\def\Wm{\ensuremath{\mathrm {W^-}}}
\def\epem{\ensuremath{\mathrm{e^+e^-}}}%
\def\mm{\ensuremath{\mathrm{\mu^+ \mu^-}}}%
\def\antibar#1{\ensuremath{#1\bar{#1}}}%
\def\nbar{\ensuremath{\bar{\nu}}}
\def\nnbar{\antibar{\nu}}%

% offline
\def\Offline{\mbox{$\overline{\rm
Off}$\hspace{.05em}\raisebox{.3ex}{$\underline{\rm line}$}}\xspace}
\def\OfflineB{\mbox{$\bf\overline{\rm\bf
Off}$\hspace{.05em}\raisebox{.2ex}{$\bf\underline{\rm\bf line}$}}\xspace}
\newcommand{\HRule}{\rule{\linewidth}{1mm}}

%%% Local Variables: 
%%% mode: latex
%%% TeX-master: t
%%% End: 




\begin{document}
\setpagewiselinenumbers
\modulolinenumbers[2]

\linenumbers

\renewcommand\linenumberfont{\small\rmfamily}
\begin{flushright}
GAP-2021-xx
\end{flushright}

\begin{flushright}
  \rule{\linewidth}{0.5mm}
  \\[17mm]
  \vspace*{-3ex}{\Large Base Line studies for SDEU}
  \large
  \parbox[b]{15cm}
  {
    \begin{flushright}
      Mauricio Su\'arez-Dur\'an and  Ioana~C.~Mari\c{s}
      \\[6mm]
      {\small Universit{\'e} Libre de Bruxelles, Belgium}
    \end{flushright}
  }
  \\[5mm]
  \rule{\linewidth}{0.5mm}
\end{flushright}
%
%\vspace*{-5ex}
%
\thispagestyle{empty}
\noindent
\begin{abstract}
  \noindent
	Second batch of SDEU	

\end{abstract}

%
%\newpage
%
\thispagestyle{empty}
$\;$
%\listoftodos
%\newpage
\noindent

\section*{Introduction}
  At the end
	\begin{figure}[!tbh]
		\centering
		\subfigure{\includegraphics[width=0.55\textwidth]{figs/mapStations.pdf}}
		\caption{Stations for first pre-production batch including
    UUB, small PMT and SSD.}
		\label{figMapSt}
	\end{figure}
\clearpage

\section*{BXL: Fit method implemented in CDAS}
In order to get a fit method user independent, we use the first
derivative of the histogram (peak and/or charge) to find out the
respective VEM value; we called this method BXL-method.\\\\The
BXL-method consist of 3 steps:
\begin{enumerate}
  \item Smoothing the histogram,
  \item Deriving the smoothed-histogram,
  \item Smoothing the smoothed-derivative-histogram, i.e. the
    histogram getting in previous step.
  \item Searching for the maximum, i.e. VEM-value.
\end{enumerate}
Here, we use the moving average method to smooth the respective
histogram, and central differences to derive them. The needed of
two smoothing is illustrated in figures \ref{figBxlMethodPeak}
and \ref{figBxlMethodCharge}. There, it is possible to see how
the first derivative of the histogram, without smoothing, is
noisy, making difficult to search the respective VEM value. The
same problem is seen for the derivative of the
smoothed-histogram, of course less noisy than the former one, for
this reason we apply another smooth, the third step.\\\\To
evaluate the BXL-method, we compared the VEM value obtained by
this method with the one obtained by fitting a second order
polynomial to the histogram (without smoot). The figures
\ref{figPkFitPoly2comparDer} and \ref{figChFitPoly2comparDer}
show this difference for peak and charge histograms. For the peak
case, the difference is $\sim0.63$\,\% for a VEM value of
$153.03$\,FADC/$8.33$\,ns for fitting (with a $\chi^2$ reduced of
$0.77$), and $154$\,FADC/$8.33$\,ns for BXL-method. Meanwhile for
charge case $\sim0.41$\,\% of difference, with VEM values of
$1249.13$\,FADC for fitting (with a $\chi^2$ reduced of
$1.01$), and $1244.00$\,FADC for BXL-method.\\After verified
for a single histogram (peak and charge, respectively), we
applied the method to all 863-station's peak and charge
histograms, from 1st December 2020, to 31st July, 2021.\\\\For
peak cases, figures \ref{figPkVemCompFitDerTimePmt1} (left and
right), and \ref{figPkVemCompFitDerTimePmt3} show the VEM values
obtained by BXL-method and by fitting a second order polynomial,
for each one of the three PMTs. As it can be seen, the VEM values
getting by BXL-method agree with the ones from the fit method,
differing in average less than $\sim0.6$\,\%. These plots show
that sometimes the BXL-method got a VEM value but fit method does
not. Figures from \ref{figPkSampleFailFitGoodDerpmt12} to
\ref{figPkSampleFailFitDerpmt3} show the histograms for which the
fitting method and/or BXL-method fail. From these results, it is
possible to see that the BXL-method is more effective to find the
VEM value. Nevertheless, figure \ref{figPkSampleFailFitDerpmt3}
show that sometimes the peak histogram has not a peak for the VEM
value, so the BXL-method fails just because there is not a local
maximum.\\\\Figures \ref{figChVemCompFitDerTimePmt1} (left and
right), and \ref{figChVemCompFitDerTimePmt3} show the VEM values
for charge histograms. There, as for peak cases, it can be seen
the agreement between the VEM values getting by BXL-method with
the ones from the fit method; with an average difference of the
order of $\sim0.7$\,\%. In these plots, it is possible to see
some outliers for the fitting method. Figures
\ref{figChSampleFailFitGoodDerpmt12} and
\ref{figChSampleFailFitGoodDerpmt3} present some samples of these
outliers, showing that the fitting method fails due to it does
not set correctly the fitting range. The BXL-method shows to be
more effective than the fitting method to find the VEM value in
charge histograms.
\clearpage

% ==================
% *** The method ***
\begin{figure}[!tbh]
  \centering
  \subfigure
  {
    \includegraphics[width=0.55\textwidth]{../plots/peakHisto863.pdf}
    \includegraphics[width=0.55\textwidth]{../plots/peakDerHisto863.pdf}
  }
  \caption{Fit BXL-method applied to a peak and charge histogram
  (UUB). Left, peak histogram in blue, and the same
  histogram smoothed in red. Right, derivative for peak histogram
  (gray); derivative for smoothed-histogram (red), and smooth of
  smoothed-derivative-histogram (green). It is possible to see
  how the green line allows to identify, very clear, the VEM
  position (dashed line).}
  \label{figBxlMethodPeak}
\end{figure}

\begin{figure}[!tbh]
  \centering
  \subfigure
  {
    \includegraphics[width=0.55\textwidth]{../plots/chargeHisto863.pdf}
    \includegraphics[width=0.55\textwidth]{../plots/chargeDerHisto863.pdf}
  }
  \caption{Fit BXL-method applied to a charge histogram (UUB).
  Left, charge histogram in blue, and the same histogram smoothed
  in red. Right, derivative for charge histogram (gray);
  derivative for smoothed-histogram (red), and smooth of
  smoothed-derivative-histogram (green). It is possible to see
  how the green line allows to identify, very clear, the VEM
  position (dashed line).}
  \label{figBxlMethodCharge}
\end{figure}
\clearpage

% ===================================
% *** Peak: Comparison with Poly2 ***
\begin{figure}[!tbh]
  \centering
  \subfigure
  {
    \includegraphics[width=0.55\textwidth]{../plots/peakFitPoly2863.pdf}
    \includegraphics[width=0.55\textwidth]{../plots/peakFitResidualsPoly2863.pdf}
  }
  \caption{Comparison between the VEM value obtained for a peak
  histogram by fitting a second order polynomial (red line), and
  the same one by BXL-method. The red vertical dashed line
  correspond to VEM value from the polynomial fit
  ($153.03$\,FADC/$8.33$\,ns), and the green one from BXL-method
  ($154.00$\,FADC/$8.33$\,ns).}
  \label{figPkFitPoly2comparDer}
\end{figure}

\begin{figure}[!tbh]
  \centering
  \subfigure
  {
    \includegraphics[width=0.55\textwidth]{../plots/chargeFittedHisto863.pdf}
    \includegraphics[width=0.55\textwidth]{../plots/chargeFitResiduals863.pdf}
  }
  \caption{Comparison between the VEM value obtained for a charge
  histogram by fitting a second order polynomial (red line), and
  the same one by BXL-method. The red vertical dashed line
  correspond to VEM value from the polynomial fit
  ($1249.13$\,FADC), and the green one from BXL-method
  ($1244.00$\,FADC).}
  \label{figChFitPoly2comparDer}
\end{figure}
\clearpage

% ===========================================
% *** Peak: Comparison with Poly2 in time *** 
\begin{figure}[!tbh]
  \centering
  \subfigure
  {
    \includegraphics[width=0.55\textwidth]{../plots/uubPeakFromDerSt863pmt1.pdf}
    \includegraphics[width=0.55\textwidth]{../plots/uubPeakFromDerSt863pmt2.pdf}
  }
  \caption{VEM values for 863-station's peak histograms (VEM
  Peak), from 1st December, 2020 to 31st July, 2021. The blue
  squares corresponds for values obtained by fitting a second
  order polynomial, and red triangles to the values from
  BXL-method. Right, VEM values for PMT1; left, VEM values for
  PMT2. It is possible to see that sometimes BXL-method gets a
  VEM value whereas that fitting method does not, and vice
  versa; see details about this in the text.}
  \label{figPkVemCompFitDerTimePmt1}
\end{figure}

\begin{figure}[!tbh]
  \centering
  \subfigure
  {
    \includegraphics[width=0.55\textwidth]{../plots/uubPeakFromDerSt863pmt3.pdf}
  }
  \caption{VEM values for PMT3 863-station's peak histograms (VEM
  Peak), from 1st December, 2020 to 31st July, 2021. The blue
  squares corresponds for values obtained by fitting a second
  order polynomial, and red triangles to the values from
  BXL-method. It is possible to see that sometimes BXL-method
  gets a VEM value whereas that fitting method does not, and
  vice versa; see details about this in the text.}
  \label{figPkVemCompFitDerTimePmt3}
\end{figure}
\clearpage

% ==================
% *** Fails Peak ***
\begin{figure}[!tbh]
  \centering
  \subfigure
  {
    \includegraphics[width=0.55\textwidth]{../plots/samplePkHistoDerVem249Pmt1.pdf}
    \includegraphics[width=0.55\textwidth]{../plots/samplePkHistoDerVem406Pmt2.pdf}
  }
  \caption{Peak histogram. Failed VEM value for PMT1 (left) and
  PMT2 (right) from fitting a second order polynomial, but a
  succesfull BXL-method, green dashed line.}
  \label{figPkSampleFailFitGoodDerpmt12}
\end{figure}

\begin{figure}[!tbh]
  \centering
  \subfigure
  {
    \includegraphics[width=0.55\textwidth]{../plots/samplePkHistoDerVem160Pmt3.pdf}
  }
  \caption{Peak histogram. Failed VEM value for PMT3 from fitting
  a second order polynomial, but a succesfull BXL-method, green
  dashed line.}
  \label{figPkSampleFailFitGoodDerpmt3}
\end{figure}

\begin{figure}[!tbh]
  \centering
  \subfigure
  {
    \includegraphics[width=0.55\textwidth]{../plots/samplePkHistoDerVem163Pmt3.pdf}
    \includegraphics[width=0.55\textwidth]{../plots/peakDerHisto863Evt62175266.pdf}
  }
  \caption{Peak histogram. Failed VEM value for PMT3 from fitting
  a second order polynomial, but a succesfull BXL-method, green
  dashed line.}
  \label{figPkSampleFailFitDerpmt3}
\end{figure}
\clearpage

% =====================================
% *** Charge: Comparison with Poly2 ***

\begin{figure}[!tbh]
  \centering
  \subfigure
  {
    \includegraphics[width=0.55\textwidth]{../plots/uubChargeFromDerSt863pmt1.pdf}
    \includegraphics[width=0.55\textwidth]{../plots/uubChargeFromDerSt863pmt2.pdf}
  }
  \caption{VEM values for 863-station's charge histograms (VEM
  Charge), from 1st December, 2020 to 31st July, 2021. The blue
  squares corresponds for values obtained by fitting a second
  order polynomial, and red triangles to the values from
  BXL-method. Right, VEM values for PMT1; left, VEM values for
  PMT2. It can be seen some outliers, both PMTs, for the fitting
  method. See details in the text.}
  \label{figChVemCompFitDerTimePmt1}
\end{figure}

\begin{figure}[!tbh]
  \centering
  \subfigure
  {
    \includegraphics[width=0.55\textwidth]{../plots/uubChargeFromDerSt863pmt3.pdf}
  }
  \caption{VEM values for 863-station's charge histograms (VEM
  Charge), PMT3 from 1st December, 2020 to 31st July, 2021. The
  blue squares corresponds for values obtained by fitting a
  second order polynomial, and red triangles to the values from
  BXL-method. It can be seen some outliers for fitting method.
  See details in the text.} 
  \label{figChVemCompFitDerTimePmt3}
\end{figure}
\clearpage

% ==================
% *** Fails Peak ***
\begin{figure}[!tbh]
  \centering
  \subfigure
  {
    \includegraphics[width=0.55\textwidth]{../plots/sampleChHistoDerVem365Pmt1.pdf}
    \includegraphics[width=0.55\textwidth]{../plots/sampleChHistoDerVem251Pmt2.pdf}
  }
  \caption{Charge histogram. Failed VEM value for PMT1 (left) and
  PMT2 (right) from fitting a second order polynomial, but a
  succesful BXL-method, green dashed line. For both PMTs, the
  fitting method fails in finding the proper interval.}
  \label{figChSampleFailFitGoodDerpmt12}
\end{figure}

\begin{figure}[!tbh]
  \centering
  \subfigure
  {
    \includegraphics[width=0.55\textwidth]{../plots/sampleChHistoDerVem237Pmt3.pdf}
  }
  \caption{Charge histogram. Failed VEM value for PMT3 from
  fitting a second order polynomial, but a succesfull BXL-method,
  green dashed line. The fitting method fails in finding the
  proper interval.}
  \label{figChSampleFailFitGoodDerpmt3}
\end{figure}
\clearpage


\subsection*{OffLine Method}
The OffLine algorithm to search for the VEM value (for peak and
charge histograms) is implemented in the SdCalibrator module,
using the {\it ad} files.\\\\This algorithm will be applied if
the maximum of counts in the histogram is bigger than 500 (a
number setting by default SdCarlibrator class), and ignoring the
last 5 bins. As first step to fit the histogram the fitting range
is setting from the shoulderLow to shoulderHigh. Where the
shoulderLow is established as the first bin with the maximum
number of counts, starting from the start bin, i.e. the start bin
is defined as the one with more than 40 counts (a number setting
by default), from this bin, the algorithm looking for the first
bin with a number of counts such that the next three bins have a
number of counts lower than the $75$\,\% of counts of this first
bin; this last percentage can be setting by the user.
\clearpage

\section*{Peak Histograms}
Comparison between OffLine SdCalibrator (ad files) and our method
for CDAS files. For peak, chi2/Ndof $<3.5$.

\begin{figure}[!tbh]
  \centering
  \subfigure
  {
    \includegraphics[width=0.55\textwidth]{../plots/offlinePkCompaSt863Pmt1.pdf}
    \includegraphics[width=0.55\textwidth]{../plots/offlineResidPkSt863Pmt1.pdf}
  }
  \caption{Peak histogram. Right, Comparison for VEM value using
  OffLine (blue line), second order polynomial fitting (red
  line), and BXL-method (green dashed line). Left, residuals for
  OffLine (blue points) and second order polynomial fit (red
  points). As it can be seen, the polynomial and BXL-method set
  properly the VEM value, but OffLine fit is not correct.}
  \label{figPkCompaOffFitDer}
\end{figure}

\begin{figure}[!tbh]
  \centering
  \subfigure
  {
    \includegraphics[width=0.55\textwidth]{../plots/offlineVEMpkSt863Pmt1.pdf}
  }
  \caption{Comparison for VEM peak values obtained by OffLine and
  BXL-method, from 1st of December, 2020 to 31st of July, 2021.
  Here, the OffLine algorithm fails to fit some histograms (blue
  dots at zero counts) but BXL-method is successful for all
  histograms.}
  \label{figPkCompVemOffBxl}
\end{figure}
\clearpage

\begin{figure}[!tbh]
  \centering
  \subfigure
  {
    \includegraphics[width=0.55\textwidth]{../plots/offlineChi2pkSt863Pmt1.pdf}
    \includegraphics[width=0.55\textwidth]{../plots/cdasChi2pkSt863Pmt1.pdf}
  }
  \caption{Stations for first pre-production batch including UUB, small PMT and SSD.}
  \label{figPkNdfserie}
\end{figure}

\begin{figure}[!tbh]
  \centering
  \subfigure
  {
    \includegraphics[width=0.55\textwidth]{../plots/offlineNDFpkSt863Pmt1.pdf}
    \includegraphics[width=0.55\textwidth]{../plots/cdasNDFpkSt863Pmt1.pdf}
  }
  \caption{Stations for first pre-production batch including UUB, small PMT and SSD.}
  \label{figPkNdfserie}
\end{figure}


\begin{figure}[!tbh]
  \centering
  \subfigure
  {
    \includegraphics[width=0.55\textwidth]{../plots/offlineChi2NdfPkSt863Pmt1.pdf}
    \includegraphics[width=0.55\textwidth]{../plots/cdasChi2NdfPkSt863Pmt1.pdf}
  }
  \caption{Stations for first pre-production batch including UUB, small PMT and SSD.}
  \label{figPkChi2Ndfserie}
\end{figure}
\clearpage

\begin{figure}[!tbh]
  \centering
  \subfigure
  {
    \includegraphics[width=0.55\textwidth]{../plots/offlineLowHihgpkSt863Pmt1.pdf}
    \includegraphics[width=0.55\textwidth]{../plots/cdasLowHighpkSt863Pmt1.pdf}
  }
  \caption{Stations for first pre-production batch including UUB, small PMT and SSD.}
  \label{figPkLowHigh}
\end{figure}


\begin{figure}[!tbh]
  \centering
  \subfigure
  {
    \includegraphics[width=0.55\textwidth]{../plots/offlineFailedPeakSt863PMT1Evt61411290.pdf}
  }
  \caption{Failed fit Peak histogram for OffLine; event Id. 61411290, Dec 25, 2020.}
  \label{figPkFailsfit}
\end{figure}
\clearpage

\section*{Charge Histograms}
Comparison between OffLine SdCalibrator (ad files) and our method
for CDAS files. For charge, chi2/Ndof $<5.5$.
\clearpage

\begin{figure}[!tbh]
  \centering
  \subfigure
  {
    \includegraphics[width=0.55\textwidth]{../plots/offlineChCompaSt863Pmt1.pdf}
    \includegraphics[width=0.55\textwidth]{../plots/offlineResidChSt863Pmt1.pdf}
  }
  \caption{Charge histogram. Right, Comparison for VEM value
  using OffLine (blue line), second order polynomial fitting (red
  line), and BXL-method (green dashed line). Left, residuals for
  OffLine (blue points) and second order polynomial fit (red
  points). As it can be seen, the polynomial and BXL-method set
  properly the VEM value, but OffLine fit is not correct.}
  \label{figChCompaOffFitDer}
\end{figure}

\begin{figure}[!tbh]
  \centering
  \subfigure
  {
    \includegraphics[width=0.55\textwidth]{../plots/offlineVEMchSt863Pmt1.pdf}
  }
  \caption{Comparison for VEM charge values obtained by OffLine
  and BXL-method, from 1st of December, 2020 to 31st of July,
  2021. Here, the OffLine algorithm fails to fit some histograms
  (blue dots at zero counts) but BXL-method is successful for all
  histograms.}
  \label{figChCompVemOffBxl}
\end{figure}
\clearpage

\begin{figure}[!tbh]
  \centering
  \subfigure
  {
    \includegraphics[width=0.55\textwidth]{../plots/offlineChi2chSt863Pmt1.pdf}
    \includegraphics[width=0.55\textwidth]{../plots/cdasChi2chSt863Pmt1.pdf}
  }
  \caption{Stations for first pre-production batch including UUB, small PMT and SSD.}
  \label{figChChi2serie}
\end{figure}

\begin{figure}[!tbh]
  \centering
  \subfigure
  {
    \includegraphics[width=0.55\textwidth]{../plots/offlineNDFchSt863Pmt1.pdf}
    \includegraphics[width=0.55\textwidth]{../plots/cdasNDFchSt863Pmt1.pdf}
  }
  \caption{Stations for first pre-production batch including UUB, small PMT and SSD.}
  \label{figChNdfserie}
\end{figure}

\begin{figure}[!tbh]
  \centering
  \subfigure
  {
    \includegraphics[width=0.55\textwidth]{../plots/offlineChi2NdfChSt863Pmt1.pdf}
    \includegraphics[width=0.55\textwidth]{../plots/cdasChi2NdfChSt863Pmt1.pdf}
  }
  \caption{Stations for first pre-production batch including UUB, small PMT and SSD.}
  \label{figChChi2Ndfserie}
\end{figure}
\clearpage


\begin{figure}[!tbh]
  \centering
  \subfigure
  {
    \includegraphics[width=0.55\textwidth]{../plots/offlineLowHihgchSt863Pmt1.pdf}
    \includegraphics[width=0.55\textwidth]{../plots/cdasLowHighchSt863Pmt1.pdf}
  }
  \caption{Stations for first pre-production batch including UUB, small PMT and SSD.}
  \label{figChLowHigh}
\end{figure}

\begin{figure}[!tbh]
  \centering
  \subfigure
  {
    \includegraphics[width=0.55\textwidth]{../plots/offlineFailedChargeSt863PMT1Evt61269719.pdf}
    \includegraphics[width=0.55\textwidth]{../plots/offlineFailedChResidSt863PMT1Evt61269719.pdf}
  }
  \caption{OffLine failed fit for Charge histogram, 863 station, event 61269719, pmt1.
  In red, the fit obtenined in CDAS.}
  \label{figChLowHigh}
\end{figure}
\clearpage

\section*{Comparison for 863 and 1740 stations}

{\bf 863 Station Peak}
\begin{figure}[!tbh]
  \centering
  \subfigure
  {
    \includegraphics[width=0.55\textwidth]{../plots/uubPeakFromDerSt863pmt1.pdf}
    \includegraphics[width=0.55\textwidth]{../plots/uubPeakFromOffSt863pmt1.pdf}
  }
  \caption{Stations for first pre-production batch including UUB, small PMT and SSD.}
  \label{figChLowHigh}
\end{figure}

\begin{figure}[!tbh]
  \centering
  \subfigure
  {
    \includegraphics[width=0.55\textwidth]{../plots/uubPeakFromDerSt863pmt2.pdf}
    \includegraphics[width=0.55\textwidth]{../plots/uubPeakFromOffSt863pmt2.pdf}
  }
  \caption{Stations for first pre-production batch including UUB, small PMT and SSD.}
  \label{figChLowHigh}
\end{figure}

\begin{figure}[!tbh]
  \centering
  \subfigure
  {
    \includegraphics[width=0.55\textwidth]{../plots/uubPeakFromDerSt863pmt3.pdf}
    \includegraphics[width=0.55\textwidth]{../plots/uubPeakFromOffSt863pmt3.pdf}
  }
  \caption{Stations for first pre-production batch including UUB, small PMT and SSD.}
  \label{figChLowHigh}
\end{figure}
\clearpage

{\bf 863 Station Charge}

\begin{figure}[!tbh]
  \centering
  \subfigure
  {
    \includegraphics[width=0.55\textwidth]{../plots/uubChargeFromDerSt863pmt1.pdf}
    \includegraphics[width=0.55\textwidth]{../plots/uubChargeFromOffSt863pmt1.pdf}
  }
  \caption{Stations for first pre-production batch including UUB, small PMT and SSD.}
  \label{figChLowHigh}
\end{figure}

\begin{figure}[!tbh]
  \centering
  \subfigure
  {
    \includegraphics[width=0.55\textwidth]{../plots/uubChargeFromDerSt863pmt2.pdf}
    \includegraphics[width=0.55\textwidth]{../plots/uubChargeFromOffSt863pmt2.pdf}
  }
  \caption{Stations for first pre-production batch including UUB, small PMT and SSD.}
  \label{figChLowHigh}
\end{figure}

\begin{figure}[!tbh]
  \centering
  \subfigure
  {
    \includegraphics[width=0.55\textwidth]{../plots/uubChargeFromDerSt863pmt3.pdf}
    \includegraphics[width=0.55\textwidth]{../plots/uubChargeFromOffSt863pmt3.pdf}
  }
  \caption{Stations for first pre-production batch including UUB, small PMT and SSD.}
  \label{figChLowHigh}
\end{figure}
\clearpage

{\bf 1740 Station Peak}
\begin{figure}[!tbh]
  \centering
  \subfigure
  {
    \includegraphics[width=0.55\textwidth]{../plots/uubPeakFromDerSt1740pmt1.pdf}
    \includegraphics[width=0.55\textwidth]{../plots/uubPeakFromOffSt1740pmt1.pdf}
  }
  \caption{Stations for first pre-production batch including UUB,
  small PMT and SSD.}
  \label{figChLowHigh}
\end{figure}

\begin{figure}[!tbh]
  \centering
  \subfigure
  {
    \includegraphics[width=0.55\textwidth]{../plots/uubPeakFromDerSt1740pmt2.pdf}
    \includegraphics[width=0.55\textwidth]{../plots/uubPeakFromOffSt1740pmt2.pdf}
  }
  \caption{Stations for first pre-production batch including UUB, small PMT and SSD.}
  \label{figChLowHigh}
\end{figure}

\begin{figure}[!tbh]
  \centering
  \subfigure
  {
    \includegraphics[width=0.55\textwidth]{../plots/uubPeakFromDerSt1740pmt3.pdf}
    \includegraphics[width=0.55\textwidth]{../plots/uubPeakFromOffSt1740pmt3.pdf}
  }
  \caption{Stations for first pre-production batch including UUB, small PMT and SSD.}
  \label{figChLowHigh}
\end{figure}
\clearpage


{\bf 1740 Station Charge}
\begin{figure}[!tbh]
  \centering
  \subfigure
  {
    \includegraphics[width=0.55\textwidth]{../plots/uubChargeFromDerSt1740pmt1.pdf}
    \includegraphics[width=0.55\textwidth]{../plots/uubChargeFromOffSt1740pmt1.pdf}
  }
  \caption{Stations for first pre-production batch including UUB, small PMT and SSD.}
  \label{figChLowHigh}
\end{figure}

\begin{figure}[!tbh]
  \centering
  \subfigure
  {
    \includegraphics[width=0.55\textwidth]{../plots/uubChargeFromDerSt1740pmt2.pdf}
    \includegraphics[width=0.55\textwidth]{../plots/uubChargeFromOffSt1740pmt2.pdf}
  }
  \caption{Stations for first pre-production batch including UUB, small PMT and SSD.}
  \label{figChLowHigh}
\end{figure}

\begin{figure}[!tbh]
  \centering
  \subfigure
  {
    \includegraphics[width=0.55\textwidth]{../plots/uubChargeFromDerSt1740pmt3.pdf}
    \includegraphics[width=0.55\textwidth]{../plots/uubChargeFromOffSt1740pmt3.pdf}
  }
  \caption{Stations for first pre-production batch including UUB, small PMT and SSD.}
  \label{figChLowHigh}
\end{figure}
\clearpage

\end{document}
