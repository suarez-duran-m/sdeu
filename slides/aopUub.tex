\documentclass[aspectratio=169]{beamer}
\usepackage[T1]{fontenc}
\usepackage[utf8]{inputenc}
\usepackage{tikz}
\usepackage{tabularx}
\usepackage[font=scriptsize]{caption}
\captionsetup[figure]{labelformat=empty}

\usetikzlibrary{tikzmark,shapes,arrows,backgrounds,fit,positioning}
\newcolumntype{C}{>{\centering\arraybackslash}X}
\newcolumntype{R}{>{\raggedleft\arraybackslash}X}

\addtobeamertemplate{navigation symbols}{}
{
	\insertframenumber{}
}

\beamertemplatenavigationsymbolsempty
\setbeamercolor{section in foot}{fg=white, bg=blue}
\setbeamercolor{subsection in foot}{fg=black, bg=white}
\setbeamerfont{footline}{size=\fontsize{6}{6}\selectfont}

\setbeamertemplate{footline}
{
  \leavevmode
  \hbox
  {
    \begin{beamercolorbox}
      [wd=.5\paperwidth,ht=2.5ex,dp=1.125ex,leftskip=.3cm,rightskip=.3cm]{subsection in foot}
    \end{beamercolorbox}
    
    \begin{beamercolorbox}
      [wd=.5\paperwidth,ht=2.5ex,dp=1.125ex,leftskip=.3cm,rightskip=.3cm plus1fil]{subsection in foot}
      \hfill
      \insertframenumber
      %\insertframenumber\,/\,\inserttotalframenumber
    \end{beamercolorbox}
  }
}



\title{UUB Charge and Peak histograms}
\author{
  Mauricio Su\'arez Dur\'an and Ioana~C.~Mari\c{s}
}
\institute{IIHE-ULB}

\titlegraphic{
  \begin{figure}[h]
    \centering
   %\includegraphics[width=5cm]{ulbLogo2.png}
    \hspace*{8.cm}
    \includegraphics[width=5.5cm]{iihe.jpeg}
  \end{figure}
}

\begin{document}
\begin{frame}
  \titlepage
\end{frame}


\begin{frame}
	\frametitle{UUB Charge and Peak histograms}
	\begin{itemize}
		\item Station studied: 863 1222 1219 1211 1740 1743 1221 1223 1217 1747 1741 1745 1818 1851 1729 1735 1746 1819 1791
		\item Data from CDAS.
		\item {\underline {Software CDAS, pre-production version.}}
	\end{itemize}
	\centering
	\includegraphics[width=.45\textwidth]{mapStations.pdf}
\end{frame}

% ========================
% *** First Bin counts ***

%\begin{frame}
%  \frametitle{UUB and UB counts in the first bin in Raw Peak Histogram}
%	Raw UUB Peak Histogram
%	\begin{figure}
%		\centering
%		\begin{tabularx}{\textwidth}{CC}
%			\begin{tabular}{l}
%				\includegraphics[width=.5\textwidth]{../plots/uubRawPeakPMT1St863.pdf}
%			\end{tabular}
%			&
%			\begin{tabular}{l}
%				\includegraphics[width=.5\textwidth]{../plots/uubRawPeakZoomPMT1St863.pdf}
%			\end{tabular}
%			\\
%			& Not zero counts in the first bin
%		\end{tabularx}
%	\end{figure}
%\end{frame}

\begin{frame}
  \frametitle{UUB Raw Peak histograms: noise at first bin? }
  {\bf IoSdHisto::Peak[pmtId][0]}
	\begin{figure}
		\centering
    \includegraphics[width=.4\textwidth]{../plots/uubRawPeakPMT1St863.pdf}
		\begin{tabularx}{\textwidth}{CC}
			\begin{tabular}{l}
				%\includegraphics[width=.44\textwidth]{../plots/uubAllAveFrstBinPMTsBars.png}
				\includegraphics[width=.44\textwidth]{../plots/uubAllAveFrstBinPMTs.png}
			\end{tabular}
			&
			\begin{tabular}{l}
        %\includegraphics[width=.44\textwidth]{../plots/ubAllAveFrstBinPMTsBarsFull.png}
				\includegraphics[width=.44\textwidth]{../plots/ubAllAveFrstBinPMTs.png}
			\end{tabular} 
		\end{tabularx}
	\end{figure}
\end{frame}

\begin{frame}
	\frametitle{From UUB raw Peak histogram to correct format}
	{\bf Applying IoSdStation::HPeak method}
	\begin{figure}
		\centering
		\begin{tabularx}{\textwidth}{CC}
      \multicolumn{2}{l}{xp[j] = j*mult + {\bf offset};
      xp[100 + j] = 100 * mult + bigbins * j * mult + {\bf offset}} 
			\\ 
      \begin{tabular}{l}
				\includegraphics[width=.45\textwidth]{../plots/uubPeakPMT1St863.pdf}
			\end{tabular}
      &
      \begin{tabular}{l}
        \includegraphics[width=.38\textwidth]{../plots/uubAllAveOffsetPMTs.png}
        \\
        \includegraphics[width=.38\textwidth]{../plots/ubAllAveOffsetPMTs.png}
      \end{tabular}
		\end{tabularx}
	\end{figure}
\end{frame}

% =======================
% *** For Peak Offset ***

%\begin{frame}
%	{\bf IoSdHisto::Peak[pmtId][0]}
%	\begin{figure}
%		\centering
%		\begin{tabularx}{\textwidth}{CC}
%			\begin{tabular}{l}
%				\includegraphics[width=.48\textwidth]{../plots/uubAllAveOffsetPMTs.png}
%			\end{tabular}
%      &
%      \begin{tabular}{r}
%				\includegraphics[width=.48\textwidth]{../plots/ubAllAveOffsetPMTs.png}
%			\end{tabular}
%		\end{tabularx}
%	\end{figure}
%\end{frame}


% ====================
% *** For Baseline ***

%\begin{frame}
%	\frametitle{Checking UUB Baseline: IoSdStation::HBase[pmt] and Calib.Base[pmt]}
%	\begin{figure}
%		\centering
%		\begin{tabularx}{\textwidth}{CC}
%			\includegraphics[width=.4\textwidth]{../plots/uubAllAveHbasePMTs.png}
%			&
%      \includegraphics[width=.4\textwidth]{../plots/uubAllAveCalibPMTs.png}
%			\\
%			\includegraphics[width=.4\textwidth]{../plots/ubAllAveHbasePMTs.png}
%			&
%			\includegraphics[width=.4\textwidth]{../plots/ubAllAveCalibPMTs.png}
%		\end{tabularx}
%	\end{figure}
%\end{frame}

% ===============================
% *** Correction for Baseline ***

\begin{frame}
	\frametitle{UUB Peak histogram correcting baseline for: HBase and Calib.Base}
	\begin{figure}
		\centering
		\begin{tabularx}{\textwidth}{CC}
			\includegraphics[width=.38\textwidth]{../plots/uubPeakCoorHBasePMT1St863.pdf}
			&
			\includegraphics[width=.38\textwidth]{../plots/uubPeakCoorCalibBasePMT1St863.pdf}
			\\
			\includegraphics[width=.38\textwidth]{../plots/uubPeakCoorHBaseZoomPMT1St863.pdf}
			&
			\includegraphics[width=.38\textwidth]{../plots/uubPeakCoorCalibBaseZoomPMT1St863.pdf}
			\\
			\multicolumn{2}{l}{Which one should we use to correct?}
		\end{tabularx}
	\end{figure}
\end{frame}
%
%% *** For first bin center ***
%
%\begin{frame}
%  \frametitle{Comparison PMT1: UUB and UB Peak histogram First Bin center}
%  Here, First Bin center: GetBinCenter(1).
%  \begin{figure}
%    \centering
%    \begin{tabularx}{\textwidth}{CC}
%			\includegraphics[width=.45\textwidth]{../plots/uubAllAveBinCntrHbasePMTs.png}
%			&
%			\includegraphics[width=.45\textwidth]{../plots/uubAllAveBinCntrCalibPMTs.png}
%      \\
%			\includegraphics[width=.45\textwidth]{../plots/ubAllAveBinCntrHbasePMTs.png}
%			&
%			\includegraphics[width=.45\textwidth]{../plots/ubAllAveBinCntrCalibPMTs.png}
%		\end{tabularx}
%	\end{figure}
%\end{frame}



% ===========================
% *** Applying for Charge ***

%\begin{frame}
%	\frametitle{Applying the previous steps to UUB Charge histograms}
%	{\bf Raw UUB Charge histogram}
%	\begin{figure}
%		\begin{tabularx}{\textwidth}{C}
%			\includegraphics[width=.7\textwidth]{../plots/uubRawChargePMT1St863.pdf}
%		\end{tabularx}
%	\end{figure}
%\end{frame}


\begin{frame}
	\frametitle{Applying the previous steps to UUB Charge histograms}
	{IoSdStation::HCharge}
	\begin{figure}
		\centering
		\begin{tabularx}{\textwidth}{CC}
      \multicolumn{2}{l}{xc[j] = mult*j + {\bf offset}; xc[400+j] = 400*mult + bigbins*mult*j + {\bf offset} }
			\\
			\begin{tabular}{l}
				\includegraphics[width=.38\textwidth]{../plots/uubChargePMT1St863.pdf}
			\end{tabular}
      &
      \begin{tabular}{l}
        \includegraphics[width=.4\textwidth]{../plots/uubAllAveOffsetChPMTs.png}
      \end{tabular}
      \\
      \begin{tabular}{l}
        \includegraphics[width=.4\textwidth]{../plots/uubAllAveOffsetChPMT3s.png}
      \end{tabular}
      &
      \begin{tabular}{l}
        \includegraphics[width=.4\textwidth]{../plots/ubAllAveOffsetChPMTs.png}
      \end{tabular}
		\end{tabularx}
	\end{figure}
\end{frame}

\begin{frame}
	\frametitle{Area/Peak}
	Fitting Histograms using:
	\begin{displaymath}
		e^{\left( a_0-\frac{1}{x\tau}\right) } + x^{-1}e^{ -\left(\frac{(\ln x - \ln\mu)^2}{2\sigma^2}\right) }
	\end{displaymath}
	
	\begin{figure}
		\begin{tabularx}{\textwidth}{CCC}
			\includegraphics[width=.35\textwidth]{../plots/ubChHistFitSt1223.pdf}
			\caption{Charge: UB-PMT1}
			&
			\includegraphics[width=.35\textwidth]{../plots/uubChHistFitSt1223.pdf}
			\caption{Charge: UUB-PMT1}
			&
			\includegraphics[width=.35\textwidth]{../plots/ubPkHistFitSt1222.pdf}
			\caption{Peak: UUB-PMT1}
		\end{tabularx}
	\end{figure}
\end{frame}



% *** For Offset *** 

%\begin{frame}
%	\frametitle{Comparison all stations: UUB and UB Offset for Charge histograms}
%  \centering
%  \includegraphics[width=.55\textwidth]{../plots/uubAllAveOffsetChPMT3s.png}
%\end{frame}

% ===============
% *** For A/P ***

%\begin{frame}
%  \frametitle{A/P Calculation}
%  {\bf For UB}
%  \begin{figure}
%    \centering
%    \begin{tabularx}{\textwidth}{CC}
%      \includegraphics[width=.4\textwidth]{../plots/ubChargeOffsetBlPMT1.pdf}
%      &
%      \includegraphics[width=.4\textwidth]{../plots/ubChargeOffsetNoBlPMT1.pdf}
%      \\
%      \begin{tabular}{l}
%        \includegraphics[width=.4\textwidth]{../plots/ubChargeNoOffsetPMT1.pdf}
%      \end{tabular}
%      &
%      \begin{tabular}{l}
%        AoP:\\
%        * For Offset - Baseline: 3.1\\
%        * For Offset: 35.5\\
%        * For No offset: 3.7\\
%        * For Peak and Charge No offset: 1.45
%      \end{tabular}
%    \end{tabularx}
%  \end{figure}
%\end{frame}
%
%\begin{frame}
%  \frametitle{A/P Calculation}
%  {\bf For UUB}
%  \begin{figure}
%    \centering
%    \begin{tabularx}{\textwidth}{CC}
%      \includegraphics[width=.4\textwidth]{../plots/uubChargeOffsetBlPMT1.pdf}
%      &
%      \includegraphics[width=.4\textwidth]{../plots/uubChargeOffsetNoBlPMT1.pdf}
%      \\
%      \begin{tabular}{l}
%        \includegraphics[width=.4\textwidth]{../plots/uubChargeNoOffsetPMT1.pdf}
%      \end{tabular}
%      &
%      \begin{tabular}{l}
%        AoP:\\
%        * For Offset - Baseline: NaN\\
%        * For Offset: 8.4\\
%        * For No offset: 8.4\\
%        * For Peak and Charge No offset: 7.9
%      \end{tabular}
%    \end{tabularx}
%  \end{figure}
%\end{frame}

\begin{frame}
  \frametitle{A/P}
  \begin{figure}
    \centering
    \begin{tabularx}{\textwidth}{CC}
      \includegraphics[width=.45\textwidth]{../plots/uubAoPHbasePMTs.png}
      &
      \includegraphics[width=.45\textwidth]{../plots/uubAoPCalibPMTs.png}
      \\
      \includegraphics[width=.45\textwidth]{../plots/ubAoPHbasePMTs.png}
      &
      \includegraphics[width=.45\textwidth]{../plots/ubAoPCalibPMTs.png}
    \end{tabularx}
  \end{figure}
  \centering
  
\end{frame}


\begin{frame}
  \frametitle{A/P Relative difference for Peak corrections}
  \begin{figure}
    PMTs with AoP lower than $4$\,ns for UUB and $2$\,ns for UB are not considered here.
    \centering
    \begin{tabularx}{\textwidth}{CC}
      \includegraphics[width=.43\textwidth]{../plots/uubAoPDiffCaHbPMTs.png}
      &
      \includegraphics[width=.43\textwidth]{../plots/ubAoPDiffCaHbPMTs.png}
      \\
      \includegraphics[width=.43\textwidth]{../plots/aopHbaseUubUbPMTs.png}
      &
      \includegraphics[width=.43\textwidth]{../plots/aoCalibUubUbPMTs.png}
    \end{tabularx}
  \end{figure}
\end{frame}


\begin{frame}
  \frametitle{A/P along time}
  \begin{figure}
    \centering
    \begin{tabularx}{\textwidth}{CC}
      \includegraphics[width=.43\textwidth]{../plots/uububAoPtimeHbPMT1St863.png}
      &
      \includegraphics[width=.43\textwidth]{../plots/uububAoPtimeHbPMT1St1743.png}
      \\
      \includegraphics[width=.43\textwidth]{../plots/uububAoPtimeHbPMT3St1740.png}
      &
      \includegraphics[width=.43\textwidth]{../plots/uububAoPtimeHbPMT2St1219.png}
    \end{tabularx}
  \end{figure}
\end{frame}


%\begin{frame}
%  \frametitle{A/P along time}
%  \begin{figure}
%    \centering
%    \begin{tabularx}{\textwidth}{CC}
%      \includegraphics[width=.43\textwidth]{../plots/uububAoPtimeHbPMT1St863pk.png}
%      &
%      \includegraphics[width=.43\textwidth]{../plots/uububAoPtimeHbPMT1St863ch.png}
%      \\
%      \includegraphics[width=.43\textwidth]{../plots/uububAoPtimeHbPMT1St1743pk.png}
%      &
%      \includegraphics[width=.43\textwidth]{../plots/uububAoPtimeHbPMT1St1743ch.png}
%    \end{tabularx}
%  \end{figure}
%\end{frame}


%\begin{frame}
%  \frametitle{A/P along time}
%  \begin{figure}
%    \centering
%    \begin{tabularx}{\textwidth}{CC}
%      \includegraphics[width=.43\textwidth]{../plots/uububAoPtimeHbPMT3St1740pk.png}
%      &
%      \includegraphics[width=.43\textwidth]{../plots/uububAoPtimeHbPMT3St1740ch.png}
%      \\
%      \includegraphics[width=.43\textwidth]{../plots/uububAoPtimeHbPMT2St1219pk.png}
%      &
%      \includegraphics[width=.43\textwidth]{../plots/uububAoPtimeHbPMT2St1219ch.png}
%    \end{tabularx}
%  \end{figure}
%\end{frame}


\begin{frame}
  \frametitle{Summary}
  {\bf First look at the  A/P}

  %\centering
  \begin{itemize}
    \item Why are there entries in the UUB peak histograms at the first bin?
    \item What is the offset for, and why is it different between the PMTs 
      (Charge histograms)?
    \item The calibration histograms should be taken from  HBase or Calib?
    \item The A/B from the UUB is a factor 20\% lower than the A/P from UB.
  \end{itemize}
\end{frame}


\begin{frame}
  \centering
  {\bf\Huge Thanks }
\end{frame}




%\begin{frame}
%  \frametitle{A/P Calculation all stations}
%  \begin{figure}
%    \centering
%    \begin{tabularx}{\textwidth}{CC}
%      \includegraphics[width=.45\textwidth]{../plots/uubAllAoPavePMT1.png}
%      &
%      \includegraphics[width=.45\textwidth]{../plots/ubAllAoPavePMT1.png}
%      \\
%      \includegraphics[width=.45\textwidth]{../plots/uubAllAoPavePMT2.png}
%      &
%      \includegraphics[width=.45\textwidth]{../plots/ubAllAoPavePMT2.png}
%    \end{tabularx}
%  \end{figure}
%\end{frame}
%
%\begin{frame}
%  \frametitle{A/P Calculation all stations}
%  \begin{figure}
%    \centering
%    \begin{tabularx}{\textwidth}{CC}
%      \includegraphics[width=.45\textwidth]{../plots/uubAllAoPavePMT3.png}
%      &
%      \includegraphics[width=.45\textwidth]{../plots/ubAllAoPavePMT3.png}
%    \end{tabularx}
%  \end{figure}
%\end{frame}


% ==============
% *** BACKUP ***


\begin{frame}
  \frametitle{Backup}
\end{frame}

% ===============================
% *** Correction for Baseline ***

\begin{frame}
	\frametitle{UUB Peak histogram correcting baseline for: HBase and Calib.Base}
	\begin{figure}
		\centering
		\begin{tabularx}{\textwidth}{CC}
			\includegraphics[width=.38\textwidth]{../plots/uubPeakCoorHBasePMT1St863.pdf}
			&
			\includegraphics[width=.38\textwidth]{../plots/uubPeakCoorCalibBasePMT1St863.pdf}
			\\
			\includegraphics[width=.38\textwidth]{../plots/uubPeakCoorHBaseZoomPMT1St863.pdf}
			&
			\includegraphics[width=.38\textwidth]{../plots/uubPeakCoorCalibBaseZoomPMT1St863.pdf}
			\\
			\multicolumn{2}{l}{Is the correction using HBase producing negative bins?}
		\end{tabularx}
	\end{figure}
\end{frame}


% *** For first bin center ***

\begin{frame}
  \frametitle{Comparison PMT1: UUB and UB Peak histogram First Bin center}
  Here, First Bin center: GetBinCenter(1).
  \begin{figure}
    \centering
    \begin{tabularx}{\textwidth}{CC}
			\includegraphics[width=.45\textwidth]{../plots/uubAllAveBinCntrHbasePMTs.png}
			&
			\includegraphics[width=.45\textwidth]{../plots/uubAllAveBinCntrCalibPMTs.png}
      \\
			\includegraphics[width=.45\textwidth]{../plots/ubAllAveBinCntrHbasePMTs.png}
			&
			\includegraphics[width=.45\textwidth]{../plots/ubAllAveBinCntrCalibPMTs.png}
		\end{tabularx}
	\end{figure}
\end{frame}


% ====================
% *** For Baseline ***

\begin{frame}
	\frametitle{Checking UUB Baseline: IoSdStation::HBase[pmt] and Calib.Base[pmt]}
	\begin{figure}
		\centering
		\begin{tabularx}{\textwidth}{CC}
			\includegraphics[width=.4\textwidth]{../plots/uubAllAveHbasePMTs.png}
			&
      \includegraphics[width=.4\textwidth]{../plots/uubAllAveCalibPMTs.png}
			\\
			\includegraphics[width=.4\textwidth]{../plots/ubAllAveHbasePMTs.png}
			&
			\includegraphics[width=.4\textwidth]{../plots/ubAllAveCalibPMTs.png}
		\end{tabularx}
	\end{figure}
\end{frame}



\begin{frame}
  \frametitle{Station 1818}
  \centering
  \includegraphics[width=1.\textwidth]{../plots/st1818pmt3.png}
\end{frame}

\begin{frame}
  \frametitle{Station 1851}
  \centering
  \includegraphics[width=1.\textwidth]{../plots/st1851pmt1.png}
\end{frame}

\begin{frame}
  \frametitle{Station 1219}
  \centering
  \includegraphics[width=1.\textwidth]{../plots/st1219pmt3.png}
\end{frame}

\begin{frame}
  \frametitle{Station 1219}
  \centering
  \includegraphics[width=1.\textwidth]{../plots/st1740pmt3.png}
\end{frame}


%\begin{frame}
%  \frametitle{Improving fit}
%  \begin{figure}
%    \centering
%    \begin{tabularx}{\textwidth}{CC}
%      \includegraphics[width=.55\textwidth]{../plots/uubSmoothOriginalPk.pdf}
%      &
%      \includegraphics[width=.55\textwidth]{../plots/uubSmoothOriginalCh.pdf}
%    \end{tabularx}
%  \end{figure}
%\end{frame}



%\begin{frame}
%  \frametitle{A/P Relative difference for Peak corrections}
%  \begin{figure}
%    \centering
%    \begin{tabularx}{\textwidth}{CC}
%      \includegraphics[width=.5\textwidth]{../plots/uubXiRteXiHbCaPMTs.png}
%      &
%      \includegraphics[width=.5\textwidth]{../plots/ubXiRteXiHbCaPMTs.png}
%    \end{tabularx}
%  \end{figure}
%\end{frame}



\end{document}
