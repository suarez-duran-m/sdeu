\documentclass[aspectratio=169]{beamer}
\usepackage[T1]{fontenc}
\usepackage[utf8]{inputenc}
\usepackage{tikz}
\usepackage{tabularx}
\usepackage[font=scriptsize]{caption}
\usepackage{multirow}
\usepackage{ulem}
\usepackage{hyperref}
\usepackage{animate} % For gif
\usepackage{media9}
\usepackage{graphicx}
\usepackage{listings}
\usepackage{xspace}

\captionsetup[figure]{labelformat=empty}

\usetikzlibrary{tikzmark,shapes,arrows,backgrounds,fit,positioning}
\newcolumntype{C}{>{\centering\arraybackslash}X}
\newcolumntype{R}{>{\raggedleft\arraybackslash}X}

\def\Offline{\mbox{$\overline{\rm Off}$\hspace{.05em}\raisebox{.4ex}{$\underline{\rm line}$}}\xspace}
\def\OfflineB{\mbox{$\bf\overline{\rm\bf Off}$\hspace{.05em}\raisebox{.4ex}{$\bf\underline{\rm\bf line}$}}\xspace}

\addtobeamertemplate{navigation symbols}{}
{
	\insertframenumber{}
}

\beamertemplatenavigationsymbolsempty
\setbeamercolor{section in foot}{fg=white, bg=blue}
\setbeamercolor{subsection in foot}{fg=black, bg=white}
\setbeamerfont{footline}{size=\fontsize{6}{6}\selectfont}

\setbeamertemplate{footline}
{
  \leavevmode
  \hbox
  {
    \begin{beamercolorbox}
      [wd=.5\paperwidth,ht=2.5ex,dp=1.125ex,leftskip=.3cm,rightskip=.3cm]{subsection in foot}
    \end{beamercolorbox}
    
    \begin{beamercolorbox}
      [wd=.5\paperwidth,ht=2.5ex,dp=1.125ex,leftskip=.3cm,rightskip=.3cm plus1fil]{subsection in foot}
      \hfill
      \insertframenumber
      %\insertframenumber\,/\,\inserttotalframenumber
    \end{beamercolorbox}
  }
}


\title{Coincidence calibration histograms in \Offline}
\author{
  Mauricio Su\'arez Dur\'an and Ioana~C.~Mari\c{s}
}
\institute{IIHE-ULB}

\titlegraphic{
  \begin{figure}[h]
    \centering
   %\includegraphics[width=5cm]{ulbLogo2.png}
    \hspace*{8.cm}
    \includegraphics[width=5.5cm]{iihe.jpeg}
  \end{figure}
}

\begin{document}
\begin{frame}
  \titlepage
\end{frame}

\begin{frame}
  \frametitle{The code that has been written}
  \begin{itemize}
    \item Implementation of new Classes:
      \begin{itemize}
        \item<1-> CDAS-DAQ (Compiled, not run tested)
        \item<1-> CDAS-User (From ASCII files to sd and ad root          
          files.)
      \end{itemize}
      \vspace{0.2cm}

    \item<2->Implementation of new variables in \Offline (From
      sd/ad files to ADST files):
      \begin{itemize}
        \item<2-> ADST/EventBrowser
        \item<2-> ADST/RecEvent/src/Traces.cc
        \item<2-> ADST/RecEvent/src/SDEvent.cc
        \item<2-> ADST/EventBrowser/src/EventBrowserSignals.h
        \item<2-> Framework/SEvent/PMTRecData.h
        \item<2-> Framework/SEvent/PMTCalibData.h
        \item<2-> EventIO/CDAS/CDASToOfflineEventConverter.cc
        \item<2-> Modules/General/RecDataWriterNG/SD2ADST.cc
        \item<2-> Modules/SdReconstruction/SdTraceCalibratorOG/SdTraceCalibrator.cc
        \item<2-> Modules/SdReconstruction/SdHistogramFitterKG/SdHistogramFitterKG.cc
      \end{itemize}
      \vspace{0.2cm}

    \item<3->Coincidence histograms able to be include into ad,
      sd and ADST root files.
  \end{itemize}
\end{frame}

\begin{frame}
  \frametitle{Implementation tested with two set of data:}

  \begin{itemize}
    \item<1-> 1st data set: February 2nd, 9th and 10th, 2022.
    \item<1-> 2nd data set: July from 24th to 31st, 2022.
  \end{itemize}
  \vspace{0.5cm}

  With these data:
  \begin{itemize}
    \item<2-> A fitting algorithm to find the
      CQ$^{\mathrm{pk}}$/CI$^{\mathrm{pk}}$ value was implemented
    \item<2-> Reconstruction of events using a 
      Q$^{\mathrm{VEM}}$, calculated from CQ$^{\mathrm{pk}}$, was
      performed.
  \end{itemize}
      
\end{frame}

\begin{frame}
  \frametitle{Results: EventBrowser visualization}
  \vspace{0.2cm}

  \begin{columns}
    \centering
    \begin{column}{.75\textwidth}
      \includegraphics[width=1.\textwidth]{/home/msd/Pictures/eventBrowser_coincidenceHistosVEM.png}
    \end{column}
  \end{columns}
  \vspace{0.5cm}

  The symmetry and presence of a "single" peak are used to
  design the fitting algorithm.
\end{frame}

\begin{frame}[fragile]
  \frametitle{Algorithm for fitting 2nd order polynomial to
  coincidence histograms}
  As a first approach, two parameter to study:
  \vspace{0.2cm}

  \begin{itemize}
    \item The $\chi^2/$Ndof     
    \item The error of the peak obtained from the fit; named
      as ErrIpk, for pulse height histograms and ErrQpk, for
      charge histograms.
  \end{itemize}
  \vspace{0.2cm}

  The algorithm was temporally implemented in the
  SdHistogramFitterKGB.cc module:
  \vspace{0.1cm}
  \begin{itemize}
    \item Find the bin with the maximum of counts (binMax)
    \item From binMax, fit a second order polynomial from binMax - n to binMax + n.
    \item Check for the distribution of $\chi^2$ , error of fit
      as a function of the number of degree of freedom.
  \end{itemize}

  \begin{lstlisting}[language=C++, basicstyle = \ttfamily\tiny]
  for ( int nFADC =8; nFADC <100; nFADC +=4 ) {
    [...]
    MakeQuadraticFitter ( coinciPeakHisto , binMax - nFADC , binMax + nFADC ). GetFitData ( qf );
    [...]
  }
  \end{lstlisting}

\end{frame}

\begin{frame}
  \frametitle{Results: Setting the range for height pulse histograms}

  \begin{columns}
    \centering
    \begin{column}{0.5\textwidth}
      \includegraphics[width=.8\textwidth]{../coincidence_histos/cdas_dev/creating_adst/results/resLeftRight_ErrIpk.pdf}
    \end{column}
    \begin{column}{0.5\textwidth}
      \includegraphics[width=.8\textwidth]{../coincidence_histos/cdas_dev/creating_adst/results/resLeftRight_redChi2Ipk.pdf}
    \end{column}
  \end{columns}

  \begin{columns}
    \centering
    \begin{column}{0.5\textwidth}
      \includegraphics[width=.8\textwidth]{../coincidence_histos/cdas_dev/creating_adst//results/resLeftRight_logPvalIpk.pdf}
    \end{column}
    \begin{column}{0.5\textwidth}
      \includegraphics[width=.8\textwidth]{../coincidence_histos/cdas_dev/creating_adst/results/resLeftRight_succesHistsoIpk.pdf}
    \end{column}
  \end{columns}
  \vspace{0.1cm}

  Results obtained with 1st data set.\\From 19 to 35 Ndof, the
  fit that agrees with: $\sigma/$Ipk< $5$\,\%, and $\chi^2/$Ndof
  < 2.0 is the one from which the CI$^{\mathrm{pk}}$ is set.
\end{frame}

\begin{frame}
  \frametitle{Setting the range for charge histograms}

  \begin{columns}
    \centering
    \begin{column}{0.5\textwidth}
      \includegraphics[width=.8\textwidth]{../coincidence_histos/cdas_dev/creating_adst/results/resLeftRight_ErrQpk.pdf}
    \end{column}
    \begin{column}{0.5\textwidth}
      \includegraphics[width=.8\textwidth]{../coincidence_histos/cdas_dev/creating_adst//results/resLeftRight_redChi2Qpk.pdf}
    \end{column}
  \end{columns}

  \begin{columns}
    \centering
    \begin{column}{0.5\textwidth}
      \includegraphics[width=.8\textwidth]{../coincidence_histos/cdas_dev/creating_adst//results/resLeftRight_logPvalQpk.pdf}
    \end{column}
    \begin{column}{0.5\textwidth}
      \includegraphics[width=.8\textwidth]{../coincidence_histos/cdas_dev/creating_adst/results/resLeftRight_succesHistsoQpk.pdf}
    \end{column}
  \end{columns}
  \vspace{0.1cm}

  Results obtained with 2nd data set.\\
  From 71 to 121 Ndof, the fit that agrees with: $\sigma/$Qpk
  < $5$\,\%, and $\chi^2/$Ndof < 2.0 is the one from which
  the CQ$^{\mathrm{pk}}$ is set.
\end{frame}

\begin{frame}
  \frametitle{Validating with Katarina's results (1st data set)}

  \Offline:
  \begin{columns}
    \centering
    \begin{column}{0.5\textwidth}
      \includegraphics[width=1.\textwidth]{../coincidence_histos/cdas_dev/creating_adst/results/deltaPmt13.pdf}
    \end{column}
    \begin{column}{0.5\textwidth}
      \includegraphics[width=1.\textwidth]{../coincidence_histos/cdas_dev/creating_adst/results/deltaPmt2.pdf}
    \end{column}
  \end{columns}

  Katarina (Applying a cut for histograms with v/h<0.8)
  \begin{columns}
    \centering
    \begin{column}{0.5\textwidth}
      \includegraphics[width=1.\textwidth]{/home/msd/Pictures/katarina_deltaPmt_dist.png}
    \end{column}
  \end{columns}
\end{frame}

\begin{frame}
  \frametitle{Validating with Katarina's results (1st data set),
  one by one}

  \Offline:
  \begin{columns}
    \centering
    \begin{column}{0.5\textwidth}
      \includegraphics[width=.95\textwidth]{../coincidence_histos/cdas_dev/creating_adst/results/coinciFactorPmt1.pdf}
    \end{column}
    \begin{column}{0.5\textwidth}
      \includegraphics[width=.95\textwidth]{../coincidence_histos/cdas_dev/creating_adst/results/coinciFactorPmt2.pdf}
    \end{column}
  \end{columns}

  \begin{columns}
    \centering
    \begin{column}{0.5\textwidth}
      \includegraphics[width=.95\textwidth]{../coincidence_histos/cdas_dev/creating_adst/results/coinciFactorPmt3.pdf}
    \end{column}
  \end{columns}
\end{frame}

\begin{frame}
  \frametitle{Calculating $\Delta$ for 1st and 2nd data set}
  \vspace{0.2cm}

  \begin{columns}
    \centering
    \begin{column}{0.5\textwidth}
      \includegraphics[width=1.\textwidth]{../coincidence_histos/cdas_dev/creating_adst/results/CQpkOverQpkPmt1.pdf}
    \end{column}
    \begin{column}{0.5\textwidth}
      \includegraphics[width=1.\textwidth]{../coincidence_histos/cdas_dev/creating_adst/results/CQpkOverQpkPmt2.pdf}
    \end{column}
  \end{columns}
  \begin{columns}
    \centering
    \begin{column}{0.5\textwidth}
      \includegraphics[width=1.\textwidth]{../coincidence_histos/cdas_dev/creating_adst/results/CQpkOverQpkPmt3.pdf}
    \end{column}
    \begin{column}{0.5\textwidth}
      {\small
      These conversion values are included into:
      SdHistogramFitterKG.xml.in as variables named
      {\it cchargeDeltaFactorPMT\{1,2,3\}}, respectively.
      \vspace{0.2cm}

      This mean, now it is possible to choose between using
      Q$^{\mathrm{pk}}$ or CQ$^{\mathrm{pk}}$ for VEM
      calibration: SdTraceCalibrator.xml.in ({\it
      useVEMfromCoinci})
      }
    \end{column}
  \end{columns}
\end{frame}


\begin{frame}
  \frametitle{Results for reconstruction (using 1st and 2nd data
  set)}
  Relative magnitude of signals per PMT
  (Only events with signal in the three PMTs)
  \vspace{0.3cm}

  Using Q$^{\mathrm{pk}}$ 
  \begin{columns}
    \centering
    \begin{column}{0.33\textwidth}
      \includegraphics[width=1.\textwidth]{../coincidence_histos/cdas_dev/reading_adst/results/relative_S1.pdf}
    \end{column}
    \begin{column}{0.33\textwidth}
      \includegraphics[width=1.\textwidth]{../coincidence_histos/cdas_dev/reading_adst/results/relative_S2.pdf}
    \end{column}
    \begin{column}{0.33\textwidth}
      \includegraphics[width=1.\textwidth]{../coincidence_histos/cdas_dev/reading_adst/results/relative_S3.pdf}
    \end{column}
  \end{columns}
  \vspace{0.2cm}
 
  Using CQ$^{\mathrm{pk}}$ 
  \begin{columns}
    \centering
    \begin{column}{0.33\textwidth}
      \includegraphics[width=1.\textwidth]{../coincidence_histos/cdas_dev/reading_adst/results/relativeCoinci_S1.pdf}
    \end{column}
    \begin{column}{0.33\textwidth}
      \includegraphics[width=1.\textwidth]{../coincidence_histos/cdas_dev/reading_adst/results/relativeCoinci_S2.pdf}
    \end{column}
    \begin{column}{0.33\textwidth}
      \includegraphics[width=1.\textwidth]{../coincidence_histos/cdas_dev/reading_adst/results/relativeCoinci_S3.pdf}
    \end{column}
  \end{columns}
\end{frame}

\begin{frame}
  \frametitle{Outlier: event 68958543}
  \vspace{0.2cm}

  \begin{columns}
    \centering
    \begin{column}{.85\textwidth}
      \includegraphics[width=1.\textwidth]{/home/msd/Pictures/68958543_coinci_event.png}
    \end{column}
  \end{columns}
\end{frame}

\begin{frame}
  \frametitle{Outlier: event 68958543}
  \vspace{0.2cm}

  \begin{columns}
    \centering
    \begin{column}{0.33\textwidth}
      \includegraphics[width=1.\textwidth]{/home/msd/Pictures/68958543_st867.png}
    \end{column}
    \begin{column}{0.33\textwidth}
      \includegraphics[width=1.\textwidth]{/home/msd/Pictures/68958543_st859.png}
    \end{column}
    \begin{column}{0.33\textwidth}
      \includegraphics[width=1.\textwidth]{/home/msd/Pictures/68958543_st868.png}
    \end{column}
  \end{columns}
  \vspace{0.2cm}
 
  \begin{columns}
    \centering
    \begin{column}{0.33\textwidth}
      \includegraphics[width=1.\textwidth]{/home/msd/Pictures/68958543_coinci_st867.png}
    \end{column}
    \begin{column}{0.33\textwidth}
      \includegraphics[width=1.\textwidth]{/home/msd/Pictures/68958543_coinci_st859.png}
    \end{column}
    \begin{column}{0.33\textwidth}
      \includegraphics[width=1.\textwidth]{/home/msd/Pictures/68958543_coinci_st868.png}
    \end{column}
  \end{columns}
\end{frame}


\begin{frame}
  \frametitle{Results for reconstruction (using 1st and 2nd data
  set)}
  Asymmetry of signals in the station
  (Only events with signal in the three PMTs)
  \vspace{0.3cm}

  Using Q$^{\mathrm{pk}}$ 
  \begin{columns}
    \centering
    \begin{column}{0.33\textwidth}
      \includegraphics[width=1.\textwidth]{../coincidence_histos/cdas_dev/reading_adst/results/asymmetry_S1.pdf}
    \end{column}
    \begin{column}{0.33\textwidth}
      \includegraphics[width=1.\textwidth]{../coincidence_histos/cdas_dev/reading_adst/results/asymmetry_S2.pdf}
    \end{column}
    \begin{column}{0.33\textwidth}
      \includegraphics[width=1.\textwidth]{../coincidence_histos/cdas_dev/reading_adst/results/asymmetry_S3.pdf}
    \end{column}
  \end{columns}
  \vspace{0.2cm}
 
  Using CQ$^{\mathrm{pk}}$ 
  \begin{columns}
    \centering
    \begin{column}{0.33\textwidth}
      \includegraphics[width=1.\textwidth]{../coincidence_histos/cdas_dev/reading_adst/results/asymmetryCoinci_S1.pdf}
    \end{column}
    \begin{column}{0.33\textwidth}
      \includegraphics[width=1.\textwidth]{../coincidence_histos/cdas_dev/reading_adst/results/asymmetryCoinci_S2.pdf}
    \end{column}
    \begin{column}{0.33\textwidth}
      \includegraphics[width=1.\textwidth]{../coincidence_histos/cdas_dev/reading_adst/results/asymmetryCoinci_S3.pdf}
    \end{column}
  \end{columns}
\end{frame}


\begin{frame}
  \frametitle{Results for reconstruction (using 1st and 2nd data
  set)}
  Distribution of the asymmetry of signals in the station
  (Only events with signal in the three PMTs)
  \vspace{0.3cm}

  Using Q$^{\mathrm{pk}}$
  \begin{columns}
    \centering
    \begin{column}{0.33\textwidth}
      \includegraphics[width=1.\textwidth]{../coincidence_histos/cdas_dev/reading_adst/results/asymmetryDist_S1.pdf}
    \end{column}
    \begin{column}{0.33\textwidth}
      \includegraphics[width=1.\textwidth]{../coincidence_histos/cdas_dev/reading_adst/results/asymmetryDist_S2.pdf}
    \end{column}
    \begin{column}{0.33\textwidth}
      \includegraphics[width=1.\textwidth]{../coincidence_histos/cdas_dev/reading_adst/results/asymmetryDist_S3.pdf}
    \end{column}
  \end{columns}
  \vspace{0.2cm}

  Using CQ$^{\mathrm{pk}}$
  \begin{columns}
    \centering
    \begin{column}{0.33\textwidth}
      \includegraphics[width=1.\textwidth]{../coincidence_histos/cdas_dev/reading_adst/results/asymmetryDistCoinci_S1.pdf}
    \end{column}
    \begin{column}{0.33\textwidth}
      \includegraphics[width=1.\textwidth]{../coincidence_histos/cdas_dev/reading_adst/results/asymmetryDistCoinci_S2.pdf}
    \end{column}
    \begin{column}{0.33\textwidth}
      \includegraphics[width=1.\textwidth]{../coincidence_histos/cdas_dev/reading_adst/results/asymmetryDistCoinci_S3.pdf}
    \end{column}
  \end{columns}
\end{frame}


\begin{frame}
  \frametitle{Results for reconstruction (using 1st and 2nd data
  set)}
  Signal in a single PMT
  (Only events with signal in the three PMTs)

  \begin{columns}
    \centering
    \begin{column}{0.5\textwidth}
      \includegraphics[width=.9\textwidth]{../coincidence_histos/cdas_dev/reading_adst/results/signalPMT1OverTot.pdf}
    \end{column}
    \begin{column}{0.5\textwidth}
      \includegraphics[width=.9\textwidth]{../coincidence_histos/cdas_dev/reading_adst/results/signalPMT2OverTot.pdf}
    \end{column}
  \end{columns}
  \begin{columns}
    \centering
    \begin{column}{0.5\textwidth}
      \includegraphics[width=.9\textwidth]{../coincidence_histos/cdas_dev/reading_adst/results/signalPMT3OverTot.pdf}
    \end{column}
    \begin{column}{0.5\textwidth}
      An extra check needed to be sure about these differences
    \end{column}
  \end{columns}
\end{frame}

\begin{frame}
  \frametitle{Remarks}
  \begin{itemize}
    \item Coincidence histograms able to be include into ad, sd and ADST root files.
    \item EventBrowser allows to see the coincidence histograms.
    \item A fitting algorithm for coincidence histogram
      implemented and tested by using 1st data set.
    \item This fitting algorithm was validated by comparison with
      an independent method, i.e. Katarina's algorithms using
      python/scipy libraries.
    \item $\Delta$ values obtained with this \Offline
      implementation agree with Katarina's results.
    \item An extra check is needed to be sure about the possible 
      improvement in signals using coincidence histograms.
  \end{itemize}
\end{frame}

 

\end{document}
