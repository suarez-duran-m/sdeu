\documentclass[aspectratio=169]{beamer}
\usepackage[T1]{fontenc}
\usepackage[utf8]{inputenc}
\usepackage{tikz}
\usepackage{tabularx}
\usepackage[font=scriptsize]{caption}
\usepackage{multirow}
\usepackage{ulem}
\captionsetup[figure]{labelformat=empty}

\usetikzlibrary{tikzmark,shapes,arrows,backgrounds,fit,positioning}
\newcolumntype{C}{>{\centering\arraybackslash}X}
\newcolumntype{R}{>{\raggedleft\arraybackslash}X}

\addtobeamertemplate{navigation symbols}{}
{
	\insertframenumber{}
}

\beamertemplatenavigationsymbolsempty
\setbeamercolor{section in foot}{fg=white, bg=blue}
\setbeamercolor{subsection in foot}{fg=black, bg=white}
\setbeamerfont{footline}{size=\fontsize{6}{6}\selectfont}

\setbeamertemplate{footline}
{
  \leavevmode
  \hbox
  {
    \begin{beamercolorbox}
      [wd=.5\paperwidth,ht=2.5ex,dp=1.125ex,leftskip=.3cm,rightskip=.3cm]{subsection in foot}
    \end{beamercolorbox}
    
    \begin{beamercolorbox}
      [wd=.5\paperwidth,ht=2.5ex,dp=1.125ex,leftskip=.3cm,rightskip=.3cm plus1fil]{subsection in foot}
      \hfill
      \insertframenumber
      %\insertframenumber\,/\,\inserttotalframenumber
    \end{beamercolorbox}
  }
}


\title{Accuracy of $\mathrm{Q}^{\mathrm{pk}}_{\mathrm{VEM}}$ fit for UB and UUB}
\author{
  Mauricio Su\'arez Dur\'an and Ioana~C.~Mari\c{s}
}
\institute{IIHE-ULB}

\titlegraphic{
  \begin{figure}[h]
    \centering
   %\includegraphics[width=5cm]{ulbLogo2.png}
    \hspace*{8.cm}
    \includegraphics[width=5.5cm]{iihe.jpeg}
  \end{figure}
}

\begin{document}
\begin{frame}
  \titlepage
\end{frame}

\begin{frame}
  \frametitle{How does the accuracy was calculated?}
  \vspace{0.5cm}
  \begin{columns}
    \begin{column}{0.5\textwidth}
      \includegraphics<1->[width=1.\textwidth]{../plots/filteredPMTsSt843.pdf}
    \end{column}
    \begin{column}{0.5\textwidth}
      \includegraphics<3->[width=1.\textwidth]{../plots/filteredSt843.pdf}
    \end{column}
  \end{columns}
  \begin{itemize}
    \item[]<1-> For each PMT, we get the
      $Q^{\mathrm{pk}}_{\mathrm{VEM}}$ distribution, then
    \item[]<2->$Q^{\mathrm{pk}}_{i} / \left< Q^{\mathrm{pk}}_{\mathrm{VEM}}
      \right>_{\mathrm{PMT}}$ was calculated.
    \item[]<3-> From the latest distribution (right plot, magenta
      line) the RMS is divided by the respective mean.
  \end{itemize}
\end{frame}

\begin{frame}
  \frametitle{How does the accuracy was calculated?}
  \vspace{0.5cm}
  \begin{columns}
    \begin{column}{0.5\textwidth}
      \includegraphics<1->[width=1.\textwidth]{../plots/filteredPMTsSt843.pdf}
    \end{column}
    \begin{column}{0.5\textwidth}
      \includegraphics<1->[width=1.\textwidth]{../plots/filteredSt843.pdf}
    \end{column}
  \end{columns}
  \begin{itemize}
    \item[]<1-> Multiple peaks for
      $Q^{\mathrm{pk}}_{\mathrm{VEM}}$ fitted.
    \item[]<2-> So, each peak (for each PMT) was fitted by a Gaus
      function, then the respective $Q^{\mathrm{pk}}_{i} /
      \left<Q^{\mathrm{pk}}_{\mathrm{VEM}}
      \right>_{\mathrm{PMT}}$ was calculated for
      $Q^{\mathrm{pk}}_{i}\pm\sigma$, and using as
      $\left<Q^{\mathrm{pk}}_{\mathrm{VEM}}
      \right>_{\mathrm{PMT}}$ the respective $\mu$.
  \end{itemize}
\end{frame}

\begin{frame}
  \frametitle{Result $RMS/Q^{pk}_{VEM}$ distribution all UUB station}
  \vspace{0.5cm}
  \begin{columns}
    \begin{column}{.8\textwidth}
      \begin{tikzpicture}
        \node(a){\includegraphics[width=1.\textwidth]{../plots/accuracyQpksFitsUbUubAllStAllPmt_v2.pdf}};
        \node(b){\includegraphics[width=1.\textwidth]{../plots/accuracyQpksFitsUbUubAllStAllPmt_v2.pdf}};
        \node<2-> at(a.center)[draw, blue, line width=1pt,
        ellipse, rotate=0, minimum width=90pt,
        minimum height=38pt, yshift=-55pt, xshift=48pt]
        {};
        %\node<3-> at(b.center)[draw, blue,line width=1pt, ellipse, rotate=90,
        %minimum width=70pt, minimum height=60pt,
        %yshift=-50pt, xshift=-25pt]{Set 2.};
      \end{tikzpicture}
    \end{column}
  \end{columns}
\end{frame}

\begin{frame}
  \frametitle{UUB Outliers stations: 827, 1746}
  \vspace{0.5cm}
  \begin{columns}
    \begin{column}{0.5\textwidth}
      \includegraphics[width=.9\textwidth]{../plots/filteredPMTsSt827.pdf}
    \end{column}
    \begin{column}{0.5\textwidth}
      \includegraphics[width=.9\textwidth]{../plots/filteredSt827.pdf}
    \end{column}
  \end{columns}
  \begin{columns}
    \begin{column}{0.5\textwidth}
      \includegraphics[width=.9\textwidth]{../plots/filteredPMTsSt1746.pdf}
    \end{column}
    \begin{column}{0.5\textwidth}
      \includegraphics[width=.9\textwidth]{../plots/filteredSt1746.pdf}
    \end{column}
  \end{columns}
\end{frame}

\begin{frame}
  \frametitle{Two approaches to calculate the accuracy without
  this subjective $\sigma$ cut}
  \begin{enumerate}
    \item Choosing a time interval by hand.
    \item Moving window algorithm.
  \end{enumerate}
\end{frame}


\begin{frame}
  \frametitle{Choosing a time interval by hand}
  \vspace{0.5cm}
  \begin{columns}
    \begin{column}{0.5\textwidth}
      \includegraphics<1->[width=1.\textwidth]{../plots/qpksVsTimeHandSt830UB.pdf}
    \end{column}
    \begin{column}{0.5\textwidth}
      \includegraphics<1->[width=1.\textwidth]{../plots/qpksVsTimeHandSt830UUB.pdf}
    \end{column}
  \end{columns}
  \begin{columns}
    \begin{column}{1.\textwidth}
      \includegraphics<2->[width=0.5\textwidth]{../plots2/accQpkFitUbUubAllStAllPmt_hand.pdf}
    \end{column}
  \end{columns}
\end{frame}

\begin{frame}
  \frametitle{Moving window algorithm}
  \begin{enumerate}
    \item<1-> Starting with the first day of August, an average of
      $Q^\mathrm{Peak}_\mathrm{VEM}$ is calculate per day for six
      days, getting a first six-day-series.
    \item<2-> A linear fit is applied to previous six-day-series, and
      the respective slope and $\chi^2$ are stored.
    \item<3-> After the first [six-day-series]$_0$, a new
      [six-day-series]$_1$ is building replacing the sixth day in
      [six-day-series]$_0$ by the next day in the respective
      month.
    \item<4-> A check in the [six-day-series]$_i$ continuity is
      applied, i.e. checking the six days are consecutive, if not
      a new series is build starting for the first day after the
      discontinuity.
  \end{enumerate}
\end{frame}

\begin{frame}
  \frametitle{Moving window algorithm results}
  \vspace{0.5cm}
  \begin{columns}
    \begin{column}{0.5\textwidth}
      \includegraphics<1->[width=.9\textwidth]{../plots2/chi2VsSlopPmt1Ub.pdf}
    \end{column}
    \begin{column}{0.5\textwidth}
      \includegraphics<1->[width=.9\textwidth]{../plots2/chi2VsSlopPmt1Uub.pdf}
    \end{column}
  \end{columns}
  \begin{columns}
    \begin{column}{0.5\textwidth}
      \includegraphics<2->[width=.9\textwidth]{../plots2/chi2VsSlopPmt1UbProjSlop.pdf}
    \end{column}
    \begin{column}{0.5\textwidth}
      \includegraphics<2->[width=.9\textwidth]{../plots2/chi2VsSlopPmt1UubProjSlop.pdf}
    \end{column}
  \end{columns}
\end{frame}

\begin{frame}
  \frametitle{Moving window algorithm results}
  \vspace{0.5cm}
  \begin{columns}
    \begin{column}{0.5\textwidth}
      \includegraphics<1->[width=.9\textwidth]{../plots2/chi2VsSlopPmt1UbProjChi2.pdf}
    \end{column}
    \begin{column}{0.5\textwidth}
      \includegraphics<1->[width=.9\textwidth]{../plots2/chi2VsSlopPmt1UubProjChi2.pdf}
    \end{column}
  \end{columns}

  \begin{itemize}
    \item A set of $Q^\mathrm{Peak}_\mathrm{VEM}$ values, per
      PMT, were used.
    \item This set corresponds to the [six-day-series] with the
      minimum $\chi^2$ among the ones with slope between $-0.5$
      and $0.5$, both UB and UUB (1 sigma of slope
      distributions).
    \item This imply that some PMTs, from some stations, were
      discarded because there was not a [six-day-series] with a
      slope into the former range. 
    \item For instance, if the PMT$_i$ from the UB station A was discarded,
  so the same PMT$_i$ for UUB version was discarded too.
\end{frame}

\end{document}
