\documentclass[aspectratio=169]{beamer}
\usepackage[T1]{fontenc}
\usepackage[utf8]{inputenc}
\usepackage{tikz}
\usepackage{tabularx}
\usepackage[font=scriptsize]{caption}
\captionsetup[figure]{labelformat=empty}

\usetikzlibrary{tikzmark,shapes,arrows,backgrounds,fit,positioning}
\newcolumntype{C}{>{\centering\arraybackslash}X}

\addtobeamertemplate{navigation symbols}{}{
	\insertframenumber{}
	}

\title{UUB Charge and Peak histograms}
\author{
  Mauricio Su\'arez Dur\'an and Ioana~C.~Mari\c{s}
}
\institute{IIHE-ULB}

\titlegraphic{
  \begin{figure}[h]
    \centering
   %\includegraphics[width=5cm]{ulbLogo2.png}
    \hspace*{8.cm}
    \includegraphics[width=5.5cm]{iihe.jpeg}
  \end{figure}
}

\begin{document}
\begin{frame}
  \titlepage
\end{frame}


\begin{frame}
	\frametitle{UUB Charge and Peak histograms}
	\begin{itemize}
		\item Station studied: 863 1222 1219 1211 1740 1743 1221 1223 1217 1747 1741 1745 1818 1851 1729 1735 1746 1819 1791
		\item Data from CDAS.
		\item {\underline {Software CDAS, pre-production version.}}
	\end{itemize}
	\centering
	\includegraphics[width=.45\textwidth]{mapStations.pdf}
\end{frame}


% ================
% *** For Peak ***

\begin{frame}
	\frametitle{PMT 1: UUB Peak Histogram}
	Raw UUB Peak Histogram
	\begin{figure}
		\centering
		\begin{tabularx}{\textwidth}{CC}
			\begin{tabular}{l}
				\includegraphics[width=.5\textwidth]{../plots/uubRawPeakPMT1St863.pdf}
			\end{tabular}
			&
			\begin{tabular}{l}
				\includegraphics[width=.5\textwidth]{../plots/uubRawPeakZoomPMT1St863.pdf}
			\end{tabular}
			\\
			& Not zero counts in the first bin
		\end{tabularx}
	\end{figure}
\end{frame}

\begin{frame}
	\frametitle{PMT 1: UUB and UB counts in the first bin in Raw Peak Histogram}
	{\bf IoSdHisto::Peak[pmtId][0]}
	\begin{figure}
		\centering
		\begin{tabularx}{\textwidth}{CC}
			\begin{tabular}{l}
				\includegraphics[width=.45\textwidth]{../plots/uubCntFirstBinPkPMT1St863.pdf}
			\end{tabular}
			&
			\begin{tabular}{l}
				\includegraphics[width=.45\textwidth]{../plots/ubCntFirstBinPkPMT1St863.pdf}
			\end{tabular}
		\end{tabularx}
	\end{figure}
\end{frame}

% =============================================================
% *** First Bin content for All stations for Raw histograms ***

\begin{frame}
	\frametitle{PMT 1: UUB and UB counts in the first bin in Raw Peak Histogram}
	{\bf IoSdHisto::Peak[pmtId][0]}
	\begin{figure}
		\centering
		\begin{tabularx}{\textwidth}{CC}
				\includegraphics[width=.55\textwidth]{../plots/uubAllFrstBinCntPMT1.png}
			&
				\includegraphics[width=.55\textwidth]{../plots/ubAllFrstBinCntPMT1.png}
		\end{tabularx}
	\end{figure}
\end{frame}


\begin{frame}
	\frametitle{PMT 1: UUB and UB counts in the first bin in Raw Peak Histogram}
	{\bf IoSdHisto::Peak[pmtId][0]}
	\begin{figure}
		\centering
		\begin{tabularx}{\textwidth}{CC}
			\begin{tabular}{l}
				\includegraphics[width=.45\textwidth]{../plots/uubAllFrstBinCntFiltUpPMT1.png}
			\end{tabular}
			&
			\begin{tabular}{l}
				\includegraphics[width=.45\textwidth]{../plots/uubAllFrstBinCntFiltDownPMT1.png}
			\end{tabular}
			\\
			\begin{tabular}{l}
				\includegraphics[width=.45\textwidth]{../plots/uubAllFrstBinCnt1851PMT1.png}
			\end{tabular}
      & 
      \begin{tabular}{l}
        \includegraphics[width=.45\textwidth]{../plots/ubAllFrstBinCntFiltUpPMT1.png}
      \end{tabular}
		\end{tabularx}
	\end{figure}
\end{frame}


\begin{frame}
	\frametitle{PMT 2: UUB and UB counts in the first bin in Raw Peak Histogram}
	{\bf IoSdHisto::Peak[pmtId][0]}
	\begin{figure}
		\centering
		\begin{tabularx}{\textwidth}{CC}
			\begin{tabular}{l}
				\includegraphics[width=.55\textwidth]{../plots/uubAllFrstBinCntPMT2.png}
			\end{tabular}
			&
			\begin{tabular}{l}
				\includegraphics[width=.55\textwidth]{../plots/ubAllFrstBinCntPMT2.png}
			\end{tabular}
		\end{tabularx}
	\end{figure}
\end{frame}


\begin{frame}
	\frametitle{PMT 2: UUB and UB counts in the first bin in Raw Peak Histogram}
	{\bf IoSdHisto::Peak[pmtId][0]}
	\begin{figure}
		\centering
		\begin{tabularx}{\textwidth}{CC}
			\begin{tabular}{l}
				\includegraphics[width=.45\textwidth]{../plots/uubAllFrstBinCntFiltUpPMT2.png}
			\end{tabular}
			&
			\begin{tabular}{l}
				\includegraphics[width=.45\textwidth]{../plots/uubAllFrstBinCntFiltDownPMT2.png}
			\end{tabular}
			\\
			\begin{tabular}{l}
				\includegraphics[width=.45\textwidth]{../plots/uubAllFrstBinCnt1851PMT2.png}
			\end{tabular}
			& 
			\begin{tabular}{l}
				\includegraphics[width=.45\textwidth]{../plots/ubAllFrstBinCntFiltUpPMT2.png}
			\end{tabular}
		\end{tabularx}
	\end{figure}
\end{frame}


\begin{frame}
	\frametitle{PMT 3: UUB and UB counts in the first bin in Raw Peak Histogram}
	{\bf IoSdHisto::Peak[pmtId][0]}
	\begin{figure}
		\centering
		\begin{tabularx}{\textwidth}{CC}
			\begin{tabular}{l}
				\includegraphics[width=.55\textwidth]{../plots/uubAllFrstBinCntPMT3.png}
			\end{tabular}
			&
			\begin{tabular}{l}
				\includegraphics[width=.55\textwidth]{../plots/ubAllFrstBinCntPMT3.png}
			\end{tabular}
		\end{tabularx}
	\end{figure}
\end{frame}


\begin{frame}
	\frametitle{PMT 3: UUB and UB counts in the first bin in Raw Peak Histogram}
	{\bf IoSdHisto::Peak[pmtId][0]}
	\begin{figure}
		\centering
		\begin{tabularx}{\textwidth}{CC}
			\begin{tabular}{l}
				\includegraphics[width=.45\textwidth]{../plots/uubAllFrstBinCntFiltUpPMT3.png}
			\end{tabular}
			&
			\begin{tabular}{l}
				\includegraphics[width=.45\textwidth]{../plots/uubAllFrstBinCntFiltDownPMT3.png}
			\end{tabular}
			\\
			\begin{tabular}{l}
				\includegraphics[width=.45\textwidth]{../plots/uubAllFrstBinCnt1851PMT3.png}
			\end{tabular}
			&
			\begin{tabular}{l}
				\includegraphics[width=.45\textwidth]{../plots/ubAllFrstBinCntFiltUpPMT3.png}
			\end{tabular}
		\end{tabularx}
	\end{figure}
\end{frame}


% =====================================
% *** Correcting raw Peak histogram ***

\begin{frame}
	\frametitle{From UUB raw Peak histogram to the correct format}
	{\bf Applying IoSdStation::HPeak method}
	\begin{figure}
		\centering
		\begin{tabularx}{\textwidth}{CC}
			\multicolumn{2}{l}{Each bin (j) of raw histogram is corrected as:} 
			\\ [2ex]
			\multicolumn{2}{l}{xp[j] = j*mult + offset;
			xp[100 + j] = 100 * mult + bigbins * j * mult + offset;} 
			\\ [2ex]
			\multicolumn{2}{l}{Here, {\it mult} and {\it bigbins} are constants and equal to 4; 
			the {\it offset} is set for each event.}
			\\
			\begin{tabular}{l}
				As an instance, for event 61219267 the \\
				first bin is set a 273 FADC. So, the \\
				baseline needs to be discounted.\\ \\
				First, the  Offset variable is checked.\\ \\
				And, there are two methods to get \\ 
				the baseline: IoSdStation::HBase and \\
				Calib->Base.
			\end{tabular} 
			& 
			\begin{tabular}{l}
				\includegraphics[width=.55\textwidth]{../plots/uubPeakPMT1St863.pdf}
			\end{tabular}
		\end{tabularx}
	\end{figure}
\end{frame}

% ==============
% *** Offset ***

\begin{frame}
	\frametitle{Checking Offset: comparison UUB and UB Peak histograms}
	\begin{figure}
		\centering
		\begin{tabularx}{\textwidth}{CC}
			\includegraphics[width=.5\textwidth]{../plots/uubOffsetPkPMT1St863.pdf}
			&
			\includegraphics[width=.5\textwidth]{../plots/ubOffsetPkPMT1St863.pdf}
		\end{tabularx}
	\end{figure}
\end{frame}

% ===================================
% *** For Offset for All Stations ***

\begin{frame}
	\frametitle{Checking Offset PMT1: comparison UUB and UB Peak histograms}
	\begin{figure}
		\centering
		\begin{tabularx}{\textwidth}{CC}
			\includegraphics[width=.55\textwidth]{../plots/uubAllOffsetPkPMT1.png}
			&
			\includegraphics[width=.55\textwidth]{../plots/ubAllOffsetPkPMT1.png}
		\end{tabularx}
	\end{figure}
\end{frame}


\begin{frame}
	\frametitle{Checking Offset PMT2 and PMT3: comparison UUB and UB Peak histograms}
	\begin{figure}
		\centering
		\begin{tabularx}{\textwidth}{CC}
			\includegraphics[width=.45\textwidth]{../plots/uubAllOffsetPkPMT2.png}
			&
			\includegraphics[width=.45\textwidth]{../plots/ubAllOffsetPkPMT2.png}
			\\
			\includegraphics[width=.45\textwidth]{../plots/uubAllOffsetPkPMT3.png}
			&
			\includegraphics[width=.45\textwidth]{../plots/ubAllOffsetPkPMT3.png}
		\end{tabularx}
	\end{figure}
\end{frame}


% ================
% *** Baseline ***

\begin{frame}
	\frametitle{Checking UUB Baseline: IoSdStation::HBase[pmt] and Calib.Base[pmt]}
	\begin{figure}
		\centering
		\begin{tabularx}{\textwidth}{CC}
			\includegraphics[width=.4\textwidth]{../plots/uubBlHbasePMT1St863.pdf}
			&
			\includegraphics[width=.4\textwidth]{../plots/uubBlCalibPMT1St863.pdf}
			\\
			\includegraphics[width=.4\textwidth]{../plots/uubBlHistHbasePMT1St863.pdf}
			&
			\includegraphics[width=.4\textwidth]{../plots/uubBlHistCalibPMT1St863.pdf}
		\end{tabularx}
	\end{figure}
\end{frame}

\begin{frame}
	\frametitle{Checking UUB Baseline all stations: IoSdStation::HBase[pmt] and Calib.Base[pmt]}
	\begin{figure}
		\centering
		\begin{tabularx}{\textwidth}{CC}
			\includegraphics[width=.45\textwidth]{../plots/uubAllBaselineHbasePMT1.png}
			&
			\includegraphics[width=.45\textwidth]{../plots/uubAllBaselineCalibPMT1.png}
			\\
			\includegraphics[width=.45\textwidth]{../plots/ubAllBaselineHbasePMT1.png}
			&
			\includegraphics[width=.45\textwidth]{../plots/ubAllBaselineCalibPMT1.png}
		\end{tabularx}
	\end{figure}
\end{frame}


\begin{frame}
	\frametitle{Checking UUB Baseline all stations: IoSdStation::HBase[pmt] and Calib.Base[pmt]}
	\begin{figure}
		\centering
		\begin{tabularx}{\textwidth}{CC}
			\includegraphics[width=.45\textwidth]{../plots/uubAllBaselineHbasePMT2.png}
			&
			\includegraphics[width=.45\textwidth]{../plots/uubAllBaselineCalibPMT2.png}
			\\
			\includegraphics[width=.45\textwidth]{../plots/ubAllBaselineHbasePMT2.png}
			&
			\includegraphics[width=.45\textwidth]{../plots/ubAllBaselineCalibPMT2.png}
		\end{tabularx}
	\end{figure}
\end{frame}

\begin{frame}
	\frametitle{Checking UUB Baseline all stations: IoSdStation::HBase[pmt] and Calib.Base[pmt]}
	\begin{figure}
		\centering
		\begin{tabularx}{\textwidth}{CC}
			\includegraphics[width=.45\textwidth]{../plots/uubAllBaselineHbasePMT3.png}
			&
			\includegraphics[width=.45\textwidth]{../plots/uubAllBaselineCalibPMT3.png}
			\\
			\includegraphics[width=.45\textwidth]{../plots/ubAllBaselineHbasePMT3.png}
			&
			\includegraphics[width=.45\textwidth]{../plots/ubAllBaselineCalibPMT3.png}
		\end{tabularx}
	\end{figure}
\end{frame}


% ===============================
% *** Correction for Baseline ***

\begin{frame}
	\frametitle{UUB Peak histogram correcting for baseline: HBase and Calib.Base}
	\begin{figure}
		\centering
		\begin{tabularx}{\textwidth}{CC}
			\includegraphics[width=.38\textwidth]{../plots/uubPeakCoorHBasePMT1St863.pdf}
			&
			\includegraphics[width=.38\textwidth]{../plots/uubPeakCoorCalibBasePMT1St863.pdf}
			\\
			\includegraphics[width=.38\textwidth]{../plots/uubPeakCoorHBaseZoomPMT1St863.pdf}
			&
			\includegraphics[width=.38\textwidth]{../plots/uubPeakCoorCalibBaseZoomPMT1St863.pdf}
			\\
			\multicolumn{2}{l}{Is the correction using HBase producing negative bins?}
		\end{tabularx}
	\end{figure}
\end{frame}


\begin{frame}
	\frametitle{Comparison: UUB and UB Peak histogram First Bin center}
	Here, First Bin center: GetBinCenter(1).
	
	\begin{figure}
		\centering
		\begin{tabularx}{\textwidth}{CC}
			\includegraphics[width=.4\textwidth]{../plots/uubPeakFirstBinHBasePMT1St863.pdf}
			&
			\includegraphics[width=.4\textwidth]{../plots/uubPeakFirstBinCalibBasePMT1St863.pdf}
			\\
			\includegraphics[width=.4\textwidth]{../plots/ubPeakFirstBinHBasePMT1St863.pdf}
			&
			\includegraphics[width=.4\textwidth]{../plots/ubPeakFirstBinCalibBasePMT1St863.pdf}
		\end{tabularx}
	\end{figure}
\end{frame}


\begin{frame}
	\frametitle{Comparison PMT1: UUB and UB Peak histogram First Bin center}
	Here, First Bin center: GetBinCenter(1).
	\begin{figure}
		\centering
		\begin{tabularx}{\textwidth}{CC}
			\includegraphics[width=.55\textwidth]{../plots/uubAllBinCenterHbasePMT1.png}
			&
			\includegraphics[width=.55\textwidth]{../plots/uubAllBinCenterCalibPMT1.png}
    \end{tabularx}
  \end{figure}
\end{frame}


\begin{frame}
  \frametitle{Comparison PMT1: UUB and UB Peak histogram First Bin center}
  Here, First Bin center: GetBinCenter(1).
  \begin{figure}
    \centering
    \begin{tabularx}{\textwidth}{CC}
			\includegraphics[width=.45\textwidth]{../plots/uubAllBinCenterHbaseZoomPMT1.png}
			&
			\includegraphics[width=.45\textwidth]{../plots/uubAllBinCenterCalibZoomPMT1.png}
      \\
			\includegraphics[width=.45\textwidth]{../plots/ubAllBinCenterHbasePMT1.png}
			&
			\includegraphics[width=.45\textwidth]{../plots/ubAllBinCenterCalibPMT1.png}
		\end{tabularx}
	\end{figure}
\end{frame}


\begin{frame}
	\frametitle{Comparison PMT2: UUB and UB Peak histogram First Bin center}
	Here, First Bin center: GetBinCenter(1).
	\begin{figure}
		\centering
		\begin{tabularx}{\textwidth}{CC}
			\includegraphics[width=.55\textwidth]{../plots/uubAllBinCenterHbasePMT2.png}
			&
			\includegraphics[width=.55\textwidth]{../plots/uubAllBinCenterCalibPMT2.png}
    \end{tabularx}
  \end{figure}
\end{frame}

\begin{frame}
  \frametitle{Comparison PMT2: UUB and UB Peak histogram First Bin center}
  Here, First Bin center: GetBinCenter(1).
  \begin{figure}
    \centering
    \begin{tabularx}{\textwidth}{CC}
			\includegraphics[width=.45\textwidth]{../plots/uubAllBinCenterHbaseZoomPMT2.png}
			&
			\includegraphics[width=.45\textwidth]{../plots/uubAllBinCenterCalibZoomPMT2.png}
      \\
			\includegraphics[width=.45\textwidth]{../plots/ubAllBinCenterHbasePMT2.png}
			&
			\includegraphics[width=.45\textwidth]{../plots/ubAllBinCenterCalibPMT2.png}
		\end{tabularx}
	\end{figure}
\end{frame}


\begin{frame}
	\frametitle{Comparison PMT3: UUB and UB Peak histogram First Bin center}
	Here, First Bin center: GetBinCenter(1).
	\begin{figure}
		\centering
		\begin{tabularx}{\textwidth}{CC}
			\includegraphics[width=.55\textwidth]{../plots/uubAllBinCenterHbasePMT3.png}
			&
			\includegraphics[width=.55\textwidth]{../plots/uubAllBinCenterCalibPMT3.png}
    \end{tabularx}
  \end{figure}
\end{frame}

\begin{frame}
  \frametitle{Comparison PMT3: UUB and UB Peak histogram First Bin center}
  Here, First Bin center: GetBinCenter(1).
  \begin{figure}
    \centering
    \begin{tabularx}{\textwidth}{CC}
			\includegraphics[width=.45\textwidth]{../plots/uubAllBinCenterHbaseZoomPMT3.png}
			&
			\includegraphics[width=.45\textwidth]{../plots/uubAllBinCenterCalibZoomPMT3.png}
      \\
			\includegraphics[width=.45\textwidth]{../plots/ubAllBinCenterHbasePMT3.png}
			&
			\includegraphics[width=.45\textwidth]{../plots/ubAllBinCenterCalibPMT3.png}
		\end{tabularx}
	\end{figure}
\end{frame}



% ===========================
% *** Applying for Charge ***

\begin{frame}
	\frametitle{Applying the previous steps to UUB Charge histograms}
	{\bf Raw UUB Charge histogram}
	\begin{figure}
		\begin{tabularx}{\textwidth}{C}
			\includegraphics[width=.7\textwidth]{../plots/uubRawChargePMT1St863.pdf}
		\end{tabularx}
	\end{figure}
\end{frame}


\begin{frame}
	\frametitle{From UUB raw Charge histogram to the correct format}
	{\bf IoSdStation::HCharge}
	\begin{figure}
		\centering
		\begin{tabularx}{\textwidth}{CC}
			\multicolumn{2}{l}{Each bin (j) of the raw histogram is set as:} 
			\\ [2ex]
			\multicolumn{2}{l}{xc[j] = mult*j + offset; xc[400+j] = 400*mult + bigbins*mult*j + offset} 
			\\ [2ex]
			\multicolumn{2}{l}{Here, {\it mult}=8 and {\it bigbins}=4, both of them constants;} 
			\\
			\multicolumn{2}{l}{the {\it offset} is set for each event.} 
			\\
			\begin{tabular}{l}
				For event 61219267 offset = 0, with \\
				a value of 0 for the first bin of the \\
				UUB Charge histogram. \\ \\
				Is not needed a correction for baseline? \\ \\
				First, the Offset variable is checked.	
			\end{tabular} 
			& 
			\begin{tabular}{l}
				\includegraphics[width=.45\textwidth]{../plots/uubChargePMT1St863.pdf}
			\end{tabular}
		\end{tabularx}
	\end{figure}
\end{frame}

% ==================
% *** For Offset ***

\begin{frame}
	\frametitle{Checking Offset: UUB and UB for Charge histograms}
	\begin{figure}
		\centering
		\begin{tabularx}{\textwidth}{CC}
			\includegraphics[width=.5\textwidth]{../plots/uubOffsetChPMT1St863.pdf}
			&
			\includegraphics[width=.5\textwidth]{../plots/ubOffsetChPMT1St863.pdf}
			\\
			\multicolumn{2}{l}{A very different behaviour for UUB respect to UB, 
			the same for the others LPMT?}
		\end{tabularx}
	\end{figure}
\end{frame}


\begin{frame}
	\frametitle{Comparison: UUB and UB Offset for Charge histograms}
	\begin{figure}
		\centering
		\begin{tabularx}{\textwidth}{CC}
			\includegraphics[width=.36\textwidth]{../plots/uubOffsetChPMT2St863.pdf}
			&
			\includegraphics[width=.36\textwidth]{../plots/ubOffsetChPMT2St863.pdf}
			\\
			\includegraphics[width=.36\textwidth]{../plots/uubOffsetChPMT3St863.pdf}
			&
			\includegraphics[width=.36\textwidth]{../plots/ubOffsetChPMT3St863.pdf}
			\\
			\multicolumn{2}{l}{Offset for UUB LPMT3 is very different form LPMT1 and
			LPMT2, expected?}
		\end{tabularx}
	\end{figure}
\end{frame}


% *** For all stations ***

\begin{frame}
	\frametitle{Comparison all stations: UUB and UB Offset for Charge histograms}
	\begin{figure}
		\centering
		\begin{tabularx}{\textwidth}{CC}
			\includegraphics[width=.55\textwidth]{../plots/uubAllOffsetChPMT1.png}
			&
			\includegraphics[width=.55\textwidth]{../plots/ubAllOffsetChPMT1.png}
		\end{tabularx}
	\end{figure}
\end{frame}


\begin{frame}
	\frametitle{Comparison all stations: UUB and UB Offset for Charge histograms}
	\begin{figure}
		\centering
		\begin{tabularx}{\textwidth}{CC}
			\includegraphics[width=.45\textwidth]{../plots/uubAllOffsetChPMT2.png}
			&
			\includegraphics[width=.45\textwidth]{../plots/ubAllOffsetChPMT2.png}
			\\
			\includegraphics[width=.45\textwidth]{../plots/uubAllOffsetChPMT3.png}
			&
			\includegraphics[width=.45\textwidth]{../plots/ubAllOffsetChPMT3.png}
			\\
			\multicolumn{2}{l}{Offset for UUB LPMT3 is very different form LPMT1 and
			LPMT2, expected?}
		\end{tabularx}
	\end{figure}
\end{frame}


% ======================================
% *** Baseline correction for Charge ***
\begin{frame}
	\frametitle{UUB Charge histogram Correcting for baseline}
	
	\begin{figure}
		\centering
		\begin{tabularx}{\textwidth}{CC}
			\includegraphics[width=.36\textwidth]{../plots/uubChargeCoorHBasePMT1St863.pdf}
			&
			\includegraphics[width=.36\textwidth]{../plots/uubChargeCoorCalibBasePMT1St863.pdf}
			\\
			\includegraphics[width=.36\textwidth]{../plots/uubChargeCoorHBaseZoomPMT1St863.pdf}
			&
			\includegraphics[width=.36\textwidth]{../plots/uubChargeCoorCalibBaseZoomPMT1St863.pdf}
			\\
			\multicolumn{2}{l}{Does the Offset value of zero include the baseline correction?} 
		\end{tabularx}
	\end{figure}
\end{frame}


\begin{frame}
	\frametitle{Comparison with UB Charge histogram Correcting for baseline}
	
	\begin{figure}
		\centering
		\begin{tabularx}{\textwidth}{CC}
			\includegraphics[width=.36\textwidth]{../plots/ubChargeCoorHBasePMT1St863.pdf}
			&
			\includegraphics[width=.36\textwidth]{../plots/ubChargeCoorCalibBasePMT1St863.pdf}
			\\
			\includegraphics[width=.36\textwidth]{../plots/ubChargeCoorHBaseZoomPMT1St863.pdf}
			&
			\includegraphics[width=.36\textwidth]{../plots/ubChargeCoorCalibBaseZoomPMT1St863.pdf}
			\\
			\multicolumn{2}{l}{For UB the correction for baseline is needed.}
		\end{tabularx}
	\end{figure}
\end{frame}


\begin{frame}
	\frametitle{Comparison: UUB and UB Charge histogram First Bin}
	
	\begin{figure}
		\centering
		\begin{tabularx}{\textwidth}{CC}
			\includegraphics[width=.42\textwidth]{../plots/uubChargeFirstBinHBasePMT1St863.pdf}
			&
			\includegraphics[width=.42\textwidth]{../plots/uubChargeFirstBinCalibBasePMT1St863.pdf}
			\\
			\includegraphics[width=.42\textwidth]{../plots/ubChargeFirstBinHBasePMT1St863.pdf}
			&
			\includegraphics[width=.42\textwidth]{../plots/ubChargeFirstBinCalibBasePMT1St863.pdf}
		\end{tabularx}
	\end{figure}
\end{frame}

% *** For all stations ***

\begin{frame}
	\frametitle{Comparison all stations: UUB and UB Charge histogram First Bin}
	
	\begin{figure}
		\centering
		\begin{tabularx}{\textwidth}{CC}
      \includegraphics[width=.45\textwidth]{../plots/uubAllBinCenterHbaseChPMT1.png}
			&
			\includegraphics[width=.42\textwidth]{../plots/ubAllBinCenterCalibChPMT1.png}
      \\
      \includegraphics[width=.45\textwidth]{../plots/uubAllBinCenterHbaseChPMT1.png}
			&
			\includegraphics[width=.42\textwidth]{../plots/ubAllBinCenterCalibChPMT1.png}
		\end{tabularx}
	\end{figure}
\end{frame}

\begin{frame}
	\frametitle{Comparison all stations: UUB and UB Charge histogram First Bin}
	
	\begin{figure}
		\centering
		\begin{tabularx}{\textwidth}{CC}
      \includegraphics[width=.45\textwidth]{../plots/uubAllBinCenterHbaseChPMT2.png}
			&
			\includegraphics[width=.42\textwidth]{../plots/ubAllBinCenterCalibChPMT2.png}
			\\
			\includegraphics[width=.42\textwidth]{../plots/uubAllBinCenterHbaseChPMT2.png}
			&
			\includegraphics[width=.42\textwidth]{../plots/ubAllBinCenterCalibChPMT2.png}
		\end{tabularx}
	\end{figure}
\end{frame}


\begin{frame}
	\frametitle{Comparison all stations: UUB and UB Charge histogram First Bin}
	
	\begin{figure}
		\centering
		\begin{tabularx}{\textwidth}{CC}
      \includegraphics[width=.45\textwidth]{../plots/uubAllBinCenterHbaseChPMT3.png}
			&
			\includegraphics[width=.42\textwidth]{../plots/ubAllBinCenterCalibChPMT3.png}
			\\
			\includegraphics[width=.42\textwidth]{../plots/uubAllBinCenterHbaseChPMT3.png}
			&
			\includegraphics[width=.42\textwidth]{../plots/ubAllBinCenterCalibChPMT3.png}
		\end{tabularx}
	\end{figure}
\end{frame}


% =====================
% *** First Summary ***

\begin{frame}
	\frametitle{Brief summary}
	\begin{itemize}
		\item {\bf For Peak histograms:}
			\begin{itemize}
					\item For UUB baseline, the bin center is most stable using HBase 
						than using Calib.Base (slide 9); the same situation is for UB.
			\end{itemize}
			\vspace{0.5cm}
		\item {\bf For Charge histograms:}
			\begin{itemize}
					\item For UUB, from slides 14, 15 and 16 the correction for 
						baseline has not sense for LPMT1 and LPMT2, but maybe if for 
						LPMT3. Nevertheless, this correction has sense for UB (slide 17).
					\item For UB, the bin center is most stable using HBase than Calib.Base.
			\end{itemize}
	\end{itemize}
	\vspace{0.5cm}
	Lets see what about the Area over Peak.

\end{frame}

\begin{frame}
	\frametitle{Area/Peak: Fitting histograms}
	%{\bf for UUB LPMT1}
	\begin{figure}
		\centering
		\begin{tabularx}{\textwidth}{CC}
			\multicolumn{2}{l}{\bf for UUB LPMT1}
			\\
			\begin{tabular}{l}
				\includegraphics[width=.35\textwidth]{../plots/uubFitPkPMT1St863.pdf}
			\end{tabular}
			&
			\begin{tabular}{l}
				\includegraphics[width=.35\textwidth]{../plots/uubFitChPMT1St863.pdf}
			\end{tabular}
			\\
			\multicolumn{2}{l}{\bf for UB LPMT1}
			\\
			\begin{tabular}{l}
				\includegraphics[width=.35\textwidth]{../plots/ubFitPkPMT1St863.pdf}
			\end{tabular}
			&
			\begin{tabular}{l}
				\includegraphics[width=.35\textwidth]{../plots/ubFitChPMT1St863.pdf}
			\end{tabular}
		\end{tabularx}
	\end{figure}
\end{frame}


% =============================
% *** Area/Peak Calculation ***

\begin{frame}
	\frametitle{Area/Peak calculation: LPMT1}
	
	\begin{figure}
		\centering
		\begin{tabularx}{\textwidth}{CC}
			\includegraphics[width=.36\textwidth]{../plots/uubApOffsetPMT1St863.pdf}
			&
			\includegraphics[width=.36\textwidth]{../plots/uubApOffsetCalibPMT1St863.pdf}
			\\
			\includegraphics[width=.36\textwidth]{../plots/ubApOffsetPMT1St863.pdf}
			&
			\includegraphics[width=.36\textwidth]{../plots/ubApOffsetCalibPMT1St863.pdf}
			\\
			\multicolumn{2}{l}{For UUB, the A/P using Calib.Base is $\sim26.7$\,\% 
			different respect of using HBase.}
		\end{tabularx}
	\end{figure}
\end{frame}


\begin{frame}
	\frametitle{Area/Peak calculation: LPMT2}
	
	\begin{figure}
		\centering
		\begin{tabularx}{\textwidth}{CC}
			\includegraphics[width=.36\textwidth]{../plots/uubApOffsetPMT2St863.pdf}
			&
			\includegraphics[width=.36\textwidth]{../plots/uubApOffsetCalibPMT2St863.pdf}
			\\
			\includegraphics[width=.36\textwidth]{../plots/ubApOffsetPMT2St863.pdf}
			&
			\includegraphics[width=.36\textwidth]{../plots/ubApOffsetCalibPMT2St863.pdf}
			\\
			\multicolumn{2}{l}{For UUB, the A/P using Calib.Base is $\sim-8.2$\,\% 
			different respect of using HBase.}
		\end{tabularx}
	\end{figure}
\end{frame}


\begin{frame}
	\frametitle{Area/Peak calculation: LPMT3}
	
	\begin{figure}
		\centering
		\begin{tabularx}{\textwidth}{CC}
			\includegraphics[width=.36\textwidth]{../plots/uubApOffsetPMT3St863.pdf}
			&
			\includegraphics[width=.36\textwidth]{../plots/uubApOffsetCalibPMT3St863.pdf}
			\\
			\includegraphics[width=.36\textwidth]{../plots/ubApOffsetPMT3St863.pdf}
			&
			\includegraphics[width=.36\textwidth]{../plots/ubApOffsetCalibPMT3St863.pdf}
			\\
			\multicolumn{2}{l}{For UUB, the A/P using Calib.Base is $\sim29.4$\,\% 
			different respect of using HBase.}
		\end{tabularx}
	\end{figure}
\end{frame}


% *** For all stations ***

\begin{frame}
	\frametitle{Area/Peak calculation for all stations: LPMT1}
	
	\begin{figure}
		\centering
		\begin{tabularx}{\textwidth}{CC}
			\includegraphics[width=.45\textwidth]{../plots/uubAllapHbasePMT1.png}
			&
			\includegraphics[width=.45\textwidth]{../plots/uubAllapCalibPMT1.png}
			\\
			\includegraphics[width=.45\textwidth]{../plots/ubAllapHbasePMT1.png}
			&
			\includegraphics[width=.45\textwidth]{../plots/ubAllapCalibPMT1.png}
		\end{tabularx}
	\end{figure}
\end{frame}

\begin{frame}
	\frametitle{Area/Peak calculation for all stations: LPMT2}
	
	\begin{figure}
		\centering
		\begin{tabularx}{\textwidth}{CC}
			\includegraphics[width=.45\textwidth]{../plots/uubAllapHbasePMT2.png}
			&
			\includegraphics[width=.45\textwidth]{../plots/ubAllapCalibPMT2.png}
			\\
			\includegraphics[width=.45\textwidth]{../plots/uubAllapHbasePMT2.png}
			&
			\includegraphics[width=.45\textwidth]{../plots/ubAllapCalibPMT2.png}
		\end{tabularx}
	\end{figure}
\end{frame}

\begin{frame}
	\frametitle{Area/Peak calculation for all stations: LPMT3}
	
	\begin{figure}
		\centering
		\begin{tabularx}{\textwidth}{CC}
			\includegraphics[width=.45\textwidth]{../plots/ubAllapHbasePMT3.png}
			&
			\includegraphics[width=.45\textwidth]{../plots/ubAllapCalibPMT3.png}
			\\
			\includegraphics[width=.45\textwidth]{../plots/ubAllapHbasePMT3.png}
			&
			\includegraphics[width=.45\textwidth]{../plots/ubAllapCalibPMT3.png}
		\end{tabularx}
	\end{figure}
\end{frame}



% =====================================
% *** For observations and comments ***

\begin{frame}
	\frametitle{Final comments and questions}
	
	\begin{figure}
		\centering
		\begin{tabularx}{\textwidth}{CC}
			\includegraphics[width=.42\textwidth]{../plots/uubRawPeakPMT1St863.pdf}
			&
			\includegraphics[width=.42\textwidth]{../plots/uubRawChargePMT1St863.pdf}
		\end{tabularx}
	\end{figure}

	\begin{itemize}
		\item In the Peak histogram is seen a peak before the EM one, this could 
			be noise, but in the Charge histogram there is not evidence of this noise,
			i.e. if considering a noise as a pulse of one single bin, so its charge 
			should be equal to the peak.
			\vspace{0.1cm}
		\item Why is the Offset setting to zero for LPMT1 and LPMT2? Why is it not
			zero for LPMT3? How is the Offset getting its value? Why is not the same 
			in UB?
			\vspace{0.1cm}
		\item For UUB, why is so different the A/P value for HBase compares to 
			Calib.Base?
			\vspace{0.1cm}
		\item {\bf Could this Offset-issue affect the A/P calculation?} How to be 
			sure?
	\end{itemize}
\end{frame}


\begin{frame}
	\frametitle{Next steps}
	\begin{itemize}
		\item To Extend this study to the others stations
	\end{itemize}
\end{frame}


% ===================
% *** For  BACKUP ***

\begin{frame}
	
	{\Huge \bf BACKUP...}

\end{frame}



\begin{frame}
	\frametitle{UUB Peak histogram Correcting for baseline}
	
	\begin{figure}
		\centering
		\begin{tabularx}{\textwidth}{CC}
			\begin{tabular}{l}
				Baseline histogram getting from \\ 
				IoSdStation::HBase \\
				xb[j] = j+offset; \\
				yb[j] = Histo->Base[pmt][j]
			\end{tabular}
			&
			\begin{tabular}{l} 
				\includegraphics[width=.34\textwidth]{../plots/uubBasePMT1St863.pdf}
			\end{tabular}
			\\
			\includegraphics[width=.44\textwidth]{../plots/uubPeakPMT1St863.pdf}
			&
			\includegraphics[width=.44\textwidth]{../plots/uubPeakCoorBlPMT1St863.pdf}
			\\
			Not corrected & Corrected, but shows negative bins\\
		\end{tabularx}
	\end{figure}
\end{frame}


\begin{frame}
	\frametitle{UUB Peak histogram Correcting for Offset}
	
	\begin{figure}
		\centering
		\begin{tabularx}{\textwidth}{CC}
			\includegraphics[width=.38\textwidth]{../plots/uubPeakPMT1St863.pdf}
			&
			\includegraphics[width=.38\textwidth]{../plots/uubPeakCoorOffPMT1St863.pdf}
			\\
			Not corrected & Corrected
			\\
			\begin{tabular}{l}
				Zooming to check the threshold, \\
				but no one can be seen.
			\end{tabular}
			&
			\begin{tabular}{l}
				\includegraphics[width=.38\textwidth]{../plots/uubPeakCoorOffZoomPMT1St863.pdf}
			\end{tabular}
		\end{tabularx}
	\end{figure}
\end{frame}

% =======================
% *** Offset for Peak ***

\begin{frame}
	\frametitle{Offset values for UUB Peak histograms}
	{\bf For LPMT1}
	\begin{figure}
		\centering
		\begin{tabularx}{\textwidth}{CC}
			\includegraphics[width=.5\textwidth]{../plots/uubOffsetPkPMT1St863.pdf}
			&
			\includegraphics[width=.5\textwidth]{../plots/uubOffsetDiffPkPMT1St863.pdf}
		\end{tabularx}
	\end{figure}
\end{frame}
			
			
\begin{frame}
	\frametitle{Offset values for UUB Peak histograms}
	{\bf For LPMT2 and LPMT3}
	\begin{figure}
		\centering
		\begin{tabularx}{\textwidth}{CC}
			\includegraphics[width=.4\textwidth]{../plots/uubOffsetPkPMT2St863.pdf}
			&
			\includegraphics[width=.4\textwidth]{../plots/uubOffsetDiffPkPMT2St863.pdf}
			\\
			\includegraphics[width=.4\textwidth]{../plots/uubOffsetPkPMT3St863.pdf}
			&
			\includegraphics[width=.4\textwidth]{../plots/uubOffsetDiffPkPMT3St863.pdf}
		\end{tabularx}
	\end{figure}
\end{frame}


\begin{frame}
	\frametitle{How to correct to obtain the right format?}
	{\bf Here, the correct format is one that shows some threshold.\\}
	\vspace{0.75cm}
	Two methos to check the baseline:
	\vspace{0.5cm}
	\begin{enumerate}
		\item IoSdStation::HBase[pmt], return a histogram with the baseline ditribution,
			according with: histo->GetXaxis()->Set(20, offset, offset + 20); so, this \\
			method depends on the Offset.
		\item Calib->Base[pmt] of the IoSdCalib class. This returns the value of the \\
			baseline, I guess it coming from some measurement performed at the WCD, \\
			but I am not sure.
	\end{enumerate}
\end{frame}



% ========================
% *** Baseline sources ***



\begin{frame}
	\frametitle{UUB Baseline values IoSdStation::HBase[pmt] and Calib->Base[pmt]}
	\centering
	\includegraphics[width=.8\textwidth]{../plots/uubDiffBlHbaseCalibPMT1St863.pdf}
\end{frame}


\begin{frame}
	\frametitle{Comparison: UUB Baseline and UB Baseline}
	\begin{figure}
		\centering
		\begin{tabularx}{\textwidth}{CC}
			\includegraphics[width=.4\textwidth]{../plots/uubBlHbasePMT1St863.pdf}
			&
			\includegraphics[width=.4\textwidth]{../plots/ubBlHbasePMT1St863.pdf}
			\\
			\includegraphics[width=.4\textwidth]{../plots/uubBlCalibPMT1St863.pdf}
			&
			\includegraphics[width=.4\textwidth]{../plots/ubBlCalibPMT1St863.pdf}
		\end{tabularx}
	\end{figure}
\end{frame}


\begin{frame}
	\frametitle{Comparison: UUB Baseline and UB Baseline}
	\begin{figure}
		\centering
		\begin{tabularx}{\textwidth}{CC}
			\includegraphics[width=.4\textwidth]{../plots/uubBlHistHbasePMT1St863.pdf}
			&
			\includegraphics[width=.4\textwidth]{../plots/ubBlHistHbasePMT1St863.pdf}
			\\
			\includegraphics[width=.4\textwidth]{../plots/uubBlHistCalibPMT1St863.pdf}
			&
			\includegraphics[width=.4\textwidth]{../plots/ubBlHistCalibPMT1St863.pdf}
		\end{tabularx}
	\end{figure}
\end{frame}


\begin{frame}
	\frametitle{Comparison: UUB Baseline and UB Baseline}
	\begin{figure}
		\centering
		\begin{tabularx}{\textwidth}{CC}
			\includegraphics[width=.5\textwidth]{../plots/uubDiffBlHbaseCalibPMT1St863.pdf}
			&
			\includegraphics[width=.5\textwidth]{../plots/ubDiffBlHbaseCalibPMT1St863.pdf}
		\end{tabularx}
	\end{figure}
\end{frame}


\begin{frame}
	\frametitle{UUB Baseline values IoSdStation::HBase[pmt] and Calib->Base[pmt]}
	\centering
	\includegraphics[width=.6\textwidth]{../plots/uubDiffBlOffsetPkCalibPMT1St863.pdf}
	\vspace{0.2cm}

	{\bf The Offset is increasing with the time, so this explain why the baseline 
	from HBase is increasing too.}
\end{frame}


\begin{frame}
	\frametitle{From UUB raw Peak histogram to the correct format, via Calib.Base[pmt]}
	\begin{figure}
		\centering
		\begin{tabularx}{\textwidth}{CC}
			\includegraphics[width=.45\textwidth]{../plots/uubPeakCorrCalibPMT1St863.pdf}
			&
			\includegraphics[width=.45\textwidth]{../plots/uubPeakCorrCalibZoomPMT1St863.pdf}
			\\
			\multicolumn{2}{c}{Now, a threshold is seen, the correct one?}
		\end{tabularx}
	\end{figure}
\end{frame}


\begin{frame}
	\frametitle{From UUB raw Peak histogram to the correct format, via Calib.Base[pmt]}
	UUB first bin for peak histograms:

	\begin{figure}
		\centering
		\begin{tabularx}{\textwidth}{CC}
			\includegraphics[width=.5\textwidth]{../plots/uubFirstBinPeakCrrCalibPMT1St863.pdf}
			&
			\includegraphics[width=.5\textwidth]{../plots/ubFirstBinPeakCrrCalibPMT1St863.pdf}
			\\
			$41*(2 \mathrm{kV}/2^{12}) = 20.0\mathrm{mV}$ 
			&
			$4*(2 \mathrm{kV}/2^{10}) = 7.8 \mathrm{mV}$
			%\multicolumn{2}{l}{\bf The threshold should be a constant, not too much here...}
		\end{tabularx} 
	\end{figure}
\end{frame}



% ==================
% *** For Charge ***

\begin{frame}
	\frametitle{Applying the previous steps to UUB Charge histograms}
	{\bf Raw UUB Charge histogram}
	\begin{figure}
		\begin{tabularx}{\textwidth}{C}
			\includegraphics[width=.7\textwidth]{../plots/uubRawChargePMT1St863.pdf}
		\end{tabularx}
	\end{figure}
\end{frame}


\begin{frame}
	\frametitle{From UUB raw Charge histogram to the correct format}
	{\bf IoSdStation::HCharge}
	\begin{figure}
		\centering
		\begin{tabularx}{\textwidth}{CC}
			\multicolumn{2}{l}{Each bin (j) of the raw histogram is set as:} 
			\\ [2ex]
			\multicolumn{2}{l}{xc[j] = mult*j + offset; xc[400+j] = 400*mult + bigbins*mult*j + offset} 
			\\ [2ex]
			\multicolumn{2}{l}{Here, {\it mult}=8 and {\it bigbins}=4, both of them constants;} 
			\\
			\multicolumn{2}{l}{the {\it offset} is set for each event.} 
			\\
			\begin{tabular}{l}
				For event 61219267 offset = 0, with \\
				a value of 0 for the first bin of the \\ \\
				UUB Charge histogram.
				\\
				Not correction for baseline?
			\end{tabular} 
			& 
			\begin{tabular}{l}
				\includegraphics[width=.45\textwidth]{../plots/uubChargePMT1St863.pdf}
			\end{tabular}
		\end{tabularx}
	\end{figure}
\end{frame}


% =========================
% *** Offset for Charge ***

\begin{frame}
	\frametitle{Offset values for UUB Charge histograms}
	{\bf For LPMT1}
	\begin{figure}
		\centering
		\begin{tabularx}{\textwidth}{CC}
			\includegraphics[width=.5\textwidth]{../plots/uubOffsetChPMT1St863.pdf}
			&
			\includegraphics[width=.5\textwidth]{../plots/uubOffsetDiffChPMT1St863.pdf}
		\end{tabularx}
	\end{figure}
\end{frame}
			
			
\begin{frame}
	\frametitle{Offset values for UUB Charge histograms}
	{\bf For LPMT2 and LPMT3}
	\begin{figure}
		\centering
		\begin{tabularx}{\textwidth}{CC}
			\includegraphics[width=.4\textwidth]{../plots/uubOffsetChPMT2St863.pdf}
			&
			\includegraphics[width=.4\textwidth]{../plots/uubOffsetDiffChPMT2St863.pdf}
			\\
			\includegraphics[width=.4\textwidth]{../plots/uubOffsetChPMT3St863.pdf}
			&
			\includegraphics[width=.4\textwidth]{../plots/uubOffsetDiffChPMT3St863.pdf}
		\end{tabularx}
	\end{figure}
\end{frame}


% =====================
% *** UB Comparison ***

% *** For Peak ***

\begin{frame}
	\frametitle{Comparison with UB version, same station}
	Raw UB Peak Histogram
	\begin{figure}
		\centering
		\begin{tabularx}{\textwidth}{CC}
			\begin{tabular}{l}
				\includegraphics[width=.5\textwidth]{../plots/ubRawPeakPMT1St863.pdf}
			\end{tabular}
			&
			\begin{tabular}{l}
				\includegraphics[width=.5\textwidth]{../plots/ubRawPeakZoomPMT1St863.pdf}
			\end{tabular}
		\end{tabularx}
	\end{figure}
\end{frame}


\begin{frame}
	\frametitle{From UB raw Peak histograma to the correct format}
	{\bf IoSdStation::HPeak}
	\begin{figure}
		\centering
		\begin{tabularx}{\textwidth}{CC}
			\multicolumn{2}{l}{Each bin (j) of raw Peak histogram is set as:} 
			\\ [2ex]
			\multicolumn{2}{l}{xp[j] = j*mult + offset; 
			xp[100 + j] = 100 * mult + bigbins * j * mult + offset;} 
			\\ [2ex]
			\multicolumn{2}{l}{Here, {\it mult}, {\it bigbins}, and {\it offset} are
			constants and fixed in the code with values of:}
			\\
			\multicolumn{2}{l}{mult=1, bigbins=3 y offset = 0.}
			\\ [2ex]
			\begin{tabular}{l}
				For event 61219267 offset = 57, with \\
				a value of 57 for the first of the histogram.
			\end{tabular} 
			&
			\begin{tabular}{l}
				\includegraphics[width=.48\textwidth]{../plots/ubPeakPMT1St863.pdf}
			\end{tabular}
		\end{tabularx}
	\end{figure}
\end{frame}


\begin{frame}
	\frametitle{UB Peak histogram Correcting for baseline}
	
	\begin{figure}
		\centering
		\begin{tabularx}{\textwidth}{CC}
			\begin{tabular}{l}
				Baseline from IoSdStation::HBase
			\end{tabular}
			&
			\begin{tabular}{l} 
				\includegraphics[width=.35\textwidth]{../plots/ubBasePMT1St863.pdf}
			\end{tabular}
			\\
			\includegraphics[width=.45\textwidth]{../plots/ubPeakPMT1St863.pdf}
			&
			\includegraphics[width=.45\textwidth]{../plots/ubPeakCoorBlPMT1St863.pdf}
			\\
			Not corrected & Corrected \\
		\end{tabularx}
	\end{figure}
\end{frame}


\begin{frame}
	\frametitle{UB Peak histogram Correcting for Offset}
	
	\begin{figure}
		\centering
		\begin{tabularx}{\textwidth}{CC}
			\includegraphics[width=.38\textwidth]{../plots/ubPeakPMT1St863.pdf}
			&
			\includegraphics[width=.38\textwidth]{../plots/ubPeakCoorOffPMT1St863.pdf}
			\\
			Not corrected & Corrected
			\\
			\begin{tabular}{l}
				Zooming to check the threshold
			\end{tabular}
			&
			\begin{tabular}{l}
				\includegraphics[width=.38\textwidth]{../plots/ubPeakCoorOffZoomPMT1St863.pdf}
			\end{tabular}
		\end{tabularx}
	\end{figure}
\end{frame}


% =======================
% *** Offset for Peak ***

\begin{frame}
	\frametitle{Offset values for UB Peak histograms}
	{\bf For LPMT1}
	\begin{figure}
		\centering
		\begin{tabularx}{\textwidth}{CC}
			\includegraphics[width=.5\textwidth]{../plots/ubOffsetPkPMT1St863.pdf}
			&
			\includegraphics[width=.5\textwidth]{../plots/ubOffsetDiffPkPMT1St863.pdf}
		\end{tabularx}
	\end{figure}
\end{frame}
			
			
\begin{frame}
	\frametitle{Offset values for UB Peak histograms}
	{\bf For LPMT2 and LPMT3}
	\begin{figure}
		\centering
		\begin{tabularx}{\textwidth}{CC}
			\includegraphics[width=.4\textwidth]{../plots/ubOffsetPkPMT2St863.pdf}
			&
			\includegraphics[width=.4\textwidth]{../plots/ubOffsetDiffPkPMT2St863.pdf}
			\\
			\includegraphics[width=.4\textwidth]{../plots/ubOffsetPkPMT3St863.pdf}
			&
			\includegraphics[width=.4\textwidth]{../plots/ubOffsetDiffPkPMT3St863.pdf}
		\end{tabularx}
	\end{figure}
\end{frame}


% ==================
% *** For Charge ***

\begin{frame}
	\frametitle{Applying the previous steps to UB Charge histograms}
	{\bf Raw UB Charge histogram}
	\begin{figure}
		\begin{tabularx}{\textwidth}{C}
			\includegraphics[width=.7\textwidth]{../plots/ubRawChargePMT1St863.pdf}
		\end{tabularx}
	\end{figure}
\end{frame}


\begin{frame}
	\frametitle{From UB raw Charge histogram to the correct format}
	{\bf IoSdStation::HCharge}
	\begin{figure}
		\centering
		\begin{tabularx}{\textwidth}{CC}
			\multicolumn{2}{l}{Each bin (j) of the raw Charge histogram is set as:} 
			\\ [2ex]
			\multicolumn{2}{l}{xc[j] = mult*j + offset; xc[400+j] = 400*mult + bigbins*mult*j + offset} 
			\\ [2ex]
			\multicolumn{2}{l}{Here, {\it mult}, {\it bigbins} and {\it offset} are constants 
			and fixed in the code with values of:}
			\\
			\multicolumn{2}{l}{mult=1, bigbins=3 y offset = 0.}
			\\
			\begin{tabular}{l}
				For event 61219267 offset = 1155, with \\ 
				a value of 1155 for the first bin of the \\
				UB Charge histogram.
			\end{tabular} 
			& 
			\begin{tabular}{l}
				\includegraphics[width=.5\textwidth]{../plots/ubChargePMT1St863.pdf}
			\end{tabular}
		\end{tabularx}
	\end{figure}
\end{frame}


% =========================
% *** Offset for Charge ***

\begin{frame}
	\frametitle{Offset values for UB Charge histograms}
	{\bf For LPMT1}
	\begin{figure}
		\centering
		\begin{tabularx}{\textwidth}{CC}
			\includegraphics[width=.5\textwidth]{../plots/ubOffsetChPMT1St863.pdf}
			&
			\includegraphics[width=.5\textwidth]{../plots/ubOffsetDiffChPMT1St863.pdf}
		\end{tabularx}
	\end{figure}
\end{frame}
			
			
\begin{frame}
	\frametitle{Offset values for UB Charge histograms}
	{\bf For LPMT2 and LPMT3}
	\begin{figure}
		\centering
		\begin{tabularx}{\textwidth}{CC}
			\includegraphics[width=.4\textwidth]{../plots/ubOffsetChPMT2St863.pdf}
			&
			\includegraphics[width=.4\textwidth]{../plots/ubOffsetDiffChPMT2St863.pdf}
			\\
			\includegraphics[width=.4\textwidth]{../plots/ubOffsetChPMT3St863.pdf}
			&
			\includegraphics[width=.4\textwidth]{../plots/ubOffsetDiffChPMT3St863.pdf}
		\end{tabularx}
	\end{figure}
\end{frame}


\begin{frame}
	\frametitle{Checking Offset differences for more UUB Satations}
	{\bf Stations to check:}
	\begin{figure}
		\centering
		\begin{tabularx}{\textwidth}{CC}
			\begin{tabular}{l}
				\includegraphics[width=.35\textwidth]{listStations.png}
			\end{tabular}
			&
			\begin{tabular}{l}
				\includegraphics[width=.45\textwidth]{mapStations.pdf}
			\end{tabular}
		\end{tabularx}
	\end{figure}
\end{frame}


\begin{frame}
	\frametitle{Checking Offset differences for more UUB Satations}
	{\bf Checking for LPMT1, Peak and Charge}
	\begin{figure}
		\centering
		\begin{tabularx}{\textwidth}{CC}
			\begin{tabular}{l}
				\includegraphics[width=.5\textwidth]{../plots/uubDiffOffsetPeakPMT1.pdf}
			\end{tabular}
			&
			\begin{tabular}{l}
				\includegraphics[width=.5\textwidth]{../plots/uubDiffOffsetChargePMT1.pdf}
			\end{tabular}
		\end{tabularx}
	\end{figure}
\end{frame}


\begin{frame}
	\frametitle{Checking Offset differences for more UUB Satations}
	{\bf Checking for LPMT2 and LPMT3, Peak and Charge}
	\begin{figure}
		\centering
		\begin{tabularx}{\textwidth}{CC}
			\begin{tabular}{l}
				\includegraphics[width=.38\textwidth]{../plots/uubDiffOffsetPeakPMT2.pdf}
			\end{tabular}
			&
			\begin{tabular}{l}
				\includegraphics[width=.38\textwidth]{../plots/uubDiffOffsetChargePMT2.pdf}
			\end{tabular}
			\\
			\begin{tabular}{l}
				\includegraphics[width=.38\textwidth]{../plots/uubDiffOffsetPeakPMT3.pdf}
			\end{tabular}
			&
			\begin{tabular}{l}
				\includegraphics[width=.38\textwidth]{../plots/uubDiffOffsetChargePMT3.pdf}
			\end{tabular}
		\end{tabularx}
	\end{figure}
\end{frame}


% =================
% *** Area/Peak ***

\begin{frame}
	\frametitle{Area/Peak: Fitting histograms}
	%{\bf for UUB LPMT1}
	\begin{figure}
		\centering
		\begin{tabularx}{\textwidth}{CC}
			\multicolumn{2}{l}{\bf for UUB LPMT1}
			\\
			\begin{tabular}{l}
				\includegraphics[width=.35\textwidth]{../plots/uubFitPkPMT1St863.pdf}
			\end{tabular}
			&
			\begin{tabular}{l}
				\includegraphics[width=.35\textwidth]{../plots/uubFitChPMT1St863.pdf}
			\end{tabular}
			\\
			\multicolumn{2}{l}{\bf for UB LPMT1}
			\\
			\begin{tabular}{l}
				\includegraphics[width=.35\textwidth]{../plots/ubFitPkPMT1St863.pdf}
			\end{tabular}
			&
			\begin{tabular}{l}
				\includegraphics[width=.35\textwidth]{../plots/ubFitChPMT1St863.pdf}
			\end{tabular}
		\end{tabularx}
	\end{figure}
\end{frame}


\begin{frame}
	\frametitle{Area/Peak: Fitting histograms}
	%{\bf for UUB LPMT1}
	\begin{figure}
		\centering
		\begin{tabularx}{\textwidth}{CC}
			\begin{tabular}{l}
				\includegraphics[width=.5\textwidth]{../plots/uubApPMT1St863.pdf}
			\end{tabular}
			&
			\begin{tabular}{l}
				\includegraphics[width=.5\textwidth]{../plots/ubApPMT1St863.pdf}
			\end{tabular}
		\end{tabularx}
	\end{figure}
\end{frame}




\end{document}
