\documentclass[aspectratio=169]{beamer}
\usepackage[T1]{fontenc}
\usepackage[utf8]{inputenc}
\usepackage{tikz}
\usepackage{tabularx}
\usepackage[font=scriptsize]{caption}
\usepackage{multirow}
\usepackage{ulem}
\usepackage{hyperref}
\usepackage{animate} % For gif
\usepackage{media9}
\usepackage{graphicx}

\captionsetup[figure]{labelformat=empty}

\usetikzlibrary{tikzmark,shapes,arrows,backgrounds,fit,positioning}
\newcolumntype{C}{>{\centering\arraybackslash}X}
\newcolumntype{R}{>{\raggedleft\arraybackslash}X}

\addtobeamertemplate{navigation symbols}{}
{
	\insertframenumber{}
}

\beamertemplatenavigationsymbolsempty
\setbeamercolor{section in foot}{fg=white, bg=blue}
\setbeamercolor{subsection in foot}{fg=black, bg=white}
\setbeamerfont{footline}{size=\fontsize{6}{6}\selectfont}

\setbeamertemplate{footline}
{
  \leavevmode
  \hbox
  {
    \begin{beamercolorbox}
      [wd=.5\paperwidth,ht=2.5ex,dp=1.125ex,leftskip=.3cm,rightskip=.3cm]{subsection in foot}
    \end{beamercolorbox}
    
    \begin{beamercolorbox}
      [wd=.5\paperwidth,ht=2.5ex,dp=1.125ex,leftskip=.3cm,rightskip=.3cm plus1fil]{subsection in foot}
      \hfill
      \insertframenumber
      %\insertframenumber\,/\,\inserttotalframenumber
    \end{beamercolorbox}
  }
}


\title{Accuracy of $\mathrm{Q}^{\mathrm{pk}}_{\mathrm{VEM}}$ fit for UB and UUB}
\author{
  Mauricio Su\'arez Dur\'an and Ioana~C.~Mari\c{s}
}
\institute{IIHE-ULB}

\titlegraphic{
  \begin{figure}[h]
    \centering
   %\includegraphics[width=5cm]{ulbLogo2.png}
    \hspace*{8.cm}
    \includegraphics[width=5.5cm]{iihe.jpeg}
  \end{figure}
}

\begin{document}
\begin{frame}
  \titlepage
\end{frame}

\begin{frame}
  \frametitle{Moving window algorithm results}
  \vspace{0.5cm}
  \begin{columns}
    \begin{column}{0.5\textwidth}
      \includegraphics[width=.9\textwidth]{../plots2/chi2VsSlopPmt1Ub.pdf}
    \end{column}
    \begin{column}{0.5\textwidth}
      \includegraphics[width=.9\textwidth]{../plots2/chi2VsSlopPmt1Uub.pdf}
    \end{column}
  \end{columns}
  \begin{columns}
    \begin{column}{0.5\textwidth}
      \includegraphics[width=.9\textwidth]{../plots2/chi2VsSlopPmt2Ub.pdf}
    \end{column}
    \begin{column}{0.5\textwidth}
      \includegraphics[width=.9\textwidth]{../plots2/chi2VsSlopPmt2Uub.pdf}
    \end{column}
  \end{columns}
\end{frame}

\begin{frame}
  \frametitle{Moving window algorithm results}
  \vspace{0.5cm}
  \begin{columns}
    \begin{column}{0.5\textwidth}
      \includegraphics[width=.9\textwidth]{../plots2/chi2VsSlopPmt3Ub.pdf}
    \end{column}
    \begin{column}{0.5\textwidth}
      \includegraphics[width=.9\textwidth]{../plots2/chi2VsSlopPmt3Uub.pdf}
    \end{column}
  \end{columns}
\end{frame}



%\begin{frame}
%  \frametitle{How the moving window works}
%  {\bf The goal:} To choose the more stable 7-days in a row for
%  fitted $Q^{Pk}_{\mathrm{VEM}}$.
%  \vspace{0.2cm}
%  \begin{enumerate}
%      {\footnotesize
%    \item An $\left<Q^{Pk}_{\mathrm{VEM}}\right>$ is calculate
%      per day, then a first 7-day-series of
%      $\left<Q^{Pk}_{\mathrm{VEM}}\right>$ is built.
%    \item A linear fit is applied to the 7-day-series, and
%      the respective slope and $\chi^2$ are stored.
%    \item A new [7-day-series]$_1$ is built by replacing the
%      seventh day in previous [7-day-series]$_0$ by next day.
%    \item A check for continuity is applied, i.e. if 7 days are
%      not consecutive a new series is built, e.g. if series
%      $i$ has a discontinuity in day 3 jumping to day 5, a new
%      7-day-series is calculated from day 5.
%      }
%  \end{enumerate}
%  \begin{center}
%    \animategraphics[label=taylor, controls, loop, width=0.4\textwidth]{2}{/home/msd/postdoc/sdeu/qpkAccuracy/week}{1}{16}
%  \end{center}
%  {\scriptsize It is possible to see how the average
%  $\left<Q^{Pk}_{\mathrm{VEM}}\right>_{7\mathrm{days}}$ is moving
%  leftward.}
%\end{frame}
%
%
%\begin{frame}
%  \frametitle{Cross Check for moving window: Stations selected
%  randomly and for PMT1}
%  $Q^{pk}_{VEM}$, slope/$\left<\mathrm{FADC}\right>$, $\chi^2$ as
%  function of time
%  \vspace{0.5cm}
%  \begin{columns}
%    \begin{column}{0.5\textwidth}
%      \includegraphics[width=1.\textwidth]{../plots2/crossCheck864UB.pdf}
%    \end{column}
%    \begin{column}{0.5\textwidth}
%      \begin{tikzpicture}
%        \node(a) {
%          \includegraphics[width=1.\textwidth]{../plots2/crossCheck864UUB.pdf}
%        };
%        \node at(a.center)[draw, red, line width=1pt, ellipse,
%        rotate=0, minimum width=15pt, minimum height=10pt,
%        xshift=35pt, yshift=34pt]{};
%      \end{tikzpicture}
%    \end{column}
%  \end{columns}
%  \vspace{0.3cm}
%
%  The moving window is over 7-days in a row. For UUB, the red
%  circle enclose the only 8-days in a row available to fit, i.e.
%  for this UUB stations, only two fits are possible.
%\end{frame}
%
%\begin{frame}
%  \frametitle{Cross Check for moving window: Station selected
%  randomly and for PMT1}
%  $Q^{pk}_{VEM}$, slope/$\left<\mathrm{FADC}\right>$, $\chi^2$ as
%  function of time
%  \vspace{0.5cm}
%  \begin{columns}
%    \begin{column}{0.5\textwidth}
%      \includegraphics[width=1.\textwidth]{../plots2/crossCheck846UB.pdf}
%    \end{column}
%    \begin{column}{0.5\textwidth}
%      \includegraphics[width=1.\textwidth]{../plots2/crossCheck846UUB.pdf}
%    \end{column}
%  \end{columns}
%\end{frame}
%
%\begin{frame}
%  \frametitle{Cross Check for moving window: Station selected
%  randomly and for PMT1}
%  $Q^{pk}_{VEM}$ slope/$\left<\mathrm{FADC}\right>$, $\chi^2$ as
%  function of time
%  \vspace{0.5cm}
%  \begin{columns}
%    \begin{column}{0.5\textwidth}
%      \includegraphics[width=1.\textwidth]{../plots2/crossCheck853UB.pdf}
%    \end{column}
%    \begin{column}{0.5\textwidth}
%      \includegraphics[width=1.\textwidth]{../plots2/crossCheck853UUB.pdf}
%    \end{column}
%  \end{columns}
%\end{frame}
%
%\begin{frame}
%  \frametitle{Cross Check for moving window: Station selected
%  randomly and for PMT1}
%  $Q^{pk}_{VEM}$, slope/$\left<\mathrm{FADC}\right>$, $\chi^2$ as
%  function of time
%  \vspace{0.5cm}
%  \begin{columns}
%    \begin{column}{0.5\textwidth}
%      \includegraphics[width=1.\textwidth]{../plots2/crossCheck1190UB.pdf}
%    \end{column}
%    \begin{column}{0.5\textwidth}
%      \includegraphics[width=1.\textwidth]{../plots2/crossCheck1190UUB.pdf}
%    \end{column}
%  \end{columns}
%\end{frame}
%
%\begin{frame}
%  \frametitle{Cross Check for moving window: Station selected
%  randomly and for PMT1}
%  $Q^{pk}_{VEM}$, slope/$\left<\mathrm{FADC}\right>$, $\chi^2$ as
%  function of time
%  \vspace{0.5cm}
%  \begin{columns}
%    \begin{column}{0.5\textwidth}
%      \includegraphics[width=1.\textwidth]{../plots2/crossCheck1213UB.pdf}
%    \end{column}
%    \begin{column}{0.5\textwidth}
%      \includegraphics[width=1.\textwidth]{../plots2/crossCheck1213UUB.pdf}
%    \end{column}
%  \end{columns}
%\end{frame}
%
%\begin{frame}
%  \frametitle{Cross Check for moving window: Station selected
%  randomly and for PMT1}
%  $Q^{pk}_{VEM}$, slope/$\left<\mathrm{FADC}\right>$, $\chi^2$ as
%  function of time
%  \vspace{0.5cm}
%  \begin{columns}
%    \begin{column}{0.5\textwidth}
%      \includegraphics[width=1.\textwidth]{../plots2/crossCheck1225UB.pdf}
%    \end{column}
%    \begin{column}{0.5\textwidth}
%      \includegraphics[width=1.\textwidth]{../plots2/crossCheck1225UUB.pdf}
%    \end{column}
%  \end{columns}
%\end{frame}
%
%\begin{frame}
%  \frametitle{Moving window algorithm results}
%  \vspace{0.5cm}
%  \begin{columns}
%    \begin{column}{0.5\textwidth}
%      \includegraphics[width=.9\textwidth]{../plots2/chi2VsSlopPmt1Ub.pdf}
%    \end{column}
%    \begin{column}{0.5\textwidth}
%      \includegraphics[width=.9\textwidth]{../plots2/chi2VsSlopPmt1Uub.pdf}
%    \end{column}
%  \end{columns}
%  {\footnotesize First cut: Log10(Pval)>-5.0}
%  \begin{columns}
%    \begin{column}{0.5\textwidth}
%      \includegraphics[width=.9\textwidth]{../plots2/chi2VsSlopPmt1UbProjSlop.pdf}
%    \end{column}
%    \begin{column}{0.5\textwidth}
%      \includegraphics[width=.9\textwidth]{../plots2/chi2VsSlopPmt1UubProjSlop.pdf}
%    \end{column}
%  \end{columns}
%\end{frame}
%
%\begin{frame}
%  \frametitle{Moving window algorithm results}
%  {\footnotesize Second cut: slope between $\mu\pm\sigma$}
%  \vspace{0.2cm}
%  \begin{columns}
%    \begin{column}{0.5\textwidth}
%      \includegraphics[width=.85\textwidth]{../plots2/chi2VsSlopPmt1UbProjChi2.pdf}
%    \end{column}
%    \begin{column}{0.5\textwidth}
%      \includegraphics[width=.85\textwidth]{../plots2/chi2VsSlopPmt1UubProjChi2.pdf}
%    \end{column}
%  \end{columns}
%  {\footnotesize Third cut: fits with $\log10(Pval)$ > $-6.0$.}  
%  \begin{columns}
%    \begin{column}{0.5\textwidth}
%      \includegraphics[width=.85\textwidth]{../plots2/logPvalDistPmt1Ub.pdf}
%        %../plots2/chi2VsSlopPmt1UbProjChi2.pdf}
%    \end{column}
%    \begin{column}{0.5\textwidth}
%      \includegraphics[width=.85\textwidth]{../plots2/logPvalDistPmt1Uub.pdf}
%        %../plots2/chi2VsSlopPmt1UubProjChi2.pdf}
%    \end{column}
%  \end{columns}
%\end{frame}
%
%\begin{frame}
%  \frametitle{Moving window algorithm results: comparison before
%  and after cuts} 
%  {\footnotesize Before cuts:}
%  \vspace{0.2cm}
%  \begin{columns}
%    \begin{column}{0.5\textwidth}
%      \includegraphics[width=.9\textwidth]{../plots2/chi2VsSlopPmt1Ub.pdf}
%    \end{column}
%    \begin{column}{0.5\textwidth}
%      \includegraphics[width=.9\textwidth]{../plots2/chi2VsSlopPmt1Uub.pdf}
%    \end{column}
%  \end{columns}
%  {\footnotesize After cuts:}
%  \begin{columns}
%    \begin{column}{0.5\textwidth}
%      \includegraphics[width=.9\textwidth]{../plots2/chi2VsSlopPmt1UbAfterCuts.pdf}
%    \end{column}
%    \begin{column}{0.5\textwidth}
%      \includegraphics[width=.9\textwidth]{../plots2/chi2VsSlopPmt1UubAfterCuts.pdf}
%    \end{column}
%  \end{columns}
%\end{frame}
%
%\begin{frame}
%  \frametitle{Moving window algorithm results: comparison before
%  and after cuts} 
%  {\footnotesize Before cuts:}
%  \vspace{0.2cm}
%  \begin{columns}
%    \begin{column}{0.5\textwidth}
%      \includegraphics[width=.9\textwidth]{../plots2/chi2VsSlopPmt2Ub.pdf}
%    \end{column}
%    \begin{column}{0.5\textwidth}
%      \includegraphics[width=.9\textwidth]{../plots2/chi2VsSlopPmt2Uub.pdf}
%    \end{column}
%  \end{columns}
%  {\footnotesize After cuts:}
%  \begin{columns}
%    \begin{column}{0.5\textwidth}
%      \includegraphics[width=.9\textwidth]{../plots2/chi2VsSlopPmt2UbAfterCuts.pdf}
%    \end{column}
%    \begin{column}{0.5\textwidth}
%      \includegraphics[width=.9\textwidth]{../plots2/chi2VsSlopPmt2UubAfterCuts.pdf}
%    \end{column}
%  \end{columns}
%\end{frame}
%
%\begin{frame}
%  \frametitle{Moving window algorithm results: comparison before
%  and after cuts} 
%  {\footnotesize Before cuts:}
%  \vspace{0.2cm}
%  \begin{columns}
%    \begin{column}{0.5\textwidth}
%      \includegraphics[width=.9\textwidth]{../plots2/chi2VsSlopPmt3Ub.pdf}
%    \end{column}
%    \begin{column}{0.5\textwidth}
%      \includegraphics[width=.9\textwidth]{../plots2/chi2VsSlopPmt3Uub.pdf}
%    \end{column}
%  \end{columns}
%  {\footnotesize After cuts:}
%  \begin{columns}
%    \begin{column}{0.5\textwidth}
%      \includegraphics[width=.9\textwidth]{../plots2/chi2VsSlopPmt3UbAfterCuts.pdf}
%    \end{column}
%    \begin{column}{0.5\textwidth}
%      \includegraphics[width=.9\textwidth]{../plots2/chi2VsSlopPmt3UubAfterCuts.pdf}
%    \end{column}
%  \end{columns}
%\end{frame}
%
%\begin{frame}
%  \frametitle{How the accuracy is calculating}
%  {\footnotesize
%  \begin{enumerate}
%    \item After the two moving-window algorithm cuts, per
%      station and per PMT, a set of 7-day-series is obtained.
%    \item The best 7-day-series, per PMT for a station, is
%      chosen as the one closest to zero.
%    \item If for the same station, a PMT was not chosen, e.g.
%      in UB version, this one is not taken into the account for
%      UUB, and vice versa.
%    \item<1-> With the chosen PMT, a singular normalized
%      distribution is built for UB and UUB version.
%  \end{enumerate}
%  }
%  
%  \begin{columns}
%    \begin{column}{0.5\textwidth}
%      \includegraphics<2>[width=.6\textwidth]{../plots/filteredPMTsSt1208.pdf}
%    \end{column}
%    \begin{column}{0.5\textwidth}
%      \includegraphics<2>[width=.6\textwidth]{../plots/filteredUbPMTsSt1208.pdf}
%    \end{column}
%  \end{columns}
%  \begin{columns}
%    \begin{column}{0.5\textwidth}
%      \includegraphics<2>[width=.6\textwidth]{../plots/filteredSt1208.pdf}
%    \end{column}
%    \begin{column}{0.5\textwidth}
%      \includegraphics<2>[width=.6\textwidth]{../plots/filteredUbSt1208.pdf}
%    \end{column}
%  \end{columns}
%\end{frame}
%
%
%\begin{frame}
%  \frametitle{How the accuracy is calculating}
%  {\footnotesize
%  \begin{enumerate}
%    \item After the two moving-window algorithm cuts, per
%      station and per PMT, a set of 7-day-series is obtained.
%    \item The best 7-day-series, per PMT for a station, is
%      chosen as the one closest to zero.
%    \item If for the same station, a PMT was not chosen, e.g.
%      in UB version, this one is not taken into the account for
%      UUB, and vice versa.
%    \item With the chosen PMT, a singular normalized
%      distribution is built for UB and UUB version.
%    \item A Gaussian function is fitted to the normalized
%      distribution and then the accuracy is calculated as:
%      $\sigma/\mu$, respectively.
%  \end{enumerate}
%  }
%\end{frame}
%
%
%\begin{frame}
%  \frametitle{Accuracy results}
%  \vspace{0.5cm}
%  \begin{columns}
%    \begin{column}{0.5\textwidth}
%      \includegraphics[width=.9\textwidth]{../plots2/accQpkFitUbUubAllStAllPmt_Stats.pdf}
%    \end{column}
%    \begin{column}{0.5\textwidth}
%      \includegraphics[width=.9\textwidth]{../plots2/accQpkFitUbUubDistDiff_Stats.pdf}
%    \end{column}
%  \end{columns}
%  \begin{columns}
%    \begin{column}{0.5\textwidth}
%      \includegraphics[width=.9\textwidth]{../plots2/accQpkFitUbUubPerSt_Stats.pdf}
%    \end{column}
%    \begin{column}{0.5\textwidth}
%      \includegraphics[width=.9\textwidth]{../plots2/accQpkFitUbUubDiffPerSt_Stats.pdf}
%    \end{column}
%  \end{columns}
%  {\scriptsize
%  For accuracy distribution, only $\sigma/\mu$ < $5.0$\,\% considered.
%  For difference, $(\sigma/\mu)_{\mathrm{UB}} - (\sigma/\mu)_{\mathrm{UUB}}$
%  > $-2.0$\,\%.
%  }
%\end{frame}
%
%\begin{frame}
%  \frametitle{Outliers: $(\sigma/\mu)_{\mathrm{UB}} -
%  (\sigma/\mu)_{\mathrm{UUB}}$ > $-2.0$\,\% }
%  \vspace{0.5cm}
%  \begin{columns}
%    \begin{column}{0.5\textwidth}
%      \includegraphics[width=.9\textwidth]{../plots2/accDistSt864ub.pdf}
%    \end{column}
%    \begin{column}{0.5\textwidth}
%      \includegraphics[width=.9\textwidth]{../plots2/accDistSt864uub.pdf}
%    \end{column}
%  \end{columns}
%  \begin{columns}
%    \begin{column}{0.5\textwidth}
%      \includegraphics[width=.9\textwidth]{../plots/qpksVsTimeSt864UUB.pdf}
%    \end{column}
%    \begin{column}{0.5\textwidth}
%      \includegraphics[width=.9\textwidth]{../plots2/st864pmt1_week1.png}
%    \end{column}
%  \end{columns}
%\end{frame}
%
%\begin{frame}
%  \frametitle{Outliers: $(\sigma/\mu)_{\mathrm{UB}} -
%  (\sigma/\mu)_{\mathrm{UUB}}$ > $-2.0$\,\% }
%  \vspace{0.5cm}
%  \begin{columns}
%    \begin{column}{0.5\textwidth}
%      \includegraphics[width=.9\textwidth]{../plots2/accDistSt1205ub.pdf}
%    \end{column}
%    \begin{column}{0.5\textwidth}
%      \includegraphics[width=.9\textwidth]{../plots2/accDistSt1205uub.pdf}
%    \end{column}
%  \end{columns}
%  \begin{columns}
%    \begin{column}{0.33\textwidth}
%      \includegraphics[width=.9\textwidth]{../plots/qpksVsTimeSt1205UUB.pdf}
%    \end{column}
%    \begin{column}{0.33\textwidth}
%      \includegraphics[width=.9\textwidth]{../plots2/st1205pmt1_week25.png}
%    \end{column}
%    \begin{column}{0.33\textwidth}
%      \includegraphics[width=.9\textwidth]{../plots2/st1205pmt3_week25.png}
%    \end{column}
%  \end{columns}
%\end{frame}
%
%
%\begin{frame}
%  \frametitle{Outliers: $(\sigma/\mu)_{\mathrm{UB}} -
%  (\sigma/\mu)_{\mathrm{UUB}}$ > $-2.0$\,\% }
%  \vspace{0.5cm}
%  \begin{columns}
%    \begin{column}{0.5\textwidth}
%      \includegraphics[width=.9\textwidth]{../plots2/accDistSt1216ub.pdf}
%    \end{column}
%    \begin{column}{0.5\textwidth}
%      \includegraphics[width=.9\textwidth]{../plots2/accDistSt1216uub.pdf}
%    \end{column}
%  \end{columns}
%  \begin{columns}
%    \begin{column}{0.5\textwidth}
%      \includegraphics[width=.9\textwidth]{../plots/qpksVsTimeSt1216UB.pdf}
%    \end{column}
%    \begin{column}{0.5\textwidth}
%      \includegraphics[width=.9\textwidth]{../plots2/st1216pmt2_week5.png}
%    \end{column}
%  \end{columns}
%\end{frame}

%\begin{frame}
%  \frametitle{Outliers: $(\sigma/\mu)_{\mathrm{UB}} -
%  (\sigma/\mu)_{\mathrm{UUB}}$ > $-2.0$\,\% }
%  \vspace{0.5cm}
%  \begin{columns}
%    \begin{column}{0.5\textwidth}
%      \includegraphics[width=.9\textwidth]{../plots2/accDistSt1740ub.pdf}
%    \end{column}
%    \begin{column}{0.5\textwidth}
%      \includegraphics[width=.9\textwidth]{../plots2/accDistSt1740uub.pdf}
%    \end{column}
%  \end{columns}
%  \begin{columns}
%    \begin{column}{0.5\textwidth}
%      \includegraphics[width=.9\textwidth]{../plots2/accDistSt1745ub.pdf}
%    \end{column}
%    \begin{column}{0.5\textwidth}
%      \includegraphics[width=.9\textwidth]{../plots2/accDistSt1745uub.pdf}
%    \end{column}
%  \end{columns}
%\end{frame}

\end{document}
