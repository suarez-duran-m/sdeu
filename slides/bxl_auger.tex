\documentclass[aspectratio=169]{beamer}
\usepackage[T1]{fontenc}
\usepackage[utf8]{inputenc}
\usepackage{tikz}
\usepackage{tabularx}
\usepackage[font=scriptsize]{caption}
\usepackage{multirow}
\captionsetup[figure]{labelformat=empty}

\usetikzlibrary{tikzmark,shapes,arrows,backgrounds,fit,positioning}
\newcolumntype{C}{>{\centering\arraybackslash}X}
\newcolumntype{R}{>{\raggedleft\arraybackslash}X}

\addtobeamertemplate{navigation symbols}{}
{
	\insertframenumber{}
}

\beamertemplatenavigationsymbolsempty
\setbeamercolor{section in foot}{fg=white, bg=blue}
\setbeamercolor{subsection in foot}{fg=black, bg=white}
\setbeamerfont{footline}{size=\fontsize{6}{6}\selectfont}

\setbeamertemplate{footline}
{
  \leavevmode
  \hbox
  {
    \begin{beamercolorbox}
      [wd=.5\paperwidth,ht=2.5ex,dp=1.125ex,leftskip=.3cm,rightskip=.3cm]{subsection in foot}
    \end{beamercolorbox}
    
    \begin{beamercolorbox}
      [wd=.5\paperwidth,ht=2.5ex,dp=1.125ex,leftskip=.3cm,rightskip=.3cm plus1fil]{subsection in foot}
      \hfill
      \insertframenumber
      %\insertframenumber\,/\,\inserttotalframenumber
    \end{beamercolorbox}
  }
}


\title{Calibration of the SD Array of the Pierre Auger Observatory}
\author{
  Mauricio Su\'arez Dur\'an
}
\institute{IIHE-ULB}

\titlegraphic{
  \begin{figure}[h]
    \centering
   %\includegraphics[width=5cm]{ulbLogo2.png}
    \hspace*{8.cm}
    \includegraphics[width=5.5cm]{iihe.jpeg}
  \end{figure}
}

\begin{document}
\begin{frame}
  \titlepage
\end{frame}

\begin{frame}
  \frametitle{The SD array}
  \centering
    \includegraphics[width=.4\textwidth]{/home/msd/Pictures/pierre_auger_sd-array.jpg}
    \begin{itemize}
      \item<1-> $1600$ WCD, each one with $3$ PMTs
      \item<2-> The Cherenkov light is measured in units of the
        signal produced by a vertical and central through-going
        (VCT) $\rightarrow$ termed a vertical-equivalent muon
        (VEM).
    \end{itemize}
\end{frame}

\begin{frame}
  \frametitle{Technical limitations}
  \begin{itemize}
      \item<1-> The bandwidth available from each SD to the CDAS
        is approximately 1200 bits per second.
      \item<2-> The local processor is an 80 MHz PowerPC 403GCX
        lacking floating-point hardware.
      \item<3-> The remoteness of the detectors.
  \end{itemize}
\end{frame}

\begin{frame}
  \frametitle{Important technical aspects}
  \begin{itemize}
    \item<1-> Digitisation at 40 MHz and 10-bit flash
      analog-to-digital converters (FADCs) to digitize the
      signals from the 3 PMTs.
    \item<2-> Two signals from each PMT one from the anode, and
      the other from the last dynode, amplified and inverted to a
      total to nominally 32 times the anode.
    \item<3-> The signal recorded by the FADC is referred to in
      units of channels (ch), with a range of $0-1023$,
      corresponding to an input range of $0-2$\,V. Each FADC bib
      corresponds to $25$\,ns.
  \end{itemize}
% \centering
%  \includegraphics[width=.5\textwidth]{/home/msd/Pictures/auger_observatory.png}
\end{frame}


\begin{frame}
  \begin{columns}
    \begin{column}{.9\textwidth}
      \includegraphics[width=1.\textwidth]{/home/msd/Pictures/table_calib_terms.png}
    \end{column}
  \end{columns}
\end{frame}

\begin{frame}
  \frametitle{Charge histograms and VEM}
  Atmospheric muons passing through the detector at
  $\sim2500$\,Hz, but the SD (in normal configuration) has no way
  to select only VCT muons.
  \begin{columns}
    \begin{column}{.35\textwidth}
      \includegraphics<2->[width=1.\textwidth]{/home/msd/Pictures/charge_vem_3-fold.png}
    \end{column}
  \end{columns}
  This peak ($Q^{\mathrm{peak}}_{\mathrm{VEM}}$) is at
  approximately $1.09$ VEM for the sum of the 3 PMTs and
  $1.03\pm0.02$) VEM for each PMT, both measured with a muon
  telescope providing the trigger in a reference tank.
\end{frame}

\begin{frame}
  \frametitle{Height histograms and trigger}
  The SD uses two triggers to identify showe candidates
  \begin{enumerate}
    \item<1-> A simple threshold trigger, satisfied when the
      signal from all 3 PMTs exceeds a set threshold (designed
      for signals close to the shower core).
    \item<2-> A time-over-threshold trigger, which requires that
      the signal from 2 of the 3 PMTs exceed a much lower
      threshold than the first one for a number of time bins
      within a given time window (designed for signals far from
      the shower core).
  \end{enumerate}
  \begin{itemize}
    \item<3-> These triggers are set in electronics units
      (channels) -a measure of the current from the PMT- so the
      station must have a reference unit for current as well
      $\rightarrow$ $I^{\mathrm{peak}}_{\mathrm{VEM}}$
  \end{itemize}
\end{frame}

\begin{frame}
  \frametitle{Height histograms and trigger}
  \begin{columns}
    \begin{column}{.4\textwidth}
      \includegraphics<1->[width=1.\textwidth]{/home/msd/Pictures/amplitude_vem_3-fold.png}
    \end{column}
  \end{columns}
  \begin{itemize}
    \item<2->This peak, like the charge histogram peak, is
      related to the peak current produced by a vertical
      through-going muom $I^{\mathrm{peak}}_{\mathrm{VEM}}$.
    \item<3-> The target trigger threshold is
      $3.2I^{\mathrm{peak}}_{\mathrm{VEM}}$ for the simple
      threshold trigger, and
      $0.2I^{\mathrm{peak}}_{\mathrm{VEM}}$ for the
      time-over-threshold trigger.
  \end{itemize}
\end{frame}

\begin{frame}
  \frametitle{VEM calibration procedure}
  \begin{enumerate}
    \item<2-> Set up the end-to-end gains to have
      $I^{\mathrm{peak}}_{\mathrm{VEM}}$ at $50$\,ch.
    \item<3-> Continually perform a local calibration to
      determine $I^{\mathrm{peak}}_{\mathrm{VEM}}$ in channels to
      adjust the electronics-level trigger.
    \item<4-> Determine the value of $Q^{peak}_{VEM}$ to high
      accuracy using charge histograms, and use the known
      conversion from $Q^{\mathrm{peak}}_{\mathrm{VEM}}$ to $1$\,VEM
  \end{enumerate}
\end{frame}

\begin{frame}
  \frametitle{1. End-to-end gain setup}
  \begin{enumerate}
    \item<1-> The end-to-end gains (i.e.
      $I^{\mathrm{peak}}_{\mathrm{VEM}}$ in electronics units) of
      each of the 3 PMTs are set up by matching a point in the
      spectrum to a measured rate from a reference tank.
    \item<2-> The reference tank is calibrated by obtaining a
      charge histogram and adjusting the PMT high voltage until
      $Q^{\mathrm{peak}}_{\mathrm{VEM}}$ of the three histograms
      agree.
    \item<3-> The singles rate of a PMT at 150 ch above baseline
      was required to be $100$\,Hz, which corresponds to a
      trigger point of roughly
      $3.0Q^{\mathrm{peak}}_{\mathrm{VEM}}$. This choice sets up
      each of the PMTs to have approximately
      $50$\,ch/$I^{\mathrm{peak}}_{\mathrm{VEM}}$.
  \end{enumerate}
  \begin{itemize}
    \item<4-> When the local station electronics is first turned
      on, each PMT high voltage is adjusting until the rate is
      $100$\,Hz at a point $150$\,ch above baseline.
  \end{itemize}
\end{frame}

\begin{frame}
  \frametitle{1. End-to-end gain setup}
  The end-to-end gain measurement implies that the PMTs in the
  SDs will not have equivalent gains-indeed, even PMTs within the
  same tank may not have equivalent gains.
  \vspace{0.5cm}
  \begin{columns}
    \begin{column}{.5\textwidth}
      \includegraphics[width=1.\textwidth]{/home/msd/Pictures/photoelectrons_dynodoVsGain.png}
    \end{column}
  \end{columns}
  The choice of $50$\,ch $I^{\mathrm{peak}}_{\mathrm{VEM}}$
  results in a mean gain of approximately $3.4\times10^5$ for a
  mean $n_{\mathrm{pe}}$/VEM$\sim94$\,pe.
\end{frame}

\begin{frame}
  \frametitle{Continual on-line calibration}
  Once the gains of the 3 PMTs are set up, the drifts of the
  value of $I^{\mathrm{peak}}_{\mathrm{VEM}}$ in electronics
  units for each detector must be compensated to ensure that the
  surface array triggers uniformly.
  \vspace{0.75cm}
  \begin{itemize}
    \item<2-> The PMT high voltage is not changed during normal operation
    \item<3-> For normal operation, this non-uniformity is
      minimal $\sim10$\,\%.
    \item<4-> Detectors which have drifted significantly
      ($>20$\,ch) from the nominal
      $I^{\mathrm{peak}}_{\mathrm{VEM}}$ of $50$\,ch are
      re-initialized.
    \item<5-> The average value of
      $I^{\mathrm{peak}}_{\mathrm{VEM}}$ for the PMTs of the SD
      is currently $46\pm4$\,ch.
  \end{itemize}
\end{frame}

\begin{frame}
  \frametitle{Continual on-line calibration}
  \begin{itemize}
    \item<1-> The value of $I^{\mathrm{peak}}_{\mathrm{VEM}}$ as
      defined in Section 2.2 is not obtained on-line since this
      would increase the dead time of the detector to
      unacceptable levels.
      \includegraphics<2>[width=.4\textwidth]{/home/msd/Pictures/amplitude_vem_3-fold.png}
    \item<3-> Instead, the trigger levels are set with respect to
      an estimate of $I^{\mathrm{peak}}_{\mathrm{VEM}}$
      ($I^{\mathrm{est.}}_{\mathrm{VEM}}$), which is defined for
      a given PMT by requiring that the rate of events satisfying
      a ``calibration trigger'' be $70$\,Hz.
    \item<4-> An event satisfies the calibration trigger if the
      signal is above $2.5I^{\mathrm{est.}}_{\mathrm{VEM}}$ for
      the given PMT and above
      $1.75I^{\mathrm{est.}}_{\mathrm{VEM}}$ for all three PMTs.
  \end{itemize}
\end{frame}

\begin{frame}
  \frametitle{How to obtain $I^{\mathrm{est.}}_{\mathrm{VEM}}$}
  $\sigma$-$\delta$ convergence algorithm
  \begin{enumerate}
    \item<1-> Start with a value of
      $I^{\mathrm{est.}}_{\mathrm{VEM}}=50$\,ch.
    \item<2-> Measure, for each PMT, the rate of events
      satisfying the calibration trigger by counting these events
      for a time $t_\mathrm{cal}$, initially set to $5$\,s.
    \item<3-> If, for a given PMT, the rate is above
      $70+\sigma$\,Hz, increase
      $I^{\mathrm{est.}}_{\mathrm{VEM}}$ by $\delta$. Likewise,
      if the rate is below $70-\sigma$\,Hz, decrease
      $I^{\mathrm{est.}}_{\mathrm{VEM}}$ by $\delta$, with
      $\sigma=2$\,Hz and $\delta=1$\,ch initially.
    \item<4-> If the rate of any single PMT is more than
      $10\sigma$ away from $70$\,Hz, adjust
      $I^{\mathrm{est.}}_{\mathrm{VEM}}$ by $5$\,ch in the
      appropriate direction, set $t_\mathrm{cal}$ to $10$\,s,
      $\sigma=1$\,ch, and repeat from step 2.
    \item<5-> Otherwise, if $t_\mathrm{cal}<60$\,s, increase
      $t_\mathrm{cal}$ by $5$\,s. If $\delta>0.1$\,ch, decrease 
      $\delta$ by $0.1$\,ch, and repeat from step (2).
  \end{enumerate}
\end{frame}


\begin{frame}
  \frametitle{Results before and after the convergence algorithm}
  \begin{columns}
    \begin{column}{.42\textwidth}
      \includegraphics[width=1.\textwidth]{/home/msd/Pictures/rate_time_convergence.png}
    \end{column}
  \end{columns}
  \begin{itemize}
    \item<2-> The rapid convergence to $20$\,Hz shows that the
      method described enables the uniform trigger levels to be
      established rapidly.
    \item<3-> A comparison of the converged
      $I^{\mathrm{est.}}_{\mathrm{VEM}}$ value with values
      obtained from a pulse height histogram gives
      $I^{\mathrm{est.}}_{\mathrm{VEM}}=(0.94\pm0.06)I^{\mathrm{peak}}_{\mathrm{VEM}}$.
  \end{itemize}
\end{frame}

\begin{frame}
  \frametitle{Continual on-line calibration} 
  $\sigma$-$\delta$ convergence algorithm for $Q^{\mathrm{peak}}_{\mathrm{VEM}}$
  \begin{itemize}
    \item<2-> The on-line calibration also estimates
      $Q^{\mathrm{peak}}_{\mathrm{VEM}}$ as well
      ($Q^{\mathrm{est.}}_{\mathrm{VEM}}$), by computing the
      charge of pulses with a peak of exactly
      $I^{\mathrm{est.}}_{\mathrm{VEM}}$, determined initially
      from the charge of the first pulse.
    \item<3-> A comparison of the converged
      $Q^{\mathrm{est.}}_{\mathrm{VEM}}$ and
      $Q^{\mathrm{peak}}_{\mathrm{VEM}}$ determined from a peak
      fit to the charge histograms yields
      $Q^{\mathrm{est.}}_{\mathrm{VEM}}=(0.96\pm0.03)Q^{\mathrm{peak}}_{\mathrm{VEM}}$.
    \item<4-> The history over the last $7t_\mathrm{cal}$
      ($60$\,s) periods of the adjustments to
      $I^{\mathrm{est.}}_{\mathrm{VEM}}$ is included with each
      event, along with $I^{\mathrm{est.}}_{\mathrm{VEM}}$,
      $Q^{\mathrm{est.}}_{\mathrm{VEM}}$, and the last $70$\,Hz
      rates for each of the 3 PMTs.
  \end{itemize}
\end{frame}


\begin{frame}
  \frametitle{Pressure dependence of the on-line calibration}
  \begin{columns}
    \begin{column}{.6\textwidth}
      \includegraphics[width=1.\textwidth]{/home/msd/Pictures/chargeVsPressure.png}
    \end{column}
  \end{columns}
  Correlation of the ratio
  $Q^{\mathrm{est.}}_{\mathrm{VEM}}$/$Q^{\mathrm{peak}}_{\mathrm{VEM}}$
  to atmospheric pressure as measured by a weather station
  located at the Los Leones fluorescence site. Note
  $1$\,hPa $ =1.020$\,g\,cm$^{-2}$ in atmospheric depth. The
  effect $\sim0.1$\,\%\,g$^{-1}$\,cm$^2$ on the trigger level
  over the  course of a year is approximately $3$\.\%.
\end{frame}

\begin{frame}
  \frametitle{Histograms created every minute}
  \begin{itemize}
    \item<1-> Charge histograms for each individual PMT.
    \item<2-> Charge histogram for the sum of all 3 PMTs.
    \item<3-> Pulse height histograms for each individual PMT.
    \item<4-> Histograms of the baseline of each FADC channel.
  \end{itemize}
\end{frame}


\begin{frame}
  \frametitle{Calibration data}
  \begin{columns}
    \begin{column}{.8\textwidth}
      \includegraphics[width=1.\textwidth]{/home/msd/Pictures/calib_histos_bl-height.png}
    \end{column}
  \end{columns}
  Example calibration data which are sent back with candidate
  event data, built from approximately 150 000 pulses collected
  in the minute prior to the event.
\end{frame}

\begin{frame}
  \frametitle{Calibration data}
  \begin{columns}
    \begin{column}{.6\textwidth}
      \includegraphics[width=1.\textwidth]{/home/msd/Pictures/calib_charge-shape-summed.png}
    \end{column}
  \end{columns}
  Example calibration data which are sent back with candidate
  event data, built from approximately 150 000 pulses collected
  in the minute prior to the event. The average of all pulse
  shapes with an integrated charge of $1.0\pm0.1$
  $Q^{\mathrm{est.}}_{\mathrm{VEM}}$.
\end{frame}

\begin{frame}
  \frametitle{Dynode/anode ratio (D/A)}
  \begin{itemize}
    \item<1-> The nominal gain between the dynode and the anode
      is 32-that is, 5 bits of overlap out of a 10 bit FADC,
      giving 15 total bits of dynamic range.
    \item<2-> the best measurement of D/A would occur simply by
      taking the peak of the dynode divided by the peak of the
      anode, and averaging over many samples.
    \item<3-> But,the dynode is amplified by two Analog Devices
      AD8012 amplifier stages, each of which has a phase delay of
      approximately $2-3$\,ns.
    \item<4-> So, D=A is therefore measured by modelling the
      anode signal shape (A) from the dynode (D) as
  \end{itemize}
  \centering
   $ A(t) = \frac{1}{R} ( (1-\epsilon)D(t) + \epsilon D( t+1 ) ) $
\end{frame}

\end{document}
