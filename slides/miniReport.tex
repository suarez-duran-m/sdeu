\documentclass[aspectratio=169]{beamer}
\usepackage[T1]{fontenc}
\usepackage[utf8]{inputenc}
\usepackage{tikz}
\usepackage{tabularx}
\usepackage[font=scriptsize]{caption}
\usepackage{multirow}
\captionsetup[figure]{labelformat=empty}

\usetikzlibrary{tikzmark,shapes,arrows,backgrounds,fit,positioning}
\newcolumntype{C}{>{\centering\arraybackslash}X}
\newcolumntype{R}{>{\raggedleft\arraybackslash}X}

\addtobeamertemplate{navigation symbols}{}
{
	\insertframenumber{}
}

\beamertemplatenavigationsymbolsempty
\setbeamercolor{section in foot}{fg=white, bg=blue}
\setbeamercolor{subsection in foot}{fg=black, bg=white}
\setbeamerfont{footline}{size=\fontsize{6}{6}\selectfont}

\setbeamertemplate{footline}
{
  \leavevmode
  \hbox
  {
    \begin{beamercolorbox}
      [wd=.5\paperwidth,ht=2.5ex,dp=1.125ex,leftskip=.3cm,rightskip=.3cm]{subsection in foot}
    \end{beamercolorbox}
    
    \begin{beamercolorbox}
      [wd=.5\paperwidth,ht=2.5ex,dp=1.125ex,leftskip=.3cm,rightskip=.3cm plus1fil]{subsection in foot}
      \hfill
      \insertframenumber
      %\insertframenumber\,/\,\inserttotalframenumber
    \end{beamercolorbox}
  }
}


\title{UUB Charge and Peak histograms}
\author{
  Mauricio Su\'arez Dur\'an and Ioana~C.~Mari\c{s}
}
\institute{IIHE-ULB}

\titlegraphic{
  \begin{figure}[h]
    \centering
   %\includegraphics[width=5cm]{ulbLogo2.png}
    \hspace*{8.cm}
    \includegraphics[width=5.5cm]{iihe.jpeg}
  \end{figure}
}

\begin{document}
\begin{frame}
  \titlepage
\end{frame}

\begin{frame}
  \frametitle{Getting VEM values in UUB} %: Peak histogram Station863}
  \begin{figure}
    \centering
    \begin{tabularx}{\textwidth}{CC}
      \begin{tabular}{l}
        \includegraphics[width=.5\textwidth]{../plots/chargeHisto863.pdf}
      \end{tabular}
      &
      \begin{tabular}{l}
        \includegraphics[width=.5\textwidth]{../plots/chargeDerHisto863.pdf}
      \end{tabular}
      \\
      \footnotesize
      \begin{tabular}{l}
        The BXL-method: \\
        1. Smoothing the histogram \\
        2. Deriving the smoothed-histogram \\
        3. Smoothing the smoothed-derivative-histogram, i.e. the
        histogram getting in previous step \\
        4. Searching for the VEM-value, i.e. first bin for
        derivative equal zero; from right to left. \\
      \end{tabular}
    \end{tabularx}
  \end{figure}
\end{frame}

\begin{frame} 
  \frametitle{Comparison with second order polynomial fitting}
  \begin{figure}
    \centering
    \begin{tabularx}{\textwidth}{CC}
      \begin{tabular}{l}
        \includegraphics[width=0.5\textwidth]{../plots/chargeFittedHisto863.pdf}
      \end{tabular}
      &
      \begin{tabular}{l}
        \includegraphics[width=0.5\textwidth]{../plots/chargeFitResiduals863.pdf}
      \end{tabular}
    \end{tabularx}
  \end{figure}
  Comparison between the VEM value obtained for a charge
  histogram by fitting a second order polynomial (red line), and
  the same one by BXL-method. The red vertical dashed line
  correspond to VEM value from the polynomial fit
  ($1249.13$\,FADC), and the green one from BXL-method
  ($1244.00$\,FADC).
\end{frame}


\begin{frame} 
  \frametitle{Comparison with OffLine}
  {\bf 863 Station}
  \begin{figure}
    \centering
    \begin{tabularx}{\textwidth}{CC}
      \begin{tabular}{l}
        \includegraphics[width=0.5\textwidth]{../plots/uubChargeFromDerSt863pmt1.pdf}
      \end{tabular}
      &
      \begin{tabular}{l}
        \includegraphics[width=0.5\textwidth]{../plots/uubChargeFromOffSt863pmt1.pdf}
      \end{tabular}
    \end{tabularx}
  \end{figure}
  Here, Charge-Fit correspond to second order polynomial fit,
  Charge-Der. is the value from BXL-method, and Charge-OffLine is
  the VEM obtained from OffLine. In average, the relative
  difference for BXL-method, respect of OffLine, is $\sim
  6.11$\,\%. In the right plot, wrong fitting from OffLine can be
  seen (values between $200$\,FADC and $400$\,FADC). For the
  month of July, there are not histograms to fit.
\end{frame}

\begin{frame} 
  \frametitle{Comparison with OffLine}
  {\bf 1740 Station}
  \begin{figure}
    \centering
    \begin{tabularx}{\textwidth}{CC}
      \begin{tabular}{l}
        \includegraphics[width=0.5\textwidth]{../plots/uubChargeFromDerSt1740pmt2.pdf}
      \end{tabular}
      &
      \begin{tabular}{l}
        \includegraphics[width=0.5\textwidth]{../plots/uubChargeFromOffSt1740pmt2.pdf}
      \end{tabular}
    \end{tabularx}
  \end{figure}
  Another station, another PMT. Charge-Fit correspond to second
  order polynomial fit, Charge-Der. is the value from BXL-method,
  and Charge-OffLine is the VEM obtained from OffLine. In
  average, the relative difference for BXL-method, respect of
  OffLine, is $\sim 0.83$\,\%. For the month of July, there are
  not histograms to fit.
\end{frame}

\begin{frame} 
  \frametitle{Comparison with OffLine}
  {\bf 1217 Station}
  \begin{figure}
    \centering
    \begin{tabularx}{\textwidth}{CC}
      \begin{tabular}{l}
        \includegraphics[width=0.5\textwidth]{../plots/uubChargeFromDerSt1217pmt3.pdf}
      \end{tabular}
      &
      \begin{tabular}{l}
        \includegraphics[width=0.5\textwidth]{../plots/uubChargeFromOffSt1217pmt3.pdf}
      \end{tabular}
    \end{tabularx}
  \end{figure}
  Another station, another PMT. Charge-Fit correspond to second
  order polynomial fit, Charge-Der. is the value from BXL-method,
  and Charge-OffLine is the VEM obtained from OffLine. In
  average, the relative difference for BXL-method, respect of
  OffLine, is $\sim 0.21$\,\%. In this case, there are histograms
  to fit for the month of July.
\end{frame}


\begin{frame} 
  \frametitle{Some examples for failed fitting for Peak
  histograms}
  \begin{figure}
    \centering
    \begin{tabularx}{\textwidth}{CC}
      \begin{tabular}{l}
        \includegraphics[width=0.46\textwidth]{../plots/samplePkHistoDerVem163Pmt3.pdf}
      \end{tabular}
      &
      \begin{tabular}{l}
        \includegraphics[width=0.46\textwidth]{../plots/peakDerHisto863Evt62175266.pdf}
      \end{tabular}
      \\
      \begin{tabular}{l}
        \includegraphics[width=0.46\textwidth]{../plots/samplePkHistoDerVem406Pmt2.pdf}
      \end{tabular}
      &
      \footnotesize
      \begin{tabular}{l}
        Top left, it can be seen how the two methods, \\
        fitting 2nd order polynomial and BXL-method, \\
        fail getting a VEM-Value. This could be explain \\
        checking the derivative plot (top right), where \\
        there is not a maximum after the EM peak. \\
        Bottom left, an example of a failed fitting but \\
        successfully BXL-method.
      \end{tabular}
    \end{tabularx}
  \end{figure}
\end{frame}


\end{document}
