\documentclass[aspectratio=169]{beamer}
\usepackage[T1]{fontenc}
\usepackage[utf8]{inputenc}
\usepackage{tikz}
\usepackage{tabularx}
\usepackage[font=scriptsize]{caption}
\usepackage{multirow}
\captionsetup[figure]{labelformat=empty}

\usetikzlibrary{tikzmark,shapes,arrows,backgrounds,fit,positioning}
\newcolumntype{C}{>{\centering\arraybackslash}X}
\newcolumntype{R}{>{\raggedleft\arraybackslash}X}

\addtobeamertemplate{navigation symbols}{}
{
	\insertframenumber{}
}

\beamertemplatenavigationsymbolsempty
\setbeamercolor{section in foot}{fg=white, bg=blue}
\setbeamercolor{subsection in foot}{fg=black, bg=white}
\setbeamerfont{footline}{size=\fontsize{6}{6}\selectfont}

\setbeamertemplate{footline}
{
  \leavevmode
  \hbox
  {
    \begin{beamercolorbox}
      [wd=.5\paperwidth,ht=2.5ex,dp=1.125ex,leftskip=.3cm,rightskip=.3cm]{subsection in foot}
    \end{beamercolorbox}
    
    \begin{beamercolorbox}
      [wd=.5\paperwidth,ht=2.5ex,dp=1.125ex,leftskip=.3cm,rightskip=.3cm plus1fil]{subsection in foot}
      \hfill
      \insertframenumber
      %\insertframenumber\,/\,\inserttotalframenumber
    \end{beamercolorbox}
  }
}


\title{Comparison of $\mathrm{Q}^{\mathrm{pk}}_{\mathrm{VEM}}$
values getting from SdCalibrator and a local fit method, for
UB and UUB}
\author{
  Mauricio Su\'arez Dur\'an and Ioana~C.~Mari\c{s}
}
\institute{IIHE-ULB}

\titlegraphic{
  \begin{figure}[h]
    \centering
   %\includegraphics[width=5cm]{ulbLogo2.png}
    \hspace*{8.cm}
    \includegraphics[width=5.5cm]{iihe.jpeg}
  \end{figure}
}

\begin{document}
\begin{frame}
  \titlepage
\end{frame}

\begin{frame}
  \frametitle{The current algorithm (OffLine SdCalibrator Module)}
  UUB Charge histogram
  \begin{center}
    \includegraphics<1>[width=.65\textwidth]{../plots/offlineChCompaSt863Pmt1.pdf}
  \end{center}
  \begin{columns}
    \begin{column}{0.5\textwidth}
      \includegraphics<2->[width=1.\textwidth]{../plots/uubChargeFromOffSt863pmt1.pdf}
    \end{column}
    \begin{column}{0.5\textwidth}
      \includegraphics<3->[width=1.\textwidth]{../plots/offlineFailedChargeSt863PMT1Evt61435819.pdf}
    \end{column}
  \end{columns}
    Find "head-and-shoulder": from the right side of the histogram,
    search for local maximum (head), surrounded by drops (shoulders)
    with shoulder/head value ratio less than
    fChargeWindowShoulderHeadRatio (as default 0.75)
\end{frame}

% Presentation of BXL-Method
\begin{frame}
  \frametitle{Obtaining initial parameters through derivatives} %: Peak histogram Station863}
  UUB Charge histogram
  \begin{columns}
    \begin{column}{0.55\textwidth}
      \includegraphics<1->[width=1.\textwidth]{../plots/chargeHisto863.pdf}
    \end{column}
    \begin{column}{0.55\textwidth}
      \includegraphics<2->[width=1.\textwidth]{../plots/chargeDerHisto863.pdf}
    \end{column}
  \end{columns}

  \begin{enumerate}
    \item<1-> Smoothing the histogram by 15-bin sliding
      window, $H_S$\\
    \item<2-> Obtain first derivative of the $H_S$
      ($\frac{f(x+1)-f(x-1)}{2h}$), $H_{DS}$ (black line)\\
    \item<3-> Smoothing $H_{DS}$, obtaining $H_{SDS}$ (red line) \\ 
    \item<4-> Searching for the VEM hump, i.e. first bin for
      $H_{SDS}$ equal to zero; from right to left. \\
    \item <5-> Fixing the fitting range using n-bin leftward
      and n-bin rightward from VEM hump.
  \end{enumerate}
\end{frame}

% An example for BXL-Method
\begin{frame} 
  \frametitle{Describing the muon hump with a second polynomial}
  UUB Charge histogram
  \begin{columns}
    \begin{column}{0.55\textwidth}
      \includegraphics[width=1.\textwidth]{../plots/chargeFitPoly2863.pdf}
    \end{column}
    \begin{column}{0.55\textwidth}
      \includegraphics[width=1.\textwidth]{../plots/chargeFitResidualsPoly2863.pdf}
    \end{column}
  \end{columns}
\end{frame}

% Improving BXL-Method
\begin{frame} 
  \frametitle{Choosing the number of bins}
  \begin{enumerate}
    \item Using the hump value from the derivative as initial parameter
    \item Fixing the number of bins to the left and right
      with an extra condition of not reaching the valley
    \item Checking the spread of the VEM values versus number
      of used n-bin
  \end{enumerate}
  \begin{center}
    \begin{columns}
      \begin{column}{0.55\textwidth}
        \includegraphics[width=1.\textwidth]{../plots/uubChRmsFitBinsLrSt863.pdf}
      \end{column}
      \begin{column}{0.55\textwidth}
        \includegraphics[width=1.\textwidth]{../plots/uubChRmsFitBinsLrSt863zoom.pdf}
      \end{column}
    \end{columns}
  \end{center}
     \hfill $\Rightarrow$ A number of about 35 bins is sufficient 
\end{frame}

% Fail for 25 bins BXL-Method
%Comparison with OffLine
\begin{frame} 
  \frametitle{863 Station, UUB}

  \begin{center}
    \begin{columns}
      \begin{column}{0.5\textwidth}
        \includegraphics[width=1.\textwidth]{../plots/uubChargeFromBlrSt863pmt2.pdf}
      \end{column}
    \end{columns}

    \begin{columns}
      \begin{column}{0.5\textwidth}
        \includegraphics[width=1.\textwidth]{../plots/uubChargeFromOffSt863pmt2.pdf}
      \end{column}
    \end{columns}
  \end{center}
  From December 2020, this method fit properly all histograms,
  including the ones for which OffLine-SdCalibrator algorithm
  fails.
\end{frame}


\begin{frame}
  \frametitle{$\mathrm{Q}^{\mathrm{pk}}_{\mathrm{VEM}}$
  Distribution for all UUB Stations (August, 2019, 2020, 2021)}
  \begin{center}
    \begin{columns}
      \begin{column}{.5\textwidth}
        \includegraphics[width=1.\textwidth]{../plots/QpkDistributionUbPmt1.pdf}
      \end{column}
      \begin{column}{.5\textwidth}
        \includegraphics[width=1.\textwidth]{../plots/QpkDistributionUbPmt2.pdf}
      \end{column}
    \end{columns}
    \begin{columns}
      \begin{column}{.5\textwidth}
        \includegraphics[width=1.\textwidth]{../plots/QpkDistributionUubPmt1.pdf}
      \end{column}
      \begin{column}{.5\textwidth}
        \includegraphics[width=1.\textwidth]{../plots/QpkDistributionUubPmt2.pdf}
      \end{column}
    \end{columns}
  \end{center}
\end{frame}

\begin{frame}
  \frametitle{$\mathrm{Q}^{\mathrm{pk}}_{\mathrm{VEM}}$
  Distribution for all UUB Stations (August, 2019, 2020, 2021)}
  \begin{center}
    \begin{columns}
      \begin{column}{.5\textwidth}
        \includegraphics[width=1.\textwidth]{../plots/QpkDistributionUbPmt3.pdf}
      \end{column}
      \begin{column}{.5\textwidth}
      \end{column}
    \end{columns}
    \begin{columns}
      \begin{column}{.5\textwidth}
        \includegraphics[width=1.\textwidth]{../plots/QpkDistributionUubPmt3.pdf}
      \end{column}
      \begin{column}{.5\textwidth}
      \end{column}
    \end{columns}
  \end{center}
\end{frame}

\begin{frame}
  \frametitle{Relative difference for
  $\mathrm{Q}^{\mathrm{pk}}_{\mathrm{VEM}}$, UUB}
  \begin{center}
    \begin{columns}
      \begin{column}{.5\textwidth}
        \includegraphics[width=1.\textwidth]{../plots/ubQpkCdasVsQpkOffAllStPMT1.pdf}
      \end{column}
      \begin{column}{.5\textwidth}
        \includegraphics[width=1.\textwidth]{../plots/uubQpkCdasVsQpkOffAllStPMT1.png}
      \end{column}
    \end{columns}
    \begin{columns}
      \begin{column}{.5\textwidth}
        \includegraphics[width=1.\textwidth]{../plots/ubRelDiffQpkUbMatchAllStPmt1.pdf}
      \end{column}
      \begin{column}{.5\textwidth}
        \includegraphics[width=1.\textwidth]{../plots/uubRelDiffQpkUbMatchAllStPmt1.pdf}
      \end{column}
    \end{columns}
  \end{center}
\end{frame}

\begin{frame}
  \frametitle{Relative difference for
  $\mathrm{Q}^{\mathrm{pk}}_{\mathrm{VEM}}$, UUB}
  \begin{center}
    \begin{columns}
      \begin{column}{.5\textwidth}
        \includegraphics[width=1.\textwidth]{../plots/ubQpkCdasVsQpkOffAllStPMT2.pdf}
      \end{column}
      \begin{column}{.5\textwidth}
        \includegraphics[width=1.\textwidth]{../plots/uubQpkCdasVsQpkOffAllStPMT2.png}
      \end{column}
    \end{columns}
    \begin{columns}
      \begin{column}{.5\textwidth}
        \includegraphics[width=1.\textwidth]{../plots/ubRelDiffQpkUbMatchAllStPmt2.pdf}
      \end{column}
      \begin{column}{.5\textwidth}
        \includegraphics[width=1.\textwidth]{../plots/uubRelDiffQpkUbMatchAllStPmt2.pdf}
      \end{column}
    \end{columns}
  \end{center}
\end{frame}

\begin{frame}
  \frametitle{Relative difference for
  $\mathrm{Q}^{\mathrm{pk}}_{\mathrm{VEM}}$, UUB}
  \begin{center}
    \begin{columns}
      \begin{column}{.5\textwidth}
        \includegraphics[width=1.\textwidth]{../plots/ubQpkCdasVsQpkOffAllStPMT3.pdf}
      \end{column}
      \begin{column}{.5\textwidth}
        \includegraphics[width=1.\textwidth]{../plots/uubQpkCdasVsQpkOffAllStPMT3.png}
      \end{column}
    \end{columns}
    \begin{columns}
      \begin{column}{.5\textwidth}
        \includegraphics[width=1.\textwidth]{../plots/ubRelDiffQpkUbMatchAllStPmt3.pdf}
      \end{column}
      \begin{column}{.5\textwidth}
        \includegraphics[width=1.\textwidth]{../plots/uubRelDiffQpkUbMatchAllStPmt3.pdf}
      \end{column}
    \end{columns}
  \end{center}
\end{frame}


\end{document}
