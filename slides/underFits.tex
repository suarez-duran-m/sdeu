\documentclass[aspectratio=169]{beamer}
\usepackage[T1]{fontenc}
\usepackage[utf8]{inputenc}
\usepackage{tikz}
\usepackage{tabularx}
\usepackage[font=scriptsize]{caption}
\usepackage{multirow}
\captionsetup[figure]{labelformat=empty}

\usetikzlibrary{tikzmark,shapes,arrows,backgrounds,fit,positioning}
\newcolumntype{C}{>{\centering\arraybackslash}X}
\newcolumntype{R}{>{\raggedleft\arraybackslash}X}

\addtobeamertemplate{navigation symbols}{}
{
	\insertframenumber{}
}

\beamertemplatenavigationsymbolsempty
\setbeamercolor{section in foot}{fg=white, bg=blue}
\setbeamercolor{subsection in foot}{fg=black, bg=white}
\setbeamerfont{footline}{size=\fontsize{6}{6}\selectfont}

\setbeamertemplate{footline}
{
  \leavevmode
  \hbox
  {
    \begin{beamercolorbox}
      [wd=.5\paperwidth,ht=2.5ex,dp=1.125ex,leftskip=.3cm,rightskip=.3cm]{subsection in foot}
    \end{beamercolorbox}
    
    \begin{beamercolorbox}
      [wd=.5\paperwidth,ht=2.5ex,dp=1.125ex,leftskip=.3cm,rightskip=.3cm plus1fil]{subsection in foot}
      \hfill
      \insertframenumber
      %\insertframenumber\,/\,\inserttotalframenumber
    \end{beamercolorbox}
  }
}


\title{UUB Charge and Peak histograms}
\author{
  Mauricio Su\'arez Dur\'an and Ioana~C.~Mari\c{s}
}
\institute{IIHE-ULB}

\titlegraphic{
  \begin{figure}[h]
    \centering
   %\includegraphics[width=5cm]{ulbLogo2.png}
    \hspace*{8.cm}
    \includegraphics[width=5.5cm]{iihe.jpeg}
  \end{figure}
}

\begin{document}
\begin{frame}
  \titlepage
\end{frame}


\begin{frame}
	\frametitle{UUB Charge and Peak histograms}
	\begin{itemize}
		\item Station studied: 863 1222 1219 1211 1740 1743 1221 1223 1217 1747 1741 1745 1818 1851 1729 1735 1746 1819 1791
		\item Data from CDAS.
		\item {\underline {Software CDAS, pre-production version.}}
	\end{itemize}
	\centering
	\includegraphics[width=.45\textwidth]{mapStations.pdf}
\end{frame}


\begin{frame}
  \frametitle{UUB Peak: Derivating histogram} %: Peak histogram Station863}
  \begin{figure}
    \centering
    \begin{tabularx}{\textwidth}{CC}
      \begin{tabular}{l}
        \includegraphics[width=.43\textwidth]{../plots/peakHisto863.pdf}
      \end{tabular}
      &
      \footnotesize
      \begin{tabular}{l}
        The algorithm:
        1. Smooth/FFT Histogram \\
        2. Derivating of smooth/FFT Histogram \\
        3. Identifing Fit range (Slope changes) \\
        4. Fitting \\
        Two function checked: \\
        - Exp. + Log-normal \\
        - 2nd order polinomium
      \end{tabular}
    \end{tabularx}

    \begin{tabularx}{\textwidth}{CC}
      \begin{tabular}{l}
        \includegraphics[width=.43\textwidth]{../plots/peakDerHisto863.pdf}
      \end{tabular}
      &
      \begin{tabular}{l}
        \includegraphics[width=.43\textwidth]{../plots/peakSmoothDerHisto863.pdf}
      \end{tabular}
    \end{tabularx}
  \end{figure}
\end{frame}


\begin{frame}
  \frametitle{UUB Peak: Fit and Residuals}
  \begin{figure}
    \centering
    \begin{tabularx}{\textwidth}{CC}
      \begin{tabular}{l}
        \includegraphics[width=.44\textwidth]{../plots/peakFittedHisto863.pdf}
      \end{tabular}
      &
      \begin{tabular}{l}
        \includegraphics[width=.44\textwidth]{../plots/peakFitPoly2863.pdf}
      \end{tabular}
      \\
      \begin{tabular}{l}
        \includegraphics[width=.44\textwidth]{../plots/peakFitResiduals863.pdf}
      \end{tabular}
      &
      \begin{tabular}{l}
        \includegraphics[width=.44\textwidth]{../plots/peakFitResidualsPoly2863.pdf}
      \end{tabular}
    \end{tabularx}
  \end{figure}
  The Exp.+Log-normal fit better than second order polinomium.
\end{frame}


%\begin{frame}
%  \frametitle{UUB Peak: Fit and Residuals smooth histogram}
%  \begin{figure}
%    \centering
%    \begin{tabularx}{\textwidth}{CC}
%      \begin{tabular}{l}
%        \includegraphics[width=.48\textwidth]{../plots/peakSmoothFittedHisto863.pdf}
%      \end{tabular}
%      &
%      \begin{tabular}{l}
%        \includegraphics[width=.48\textwidth]{../plots/peakSmoothFitPoly2863.pdf}
%      \end{tabular}
%      \\
%      \begin{tabular}{l}
%        \includegraphics[width=.48\textwidth]{../plots/peakSmoothFitResiduals863.pdf}
%      \end{tabular}
%      &
%      \begin{tabular}{l}
%        \includegraphics[width=.48\textwidth]{../plots/peakFitSmoothResidualsPoly2863.pdf}
%      \end{tabular}
%    \end{tabularx}
%  \end{figure}
%\end{frame}


\begin{frame}
  \frametitle{UUB Peak: Residuals distribution}
  \begin{figure}
    \centering
    \begin{tabularx}{\textwidth}{C}
      \begin{tabular}{l}
        \includegraphics[width=.7\textwidth]{../plots/peakResidualsDist863.pdf}
      \end{tabular}
%      &
%      \begin{tabular}{l}
%        \includegraphics[width=.5\textwidth]{../plots/peakSmoothResidualsDist863.pdf}
%      \end{tabular}
    \end{tabularx}
  \end{figure}
\end{frame}


\begin{frame}
  \frametitle{UUB Peak: applying all histograms St. 863, Chi and Prob. distributions}
  \begin{figure}
    \centering
    \begin{tabularx}{\textwidth}{CC}
      \begin{tabular}{l}
        \includegraphics[width=.48\textwidth]{../plots/uubChisDistPmtsPk863.pdf}
      \end{tabular}
      &
      \begin{tabular}{l}
        \includegraphics[width=.48\textwidth]{../plots/uubProbDistPmtsPk863.pdf}
      \end{tabular}
    \end{tabularx}
  \end{figure}
\end{frame}


% =========================================
% *** *** *** Charge Histograms *** *** ***

\begin{frame}
  \frametitle{UUB Charge: Derivating histogram} %: Peak histogram Station863}
  \begin{figure}
    \centering
    \begin{tabularx}{\textwidth}{CC}
      \begin{tabular}{l}
        \includegraphics[width=.49\textwidth]{../plots/chargeHisto863.pdf}
      \end{tabular}
      &
      \begin{tabular}{l}
        \includegraphics[width=.49\textwidth]{../plots/chargeHistoZoom863.pdf}
      \end{tabular}
      \\
      \begin{tabular}{l}
        \includegraphics[width=.49\textwidth]{../plots/chargeDerHisto863.pdf}
      \end{tabular}
      &
      \begin{tabular}{l}
        \includegraphics[width=.49\textwidth]{../plots/chargeSmoothDerHisto863.pdf}
      \end{tabular}
    \end{tabularx}
  \end{figure}
\end{frame}


\begin{frame}
  \frametitle{UUB Charge: Fit and Residuals}
  \begin{figure}
    \centering
    \begin{tabularx}{\textwidth}{CC}
      \begin{tabular}{l}
        \includegraphics[width=.44\textwidth]{../plots/chargeFittedHisto863.pdf}
      \end{tabular}
      &
      \begin{tabular}{l}
        \includegraphics[width=.44\textwidth]{../plots/chargeFitPoly2863.pdf}
      \end{tabular}
      \\
      \begin{tabular}{l}
        \includegraphics[width=.44\textwidth]{../plots/chargeFitResiduals863.pdf}
      \end{tabular}
      &
      \begin{tabular}{l}
        \includegraphics[width=.44\textwidth]{../plots/chargeFitResidualsPoly2863.pdf}
      \end{tabular}
    \end{tabularx}
  \end{figure}
  The Exp.+Log-normal fit better than second order polinomium.
\end{frame}


%\begin{frame}
%  \frametitle{UUB Charge: Fit and Residuals smooth hitogram}
%  \begin{figure}
%    \centering
%    \begin{tabularx}{\textwidth}{CC}
%      \begin{tabular}{l}
%        \includegraphics[width=.48\textwidth]{../plots/chargeFFTFitLogNorm863.pdf}
%      \end{tabular}
%      &
%      \begin{tabular}{l}
%        \includegraphics[width=.48\textwidth]{../plots/chargeFFTFitPoly2863.pdf}
%      \end{tabular}
%      \\
%      \begin{tabular}{l}
%        \includegraphics[width=.48\textwidth]{../plots/chargeFFTLogNormResiduals863.pdf}
%      \end{tabular}
%      &
%      \begin{tabular}{l}
%        \includegraphics[width=.48\textwidth]{../plots/chargeFitFFTResidualsPoly2863.pdf}
%      \end{tabular}
%    \end{tabularx}
%  \end{figure}
%\end{frame}


\begin{frame}
  \frametitle{UUB Charge: Residuals distribution}
  \begin{figure}
    \centering
    \begin{tabularx}{\textwidth}{C}
      \begin{tabular}{l}
        \includegraphics[width=.7\textwidth]{../plots/chargeResidualsDist863.pdf}
      \end{tabular}
    \end{tabularx}
  \end{figure}
\end{frame}


\begin{frame}
  \frametitle{UUB Charge: applying all histograms St. 863, Chi and Prob. distributions}
  \begin{figure}
    \centering
    \begin{tabularx}{\textwidth}{CC}
      \begin{tabular}{l}
        \includegraphics[width=.48\textwidth]{../plots/uubChisDistPmtsCh863.pdf}
      \end{tabular}
      &
      \begin{tabular}{l}
        \includegraphics[width=.48\textwidth]{../plots/uubProbDistPmtsCh863.pdf}
      \end{tabular}
    \end{tabularx}
  \end{figure}
\end{frame}


% ===================
% *** *** AoP *** ***

\begin{frame}
  \frametitle{UUB AoP Station 863: Peak, Charge and AoP}
  \begin{figure}
    \centering
    \begin{tabularx}{\textwidth}{CC}
      \begin{tabular}{l}
        \includegraphics[width=.48\textwidth]{../plots/uubPeakDistPmts863.pdf}
      \end{tabular}
      &
      \begin{tabular}{l}
        \includegraphics[width=.48\textwidth]{../plots/uubChargeDistPmts863.pdf}
      \end{tabular}
      \\
      \begin{tabular}{l}
        \includegraphics[width=.48\textwidth]{../plots/uubAoPSt863Pmts.pdf}
      \end{tabular}
    \end{tabularx}
  \end{figure}
\end{frame}

% ========================================
% *** *** *** For Station 1740 *** *** ***

\begin{frame}
  \frametitle{UUB Peak Station 1740: Chi and Prob. distributions all histograms}
  \begin{figure}
    \centering
    \begin{tabularx}{\textwidth}{CC}
      \begin{tabular}{l}
        \includegraphics[width=.48\textwidth]{../plots/uubChisDistPmtsPk1740.pdf}
      \end{tabular}
      &
      \begin{tabular}{l}
        \includegraphics[width=.48\textwidth]{../plots/uubProbDistPmtsPk1740.pdf}
      \end{tabular}
    \end{tabularx}
  \end{figure}
  For $\chi^2 / \mathrm{ndf}$ plot, all histograms with $\chi^2 / \mathrm{ndf}$ 
  bigger than 6 are counted as 6.
\end{frame}


\begin{frame}
  \frametitle{UUB Charge Station 1740: Chi and Prob. distributions all histograms}
  \begin{figure}
    \centering
    \begin{tabularx}{\textwidth}{CC}
      \begin{tabular}{l}
        \includegraphics[width=.48\textwidth]{../plots/uubChisDistPmtsCh1740.pdf}
      \end{tabular}
      &
      \begin{tabular}{l}
        \includegraphics[width=.48\textwidth]{../plots/uubProbDistPmtsCh1740.pdf}
      \end{tabular}
    \end{tabularx}
  \end{figure}
  For $\chi^2 / \mathrm{ndf}$ plot, all histograms with $\chi^2 / \mathrm{ndf}$ 
  bigger than 6 are counted as 6.
\end{frame}


\begin{frame}
  \frametitle{UUB Charge Station 1740: Failed fit}
  \begin{figure}
    \centering
    \begin{tabularx}{\textwidth}{CC}
      \begin{tabular}{l}
        \includegraphics[width=.48\textwidth]{../plots/uubPeakFailHisto1740Pmt1.pdf}
      \end{tabular}
      &
      \begin{tabular}{l}
        \includegraphics[width=.48\textwidth]{../plots/uubChargeFailHisto1740Pmt1.pdf}
      \end{tabular}
      \\
      \begin{tabular}{l}
        \includegraphics[width=.48\textwidth]{../plots/uubPeakFailHisto1740Pmt3.pdf}
      \end{tabular}
    \end{tabularx}
  \end{figure}
\end{frame}


% ===================
% *** *** AoP *** ***

\begin{frame}
  \frametitle{UUB AoP Station 1740: Peak, Charge and AoP}
  \begin{figure}
    \centering
    \begin{tabularx}{\textwidth}{CC}
      \begin{tabular}{l}
        \includegraphics[width=.48\textwidth]{../plots/uubPeakDistPmts1740.pdf}
      \end{tabular}
      &
      \begin{tabular}{l}
        \includegraphics[width=.48\textwidth]{../plots/uubChargeDistPmts1740.pdf}
      \end{tabular}
      \\
      \begin{tabular}{l}
        \includegraphics[width=.48\textwidth]{../plots/uubAoPSt1740Pmts.pdf}
      \end{tabular}
    \end{tabularx}
  \end{figure}
\end{frame}



% =====================================
% *** *** *** Comparison UB *** *** ***

\begin{frame}
  \frametitle{UB Peak: Derivating histogram} %: Peak histogram Station863}
  \begin{figure}
    \centering
    \begin{tabularx}{\textwidth}{CC}
      \begin{tabular}{l}
        \includegraphics[width=.43\textwidth]{../plots/peakHisto863.pdf}
      \end{tabular}
      &
      \footnotesize
      \begin{tabular}{l}
        The algorithm:
        1. Smooth/FFT Histogram \\
        2. Derivating of smooth/FFT Histogram \\
        3. Identifing Fit range (Slope changes) \\
        4. Fitting \\
        Two function checked: \\
        - Exp. + Log-normal \\
        - 2nd order polinomium
      \end{tabular}
    \end{tabularx}

    \begin{tabularx}{\textwidth}{CC}
      \begin{tabular}{l}
        \includegraphics[width=.43\textwidth]{../plots/ubpeakDerHisto863.pdf}
      \end{tabular}
      &
      \begin{tabular}{l}
        \includegraphics[width=.43\textwidth]{../plots/ubpeakSmoothDerHisto863.pdf}
      \end{tabular}
    \end{tabularx}
  \end{figure}
\end{frame}


\begin{frame}
  \frametitle{UB Peak: Fit and Residuals}
  \begin{figure}
    \centering
    \begin{tabularx}{\textwidth}{CC}
      \begin{tabular}{l}
        \includegraphics[width=.44\textwidth]{../plots/ubpeakFittedHisto863.pdf}
      \end{tabular}
      &
      \begin{tabular}{l}
        \includegraphics[width=.44\textwidth]{../plots/ubpeakFitPoly2863.pdf}
      \end{tabular}
      \\
      \begin{tabular}{l}
        \includegraphics[width=.44\textwidth]{../plots/ubpeakFitResiduals863.pdf}
      \end{tabular}
      &
      \begin{tabular}{l}
        \includegraphics[width=.44\textwidth]{../plots/ubpeakFitResidualsPoly2863.pdf}
      \end{tabular}
    \end{tabularx}
  \end{figure}
  The Exp.+Log-normal fit better than second order polinomium.
\end{frame}


\begin{frame}
  \frametitle{UB Charge: Residuals distribution}
  \begin{figure}
    \centering
    \begin{tabularx}{\textwidth}{C}
      \begin{tabular}{l}
        \includegraphics[width=.7\textwidth]{../plots/ubpeakResidualsDist863.pdf}
      \end{tabular}
    \end{tabularx}
  \end{figure}
\end{frame}

% =========================================
% *** *** *** Charge Histograms *** *** ***

\begin{frame}
  \frametitle{UB Charge: Derivating histogram} %: Peak histogram Station863}
  \begin{figure}
    \centering
    \begin{tabularx}{\textwidth}{CC}
      \begin{tabular}{l}
        \includegraphics[width=.49\textwidth]{../plots/ubchargeHisto863.pdf}
      \end{tabular}
      &
      \begin{tabular}{l}
        \includegraphics[width=.49\textwidth]{../plots/ubchargeHistoZoom863.pdf}
      \end{tabular}
      \\
      \begin{tabular}{l}
        \includegraphics[width=.49\textwidth]{../plots/ubchargeDerHisto863.pdf}
      \end{tabular}
      &
      \begin{tabular}{l}
        \includegraphics[width=.49\textwidth]{../plots/ubchargeSmoothDerHisto863.pdf}
      \end{tabular}
    \end{tabularx}
  \end{figure}
\end{frame}


\begin{frame}
  \frametitle{UB Charge: Fit and Residuals}
  \begin{figure}
    \centering
    \begin{tabularx}{\textwidth}{CC}
      \begin{tabular}{l}
        \includegraphics[width=.44\textwidth]{../plots/ubchargeFittedHisto863.pdf}
      \end{tabular}
      &
      \begin{tabular}{l}
        \includegraphics[width=.44\textwidth]{../plots/ubchargeFitPoly2863.pdf}
      \end{tabular}
      \\
      \begin{tabular}{l}
        \includegraphics[width=.44\textwidth]{../plots/ubchargeFitResiduals863.pdf}
      \end{tabular}
      &
      \begin{tabular}{l}
        \includegraphics[width=.44\textwidth]{../plots/ubchargeFitResidualsPoly2863.pdf}
      \end{tabular}
    \end{tabularx}
  \end{figure}
  The Exp.+Log-normal fit better than second order polinomium.
\end{frame}


\begin{frame}
  \frametitle{UB Charge: Residuals distribution}
  \begin{figure}
    \centering
    \begin{tabularx}{\textwidth}{C}
      \begin{tabular}{l}
        \includegraphics[width=.7\textwidth]{../plots/ubchargeResidualsDist863.pdf}
      \end{tabular}
    \end{tabularx}
  \end{figure}
\end{frame}


































\end{document}
